\section{Introducción}

  En la actualidad la industria musical obtiene ganancias a través de la creación y divulgación de la música de manera física y digital [1], dejando que aficionados y emprendedores de la música no tengan oportunidad de avanzar en su carrera, ya sea por la falta de creatividad, tiempo y/o recursos, haciendo que la creación de nuevas letras para sus canciones sea un gran obstáculo.
  Teniendo esto en cuenta, vimos una oportunidad y haciendo uso de las nuevas tecnologías, decidimos desarrollar una aplicación web la cual permite la generación de letras, las cuales pueden ser usadas posteriormente para integrarlas con sus temas musicales, creando canciones.\\\\
  La generación de texto es una de las tareas más populares y desafiantes en el área de procesamiento del lenguaje natural. Recientemente, se han desarrollado una gran cantidad de trabajos (Generating Text with Recurrent Neural Networks [2], Convolutional Neural Networks for Sentence Classification [3]) los cuales propusieron generar texto utilizando redes neuronales recurrentes y redes neuronales convolucionales. Sin embargo, la mayoría de los trabajos actuales solo se enfocan en generar una o varias oraciones, ni siquiera un párrafo largo y mucho menos una canción completa.\\\\
  Las letras de canciones, como un tipo de texto, tienen algunas características propias, estas se constituyen de coros, estribillos, rimas, versos y en algunos casos, patrones de repetición. Se entiende por coro que se trata de una estrofa la cual se repite varias veces dentro de una composición, acentuando la idea más importante de la canción, el estribillo es un conjunto de palabras o frases con la que se inicia la composición y la cual se repite al final de una estrofa de esta. Mientras que el verso es la parte encargada de comenzar a desarrollar la idea a transmitir, trata de contarnos el tema de la canción. Estas características hacen que generar letras musicales sea mucho más difícil que generar textos normales.\\\\
  La generación de letras por medio de inteligencia artificial dado un estilo y un tema es un asunto poco trabajado. Por lo tanto, nos enfocamos en trabajar este problema. Estamos interesados en ver si el modelo propuesto puede aprender diferentes características en un género musical y generar letras que sean acorde a este.
  