
\documentclass[12pt, a4paper, titlepage]{report}
\usepackage[spanish]{babel}
\usepackage[utf8]{inputenc}
\usepackage[linesnumbered,lined,boxed,commentsnumbered]{algorithm2e}
\usepackage{enumitem,kantlipsum}
\usepackage{array}
% Códigos y codificación para caracteres en español
\usepackage{listings}
\usepackage{color}
\lstset{literate=
  {á}{{\'a}}1 {é}{{\'e}}1 {í}{{\'i}}1 {ó}{{\'o}}1 {ú}{{\'u}}1
  {Á}{{\'A}}1 {É}{{\'E}}1 {Í}{{\'I}}1 {Ó}{{\'O}}1 {Ú}{{\'U}}1
  {à}{{\`a}}1 {è}{{\`e}}1 {ì}{{\`i}}1 {ò}{{\`o}}1 {ù}{{\`u}}1
  {À}{{\`A}}1 {È}{{\'E}}1 {Ì}{{\`I}}1 {Ò}{{\`O}}1 {Ù}{{\`U}}1
  {ä}{{\"a}}1 {ë}{{\"e}}1 {ï}{{\"i}}1 {ö}{{\"o}}1 {ü}{{\"u}}1
  {Ä}{{\"A}}1 {Ë}{{\"E}}1 {Ï}{{\"I}}1 {Ö}{{\"O}}1 {Ü}{{\"U}}1
  {â}{{\^a}}1 {ê}{{\^e}}1 {î}{{\^i}}1 {ô}{{\^o}}1 {û}{{\^u}}1
  {Â}{{\^A}}1 {Ê}{{\^E}}1 {Î}{{\^I}}1 {Ô}{{\^O}}1 {Û}{{\^U}}1
  {œ}{{\oe}}1 {Œ}{{\OE}}1 {æ}{{\ae}}1 {Æ}{{\AE}}1 {ß}{{\ss}}1
  {ű}{{\H{u}}}1 {Ű}{{\H{U}}}1 {ő}{{\H{o}}}1 {Ő}{{\H{O}}}1
  {ç}{{\c c}}1 {Ç}{{\c C}}1 {ø}{{\o}}1 {å}{{\r a}}1 {Å}{{\r A}}1
  {€}{{\EUR}}1 {£}{{\pounds}}1
}
%%Appendix
\usepackage[toc,page]{appendix}
%%Tablas
\usepackage{tabularx}
%%otros
\usepackage{float}
\usepackage{subfig}
\usepackage{comment}
\usepackage{booktabs}% http://ctan.org/pkg/booktabs
\newcommand{\tabitem}{~~\llap{\textbullet}~~}
%%Imágenes
\usepackage{graphicx}
%%Colores de texto 
\usepackage{xcolor}
\usepackage{colortbl}
%%Links
\usepackage[hidelinks]{hyperref}
%%Comentarios
\usepackage{verbatim}
%%PARA IMÁGENES EN LÍNEA
%\usepackage[english]{babel}

%%Acrónimos
\usepackage[acronym]{glossaries}


%%código
% \usepackage{listings}
\usepackage{minted}
\newcommand*{\img}[1]{%
    \raisebox{-.3\baselineskip}{%
        \includegraphics[
        height=2.5cm,
        width=3cm,
        keepaspectratio,
        ]{#1}%
    }%
}

%---------------------------GLOSARIO----------------------------%
\usepackage{glossaries}

\makeglossaries
\newglossaryentry{Convolucional}
{
    name=Convolucional,
    description={Es un operador matemático el cual transforma dos funciones en una tercera función.\cite{refRAE}}
}
\newglossaryentry{Recurrente}{
    name=Recurrente,
    description={Algo que vuelve a ocurrir o a aparecer pasado un intervalo.\cite{refRAE}}
}
\newglossaryentry{Reconocimiento}{
    name=Reconocimiento,
    description={Examinar algo o a alguien para así poder conocer su identidad, estado, naturaleza y circunstancias.  Distinguir o identificar a una persona o una cosa entre varas por una serie de características.\cite{refOxfordLex}}
}
\newglossaryentry{Aprende}{
    name=Aprende,
    description={Adquirir el conocimiento de algo por medio del estudio, análisis o de la experiencia.\cite{refRAE}}
}
\newglossaryentry{Sistema}{
    name=Sistema,
    description={Conjunto de elementos con relaciones de interacción e interdependencia las cuales le confieren entidad propia al formar un todo unificado.\cite{refSistema}}
}
\newglossaryentry{Estado}{
    name=Estado,
    description={Situación en que se encuentra alguien o algo, y en especial cada uno de sus sucesivos modos de ser o estar.\cite{refRAE}}
}
\newglossaryentry{Armonia}{
    name=Armonía,
    description={Unión y combinación de sonidos simultáneos y diferentes, pero acordes.\cite{refRAE}}
}
\newglossaryentry{Melodia}{
    name=Melodía,
    description={Es una sucesión lineal ordenada y coherente de sonidos musicales los cuales forman una unidad estructurada con un sentido musical, independiente del acompañamiento. \\Sucesión de sonidos que por su manera de combinarse resulta agradable de oír.\cite{refOxfordLex}}
}
\newglossaryentry{Estrofa}{
    name=Estrofa,
    description={Conjunto de versos que generalmente se ajustan a una medida y a un ritmo determinado y constante.\cite{refRAE}}
}
\newglossaryentry{Acordes}{
    name=Acordes,
    description={Conjunto de notas musicales sonando al mismo tiempo.\cite{refRAE}}
}
\newglossaryentry{Estocastico}{
    name=Estocástico,
    description={Que está sometido al azar y que es un objeto de análisis estadístico.\cite{refRAE}}
}
\newglossaryentry{Maquinas de Estados Finitos}{
    name=Máquinas de Estados Finitos,
    description={Conocidas también como Finite State Machines por su traducción en inglés, nos sirven para realizar procesos bien definidos en un tiempo discreto. Reciben una entrada, realizan un proceso y nos entregan una salida.\cite{refMaquinasFinitas}}
}

%------------------FINAL DE GLOSARIO-----------------------%

%---------------------------ACRÓNIMOS----------------------------%

\newacronym{orm}{ORM}{Object Relational Manager}
\newacronym{wsgi}{WSGI}{Web Server Gateway Interface}
\newacronym{dbms}{DBMS}{Sistema de gestión de base de datos}
\newacronym{aws}{AWS}{Amazon Web Services}

%------------------FINAL DE ACRÓNIMOS-----------------------%

%------------------ESTABLECER COLORES----------------------%
\definecolor{guindapoli}{RGB}{102, 0, 51}
\definecolor{azulescom}{RGB}{0, 0, 102}
\definecolor{azulclaro}{RGB}{222, 232, 255}
\definecolor{azulfuerte}{RGB}{60, 150, 250}
%----------------------------COLORES PARA CÓDIGO -----%
\definecolor{commentcolor}{RGB}{ 192, 192, 192 }
\definecolor{backcolour}{RGB}{ 249, 249, 249 }

%------------------FIN DE COLORES------------------%

%ESTILO DE CÓDIGO
\lstdefinestyle{codeStyle}{
  backgroundcolor=\color{backcolour},   commentstyle=\color{commentcolor},
  keywordstyle=\color{guindapoli},
  numberstyle=\tiny\color{azulescom},
  stringstyle=\color{azulfuerte},
  basicstyle=\ttfamily\footnotesize,
  breakatwhitespace=false,         
  breaklines=true,                 
  captionpos=b,                    
  keepspaces=true,                 
  numbers=left,                    
  numbersep=5pt,                  
  showspaces=false,                
  showstringspaces=false,
  showtabs=false,                  
  tabsize=3
}
\definecolor{delim}{RGB}{20,105,176}

\lstdefinelanguage{json}{
    basicstyle=\normalfont\ttfamily,
    numbers=left,
    numberstyle=\tiny\color{azulescom},
    stringstyle=\color{azulfuerte},
    stepnumber=1,
    numbersep=8pt,
    showstringspaces=false,
    breaklines=true,
    backgroundcolor=\color{backcolour},
    literate=
     *{0}{{{\color{azulfuerte}0}}}{1}
      {1}{{{\color{azulfuerte}1}}}{1}
      {2}{{{\color{azulfuerte}2}}}{1}
      {3}{{{\color{azulfuerte}3}}}{1}
      {4}{{{\color{azulfuerte}4}}}{1}
      {5}{{{\color{azulfuerte}5}}}{1}
      {6}{{{\color{azulfuerte}6}}}{1}
      {7}{{{\color{azulfuerte}7}}}{1}
      {8}{{{\color{azulfuerte}8}}}{1}
      {9}{{{\color{azulfuerte}9}}}{1}
      {:}{{{\color{guindapoli}{:}}}}{1}
      {,}{{{\color{guindapoli}{,}}}}{1}
      {\{}{{{\color{guindapoli}{\{}}}}{1}
      {\}}{{{\color{guindapoli}{\}}}}}{1}
      {[}{{{\color{guindapoli}{[}}}}{1}
      {]}{{{\color{guindapoli}{]}}}}{1},
}

\lstset{style=codeStyle}


\definecolor{lightgray}{rgb}{.9,.9,.9}
\definecolor{darkgray}{rgb}{.4,.4,.4}
\definecolor{purple}{rgb}{0.65, 0.12, 0.82}

\lstdefinelanguage{JavaScript}{
  keywords={typeof, new, true, false, catch, function, return, null, catch, switch, var, if, in, while, do, else, case, break, let, continue},
  keywordstyle=\color{blue}\bfseries,
  ndkeywords={class, export, boolean, throw, implements, import, this},
  ndkeywordstyle=\color{darkgray}\bfseries,
  identifierstyle=\color{black},
  sensitive=false,
  comment=[l]{//},
  morecomment=[s]{/*}{*/},
  commentstyle=\color{purple}\ttfamily,
  stringstyle=\color{red}\ttfamily,
  morestring=[b]',
  morestring=[b]"
}

\lstset{
   language=JavaScript,
   backgroundcolor=\color{lightgray},
   extendedchars=true,
   basicstyle=\footnotesize\ttfamily,
   showstringspaces=false,
   showspaces=false,
   numbers=left,
   numberstyle=\footnotesize,
   numbersep=9pt,
   tabsize=2,
   breaklines=true,
   showtabs=false,
   captionpos=b
}
% Rename Appendice to Anexos
\renewcommand\appendixpagename{Anexos}
\renewcommand\appendixtocname{Anexos}

\begin{document}
	%PARA QUE DETECTE HASTA SUBSUBSECTION
	\setcounter{secnumdepth}{3}
	
%   %   %   %   %   %   %   %   %   %
%		                       					  %
%   			PORTADA 				  %
%                               				  %
%   %   %   %   %   %   %   %   %	%
	\begin{titlepage}	
		
		\newcommand{\HRule}{\rule{\linewidth}{0.5mm}}									%%%\left
 																					%%%
\begin{minipage}{0.48\textwidth} \begin{flushleft}
\includegraphics[scale = 0.10]{imagenes/MarcoTeorico/logoescom.png}
\end{flushleft}\end{minipage}
\begin{minipage}{0.48\textwidth} \begin{flushright}
\includegraphics[scale = 0.55]{imagenes/MarcoTeorico/logoipn.png}
\end{flushright}\end{minipage}

													 								%%%
\vspace*{.25cm}								%%%
		
		\begin{center}
			
			\begin{LARGE}
				\textcolor{guindapoli}{INSTITUTO POLITÉCNICO NACIONAL}\\
			\end{LARGE}	
			
			\vspace*{0.2in}
			
			\begin{Large}
				\textcolor{azulescom}{ESCUELA SUPERIOR DE CÓMPUTO}\\
			\end{Large}		
			
			\vspace*{0.4in}
			
			\begin{large}
				Trabajo Terminal.\\
			\end{large}
			
			\vspace*{0.2in}
			
			\begin{Large}
				\textbf{Generador de versos musicales en el idioma
inglés por medio de procesamiento de lenguaje
natural y redes neuronales}\\
			\end{Large}
			
			\vspace*{0.2in}
			
			\begin{large}
				TT2020-B002.\\
			\end{large}
			
			\vspace*{0.2in}
			
			\rule{80mm}{.1mm}\\
			\vspace*{0.1in}
			
			\begin{large}
				\begin{center}
					\textbf{Presentan}:\\
					Espinosa de los Monteros Lechuga Jaime Daniel\\
					Nava Romo Edgar Adrián\\
					Salgado Gómez Alfredo Emilio\\
				\end{center}
			\end{large}
			
			\begin{large}
				\textbf{Directores}:\\
				Olga Kolesnikova\\
				Ariel López Rojas\\
			\end{large}
			
		\end{center}
	
	\end{titlepage}

%   %   %   %   %   %   %   %   %   %
%		                       					  %
%   				ÍNDICE 				  	  %
%                               				  %
%   %   %   %   %   %   %   %   %	%
	\begin{appendices}
		\renewcommand*\contentsname{{\textcolor{azulescom}{Índice.}}}
		\tableofcontents
		\newpage
		%%índice de figuras
		\renewcommand*\listfigurename{{\textcolor{azulescom}{Índice de figuras.}}}
		\listoffigures
		\newpage
		%%Índice de tablas
		\newpage
		\renewcommand*\listtablename{{\textcolor{azulescom}{Índice de cuadros.}}}
		\listoftables
		
		\newpage
		\renewcommand*\glossaryname{{\textcolor{azulescom}{Glosario.}}}
		\printglossary
		\renewcommand*\glossaryname{{\textcolor{azulescom}{Acrónimos.}}}
		\printglossary[title={\textcolor{azulescom}{Acrónimos.}}, type=\acronymtype]
	\end{appendices}
	
   \textcolor{guindapoli}{\part{Trabajo Terminal I}}
    
    \renewcommand\thechapter{\arabic{chapter}}
    \renewcommand{\appendixname}{Capítulo}
    \renewcommand{\lstlistingname}{C\'odigo}
    \renewcommand{\thepart}{}
    \renewcommand{\partname}{}
    %   %   %   %   %   %   %   %   %   %
    %		           Capítulo 1  				 %
    %   			Introducción 			   %
    %                               				  %
    %   %   %   %   %   %   %   %   %	%
    \chapter{\textcolor{azulescom}{Introducción}}
    La industria musical obtiene ganancias a través de la creación y divulgación de la música de manera física y digital (Bourreau and Gensollen 2006 \cite{Bourreau_and_Gensollen}), dejando que aficionados y emprendedores de la música no tengan oportunidad de avanzar en su carrera por falta de creatividad, tiempo y/o recursos, haciendo que la creación de nuevas letras para sus canciones sea un gran obstáculo, nuestra propuesta implica la utilización de nuevas tecnologías que permitan la generación de letras para integrar con sus canciones. Esta es una de las tareas más populares y desafiantes en el área de procesamiento del lenguaje natural. Hay una gran cantidad de trabajos (Generating Text with Recurrent Neural Networks \cite{Generating_Text_with_RNN}, Convolutional Neural Networks for Sentence Classification\cite{CNN_for_Sentence_Classification}) que proponen generar texto utilizando redes neuronales recurrentes y/o convolucionales. Sin embargo, la mayoría de los trabajos actuales solo se enfocan en generar una o varias oraciones, ni siquiera un párrafo largo y mucho menos una canción completa.\par
    
    Las letras de canciones, como un tipo de texto, tienen algunas características propias, estas se constituyen de rimas, versos, coros y en algunos casos, patrones de repetición. Coro se refiere a la parte de una canción que se repite sin modificaciones dentro de la misma después de un verso, mientras que en el verso suelen cambiar una o varias líneas que lo componen. Estas características particulares hacen que generar letras musicales sea mucho más difícil que textos normales.\par
    
    La mayoría de las investigaciones actuales sobre generación de letras vienen con muchas condiciones, como dar una pieza de melodía (Automatic Generation of Melodic Accompaniments for Lyrics \cite{Automatic_Generation_of_Melodic_Accompaniments_for_Lyrics}), o solo generar un tipo específico de letra (Conditional Rap Lyrics Generation with Denoising Autoencoders \cite{Conditional_Rap_Lyrics_Generation}). Sin embargo, la generación de letras por medio de inteligencia artificial, dado un estilo y tema en particular, ha sido muy poco trabajado y es debido a esto que planeamos centrarnos en este nuevo problema. Estamos interesados en ver si el modelo propuesto puede aprender diferentes características de un género musical y generar letras que sean acorde a este. Actualmente, en el mercado se encuentran cuatro aplicaciones web que tienen una funcionalidad similar a la propuesta en este Trabajo Terminal:\par
    
    \begin{itemize}
    	\item These lyrics do not exist.
    	\item Bored humans - lyrics\_generator.
    	\item DeepBeat.
    	\item Premium Lyrics.
    \end{itemize}
\newpage
	En el cuadro 1 que se presenta a continuación, se muestran las características de aplicaciones web similares y comparándolas con nuestra propuesta:

	\begin{table}[hbt!]\caption{Resumen de productos similares comparados con nuestra propuesta}% title of Table
	\centering % used for centering table
	\resizebox{15cm}{!} {
		\begin{tabular}{c c c}% centered columns (3 columns)
			\hline\hline                        %inserts double horizontal lines
			Software & Características & Precio en el mercado \\ [0.5ex]% inserts table %heading
			\hline                  % inserts single horizontal line
			\begin{tabular}[c]{@{}l@{}}These Lyrics do not Exist\end{tabular}
			& \begin{tabular}[c]{@{}l@{}}Aplicación web que genera\\letras completamente originales\\de varios temas, hace\\uso de IA para generar coros\\y versos originales; se puede\\escoger el tema principal de\\la letra, género musical e incluso\\el estado de ánimo al\\que iría dirigido.\end{tabular}
			& \begin{tabular}[c]{@{}l@{}}Gratuito (Contiene Anuncios)\end{tabular} \\% inserting body of the table 
			\hline
			\begin{tabular}[c]{@{}l@{}}Boredhumans \- lyrics\_generator
			\end{tabular}
			& \begin{tabular}[c]{@{}l@{}}Aplicación web en el que la\\IA fue entrenada con una\\base de datos con miles de\\letras para generar letras de\\canciones totalmente nuevas.\\La letra que crea es única\\y no una copia de alguna\\que exista actualmente, sin\\embargo, no permite customizar\\la letra.\end{tabular}
			& \begin{tabular}[c]{@{}l@{}}Gratuito\end{tabular} \\
			\hline
			\begin{tabular}[c]{@{}l@{}}DeepBeat\end{tabular}
			& \begin{tabular}[c]{@{}l@{}}Aplicación web que por medio\\de IA genera letras de\\música enfocada principalmente\\en el género rap. Si\\una línea no es del agrado\\se puede sustituir por alguna\\de las otras propuestas\\de las que ofrece.\end{tabular}
			& \begin{tabular}[c]{@{}l@{}}Gratuito\end{tabular} \\
			\hline
			\begin{tabular}[c]{@{}l@{}}Premium Lyrics\end{tabular}
			& \begin{tabular}[c]{@{}l@{}}Aplicación web que proporciona\\versos compuestos\\en distintos idiomas por\\artistas independientes que\\se escogen manualmente de\\acuerdo a su originalidad y\\calidad.\end{tabular}
			& \begin{tabular}[c]{@{}l@{}}3\$ a 75\$ Dólares por letra musical\end{tabular} \\
			\hline
			\begin{tabular}[c]{@{}l@{}}Nuestra propuesta\end{tabular}
			& \begin{tabular}[c]{@{}l@{}}Aplicación web que haciendo uso\\ de una IA va a generar\\letras musicales originales\\a partir de un género\\musical en exclusivo, lo que\\asegurará un resultado nal\\con coros y versos distintos\\cada vez que se utilice.\end{tabular}
			& \begin{tabular}[c]{@{}l@{}}Gratuito\end{tabular} \\
			[1ex]      % [1ex] adds vertical space
			\hline%inserts single line
		\end{tabular}\label{table:nonlin}% is used to refer this table in the text
	}
\end{table}

\section{Objetivos.}
	\subsection{Objetivo general.}
	Crear una herramienta de apoyo para estudiantes o aficionados interesados en este rubro que se les dificulte componer nuevas letras musicales de un solo género musical debido a la carencia de creatividad, la falta de conocimientos en la estructura del género o que no tengan inspiración suficiente para poder crear nuevas canciones, esto con el fin de impulsar la carrera de futuros artistas en la industria musical que no tengan los suficientes recursos para poder contratar servicios particulares de compositores.
	
	\subsection{Objetivos particulares.}
		\begin{itemize}
			\item Generar un conjunto de datos (dataset) con letras musicales de un género musical para efecto de entrenamiento de la red semántica.
    		\item Hacer uso de alguna herramienta de aprendizaje automático (machine learning) e implementar su uso en la nube para ayudar a procesar las letras musicales de un género en específico.
    		\item Implementar un módulo analizador de semántica para entrenar redes neuronales.
    		\item Desarrollar una interfaz web intuitiva en versión prototipo que utilice una aplicación web alojada en la nube para la visualización del verso musical generado a partir de un género.
    	\end{itemize}
		
    	\section{Justificación.}
    	El crear nuevas composiciones musicales puede llegar a ser muy difícil, estresante e incluso agotador para cualquier aficionado o incluso algunos expertos en este medio, esto se debe a la falta de creatividad y/o tiempo de quien lo quiera realizar \cite{What_about_the_music}. En ocasiones se pueden contratar servicios particulares para la producción de la letra de una canción, sin embargo, puede ser muy costoso y en ocasiones el resultado final no alcanza a llenar las expectativas de la inversión que se hace; por ende, se pretende crear una herramienta para estudiantes, aficionados o cualquier persona interesada en este rubro que se les dificulte componer nuevas letras musicales.\\
    	
    	Normalmente las letras musicales creadas por el humano, tienden a estar compuestas por patrones de acuerdo al género musical \cite{genero_musical_en_la_musica_popular}. Algunos ejemplos de estos patrones pueden ser las rimas, enunciados, frases cortas y que tengan una semántica correcta, estos pueden ser encontrados por medio de procesamiento de lenguaje natural y una investigación profunda en la composición de letras de estos géneros.\\
    	
    	Se eligió el idioma inglés debido a que existe una gran cantidad de bases de datos para procesar, al igual que herramientas y documentación para este idioma.\\
    	
    	Nos proponemos orientar esta solución en un entorno de nube, donde la información de configuración, servicios y datos necesarios pueden mantenerse de forma independiente a la implementación, facilitando la adaptación y flexibilidad de la plataforma.\\
    	
    	Nuestro proyecto ayudará al usuario utilizando herramientas como el procesamiento de lenguaje natural, redes neuronales, aprendizaje de máquina (machine learning) y servidores en la nube. Se hará uso de un conjunto de datos (dataset) y herramientas alojadas en la nube (Google Cloud Platform o Amazon Web Services) para procesar estos datos; se pretende utilizar un módulo que encuentre patrones por medio de redes neuronales para analizar la semántica mediante técnicas y herramientas ya existentes de procesamiento de lenguaje natural. Se van a realizar pruebas y experimentos con estas herramientas antes de la implementación (Bert \cite{Bert}, spaCy\cite{spaCy}) para poder generar versos. A su vez se va a desarrollar una interfaz web intuitiva en versión prototipo donde el usuario va a poder utilizar esta herramienta la cual le va a mostrar la letra musical que se va a generar en ese momento.\\
    	
    	A diferencia de los proyectos señalados en la Tabla 1 nuestra propuesta se va a centrar en generar letras musicales con métodos y tecnologías distintas a los que se usaron, esto es, aunque se utilicen los mismos géneros musicales se tendrán resultados completamente diferentes con propuestas distintas.\\
    	
    	En el desarrollo de este proyecto haremos uso de los conocimientos adquiridos durante el transcurso de la carrera. Se van a utilizar técnicas de diseño de proyectos aprendidas en el curso de Ingeniería de Software, se van a aplicar los conocimientos de programación adquiridos en unidades de aprendizaje como Inteligencia Artificial, Procesamiento de Lenguaje Natural, Web Application Development, Programación Orientada a Objetos, Análisis de Algoritmos, así como técnicas de construcción de documentos y análisis de semántica vistas en Análisis y Diseño orientado a Objetos, y Comunicación Oral y Escrita.
    	
	    \section{Metodolog\'ia.}
	    Para el desarrollo de este trabajo terminal se utilizará la metodología ágil Scrumban, que combina algunas partes de la 
	    metodología Scrum y Kanban, debido a que este es un proceso de gestión el cual reduce la complejidad en el desarrollo de 
	    productos para satisfacer las necesidades de los clientes. Además, permite trabajar colaborativamente de manera eficiente, 
	    es decir, en equipo, para obtener el mejor resultado posible.\\\\
	    Scrumban combina la estructura que utiliza Scrum con los métodos basados en flujo junto con la visualización de Kanban.
	    Es decir, se le permite a los equipos tener la agilidad de Scrum y la simplicidad de Kanban sin tener que utilizar las 
	    actualizaciones de roles y es más sencillo de adoptar.\\\\
	    En la siguiente tabla se pueden observar las principales diferencias entre las tres metodologías:
	    
	    \begin{table}[hbt!]\caption{Tabla comparativa de los distintos servidores web contemplados}% title of Table
	    	\centering % used for centering table
	    	\resizebox{15cm}{!} {
	    		\begin{tabular}{c c c c}% centered columns (3 columns)
	    			\hline\hline                        %inserts double horizontal lines
	    			& Scrum & Kanban & Scrumban \\ [0.5ex]% inserts table %heading
	    			\hline                  % inserts single horizontal line
	    			Procesos
	    			& \begin{tabular}[c]{@{}l@{}}Iterativo e incremental\\ desarrollando sprints\end{tabular}
	    			& \begin{tabular}[c]{@{}l@{}}Continuo\end{tabular}
	    			& \begin{tabular}[c]{@{}l@{}}Iterativo e incremental \\de forma continua\\ desarrollando iteraciones\end{tabular} \\% inserting body of the table 
	    			\hline
	    			Personas
	    			& \begin{tabular}[c]{@{}l@{}}Las personas son el\\ centro\end{tabular}
	    			& \begin{tabular}[c]{@{}l@{}}Las personas son el pilar\end{tabular}
	    			& \begin{tabular}[c]{@{}l@{}}Equipo motivado con \\personas como pilar\\ y en el centro\end{tabular}\\
	    			\hline
	    			Producto
	    			& Foco en la efectividad
	    			& \begin{tabular}[c]{@{}l@{}}Foco en la eficiencia\end{tabular}
	    			& \begin{tabular}[c]{@{}l@{}}Balance entre \\efectividad y eficiencia\end{tabular}\\
	    			\hline
	    			Organizaci\'on
	    			& \begin{tabular}[c]{@{}l@{}}Mejora continua del producto\end{tabular}
	    			& Mejora continua del proceso
	    			& \begin{tabular}[c]{@{}l@{}}Mejora continua \\del producto y \\del proceso\end{tabular}\\
	    			\hline
	    			Equipo
	    			& \begin{tabular}[c]{@{}l@{}}De 3 a 9 personas\end{tabular}
	    			& \begin{tabular}[c]{@{}l@{}}No hay limitaciones\end{tabular}
	    			& \begin{tabular}[c]{@{}l@{}}El equipo no requiere\\ de un número\\ específico de \\integrantes\end{tabular}\\
	    			\hline
	    			Roles
	    			& \begin{tabular}[c]{@{}l@{}}Scrum master\\Product owner\\Scrum team\end{tabular}
	    			& \begin{tabular}[c]{@{}l@{}}Pueden incluir \\especialistas o\\ integrantes generalizados\end{tabular}
	    			& \begin{tabular}[c]{@{}l@{}}No requiere un rol específico\end{tabular}\\
	    			[1ex]      % [1ex] adds vertical space
	    		\end{tabular}\label{table:VentajasServidor}% is used to refer this table in the text
	    	}
	    \end{table}
	    
	    Scrumban utiliza iteraciones y se monitorea con la ayuda de un tablero visual. Las reuniones para planificar se llevan a cabo cuando son necesarias para determinar las tareas a implementar hasta la próxima iteración. Para que estas iteraciones se mantengan cortas, se utiliza un límite de trabajo en progreso (WIP por sus siglas en inglés Work in Progress). Cuando WIP cae por debajo del nivel predeterminado, se establece un activador de planificación bajo demanda para que el equipo sepa cuándo planificar a continuación.\\
	    \\
	    Iteración\\
	    En Scrumban, las iteraciones son cortas para garantizar que el equipo pueda adaptarse al entorno cambiante durante el proyecto. La duración de estas iteraciones en este proyecto se medirán como máximo en lapsos de dos semanas.\\
	    \\
	    Priorización\\
	    La priorización se da de tal forma que las tareas más importantes se colocan en la parte superior de la tabla de planificación seguidas por las tareas menos importantes.\\
	    Antes de llegar al tablero las tareas deben pasar por 3 etapas donde se van depurando las tareas a realizar para largo plazo (1 año), medio plazo (6 meses) y corto plazo (3 meses) siendo de esta última de donde salen las tareas más claras que se pueden completar y que ganan mayor prioridad para la próxima iteración.\\
	    \\
	    Principio de Elección\\
	    Cada miembro del equipo elige solo qué tarea de la sección “Tareas Pendientes” va a completar a continuación.\\
	    \\
	    Congelación de Funciones\\
	    Se utiliza en Scrumban cuando se acerca la fecha límite del proyecto, significando que sólo pueden trabajar sobre las tareas previamente pensadas sin cabida para implementar nuevas características.\\
	    \\
	    Triaje\\
	    Ocurre después de la congelación de funciones, y es el punto donde el gerente del proyecto decide cuáles de las características en desarrollo se completarán y cuáles quedarán sin terminar.\\
	    
	    \begin{figure}[H] \caption{Funcionamiento de la planificación por cubetas}
	    	\includegraphics[width=12cm]{./imagenes/Introduccion/scrumban_buckets.jpg}
	    	\centering 
	    \end{figure}
	    
        \section{Organización del documento}
            Para dar inicio a este trabajo terminal, presentamos de manera breve la estructura de este reporte, con el objetivo de que el lector pueda tener un mejor entendimiento del trabajo.
            

            \subsection{Capítulo 2. Marco teórico.}
                En esta parte del documento, se describen puntos esenciales de nuestro trabajo como definiciones, técnicas, herramientas y servicios, así como investigación realizada para llevar a cabo en la implementación dentro del trabajo. 
             
            \subsection{Capítulo 3. Análisis.}
                Dentro de este capitulo se analiza el estudio de factibilidad tanto t\'ecnico como operativo y económico con la finalidad de conocer los recursos necesarios para la elaboración de este trabajo terminal. Se mencionan resumidamente las herramientas a utilizar y se explica de manera general la arquitectura del sistema  y el diagrama de casos de uso general. Finalmente se muestra el análisis de los tres componentes que tenemos en el sistema.
             
            \subsection{Capítulo 4. Diseño.}
                En el cuarto capítulo, nos adentraremos en el desarrollo de los prototipos, es decir, se encuentran los diagramas pertinentes para poder modelar nuestro trabajo terminal y proceder a la etapa de implementación. En este capítulo se desarrollan y explican los siguientes diagramas: 'Casos de uso', 'Flujo', 'Flujo de datos', 'Clases', 'Secuencia' y 'Actividades' y se muestra la interfaz de usuario propuesta junto con los requisitos de diseño.
                
            \subsection{Capítulo 5. Desarrollo.}
                En este capítulo mostraremos lo que se ha desarrollado para este Trabajo Terminal I (Iteración I e Iteración II). Se muestra que el prototipo cumpla los requerimientos que se le impusieron en la sección de análisis.
            
            \subsection{Capítulo 6. Avances y trabajo por hacer.}
                En el último capítulo de este reporte, hablaremos de los avances que hemos logrado a lo largo de la asignatura de Trabajo Terminal I y además, exponemos el trabajo esperado para la asignatura de Trabajo Terminal II.
                
    %   %   %   %   %   %   %   %   %   %
    %		        Capítulo 2					%
    %   	MARCO TEÓRICO 				%
    %                               				  %
    %   %   %   %   %   %   %   %   %	%        
    
	\chapter{\textcolor{azulescom}{Marco teórico}}

%		\subsection{Github Student Pack}
%		Se solicitó Github Students Pack para obtener el dominio donde se hospeda la aplicación web por un periodo de doce meses sin ningún costo por medio de namecheap.com, para ello primero se ingresa a la página principal del servicio de {Github Students Pack} {https://education.github.com/pack}, se crea una cuenta junto con el correo proporcionado por el Instituto Politécnico Nacional.\par
%		\par ACOMODAR FOTOS DEL PROCESO CON EL PROCEDIMIENTO \par Una vez que la cuenta se activó, en la página web a un lado de este nos aparecerá el nombre de la institución educativa, en este caso, el Instituto Politécnico Nacional es la institución educativa que nos respalda. Y por último se nos pide agregar el que vamos a hacer en la plataforma.\par
%		\begin{figure}[H]
%			\includegraphics[width=12cm]{./imagenes/MarcoTeorico/GitHub/GitS1.png}
%			\centering 
%			\caption{Github realacionado con el correo institucional}
%		\end{figure}
%		Ya cuando enviamos la solicitud con nuestra información nos aparece un mensaje en el cual nos dicen que obtendremos una respuesta sobre nuestra solicitud en menos de 22 días.\par
%		\begin{figure}[H]
%			\includegraphics[width=12cm]{./imagenes/MarcoTeorico/GitHub/GitS2.png}
%			\centering 
%			\caption{Tiempo de espera de aprovación de la solicitud}
%		\end{figure}
%		Nunca se obtuvo una respuesta en forma de correo electronico por parte de Github, sino que simplemente a la cuenta con la que solicitamos este paquete de estudiante se le otorgo, este es posible saberlo, ya que entrando a la cuenta si checamos nuestro perfil, en la parte inferior aparecerá un icono de pro, con el cual se nos da el acceso solicitado.
%		\begin{figure}[H]
%			\includegraphics[width=4cm]{./imagenes/MarcoTeorico/GitHub/Githubpro.png}
%			\centering 
%			\caption{Activación del students pack}
%		\end{figure}
		\section{Inteligencia Artificial}
		Por Inteligencia Artificial se entiende a una simulación de procesos computacionales cognitivos para que simulen el comportamiento de una mente humana, estos comportamientos abarcan diferentes áreas de investigación, como el razonamiento, aprendizaje, percepción, comunicación y la capacidad de desarrollarse en entornos más complejos.
		“La inteligencia no es una dimensión única, sino un espacio profusamente estructurado de capacidades diversas para procesar la información. Del mismo modo, la Inteligencia Artificial utiliza muchas técnicas diferentes para resolver una gran variedad de tareas que se encuentran en todas partes.” \cite{Inteligencia_Artificial}\par
		En ciencias de la computación, el término de inteligencia artificial se refiere a cualquier inteligencia similar a la humana exhibida por una computadora, robot u otra máquina. En el uso popular, la inteligencia artificial hace referencia a la capacidad de una computadora o máquina para imitar las capacidades del cerebro humano, es decir, aprender de ejemplos y experiencias, reconocer objetos, comprender y responder al lenguaje, tomar decisiones, resolver problemas y combinar estas y otras capacidades para realizar funciones que un ser humano podría realizar. \cite{refInteligencia_Artificial2}\par
		Como objetivo, la Inteligencia Artificial pretende hacer un uso de los recursos tecnológicos para desarrollar modelos computacionales y poder resolver problemas del mundo real evolucionando constantemente.
	    \section{Redes neuronales}
	    Una red neuronal es una herramienta derivada de la inteligencia artificial que utiliza modelos matemáticos para ser utilizada como un mecanismo de predicción de texto.\\
	    El aprendizaje profundo, o deep learning, es un nombre o enfoque reciente de la inteligencia artificial llamada redes neuronales. Las redes neuronales son una forma de hacer que las computadoras aprendan, donde la computadora aprende a realizar alguna tarea analizando ejemplos de entrenamiento. Por lo general, estos ejemplos han sido etiquetados previamente.\par
		Modelado vagamente en el cerebro humano, una red neuronal consiste en miles de millones de nodos de procesamiento los cuales están densamente interconectados. En la actualidad las redes neuronales están organizadas como capas de nodos, y estas capas están “alimentadas hacia adelante” (feed-forward), es decir, la información que se mueve a través de ellas solo fluye en una sola dirección. Un solo nodo puede estar conectado a muchos nodos en capas inferiores de las cuales recibe información y a su vez, puede estar conectado a nodos en capas superiores a los cuales envía información.\par
		Cuando una red neuronal está siendo entrenada la información de entrenamiento pasa a alimentar las capas inferiores (capas de entrada), donde será procesada y pasada a capas posteriores donde será transformada hasta llegar a las capas superiores (capas de salida).\cite{refQueSonRedesNeu}

%        	\subsection{Tipos de Redes Neuronales}
%        	    Los diferentes tipos de redes neuronales usan diferentes principios para determinar sus reglas. Cada uno de los tipos de redes neuronales tienen sus propias debilidades y fortalezas, a continuación, describiremos algunas de las más utilizadas y sus aplicaciones.
%        	    \subsubsection{Feedforward Neural Network – Artificial Neuron}
%				Una de las redes neuronales más simples, donde la información pasa a través de diferentes nodos de entrada, hasta alcanzar los nodos de salida. En otras palabras, la información solo fluye en una sola dirección, generalmente se hace uso de esta red para tecnologías que requieran \Gls{Reconocimiento} de patrones o clasificación de objetos.\cite{refTiposRedesNeu3}\par
%				\begin{figure}[H]
%					\includegraphics[width=12cm]{./imagenes/MarcoTeorico/RedesN/Feedforward.png}
%					\centering 
%					\caption{Diagrama Red Neuronal Feedforward}
%				\end{figure}
%				\subsubsection{Radial Basis Function Neural Network}
%				Este tipo de red considera la distancia de cualquier punto en relación con el centro. Estas redes suelen estar constituidas por 2 capas. En la capa interior, donde la información se combina con la función de base radial, para luego, las salidas de esta capa se tomen en cuenta para calcular la próxima salida. Este tipo de redes suele implementarse en sistemas que requieran una restauración.\cite{refTiposRedesNeu2}\par
%				\begin{figure}[H]
%					\includegraphics[width=12cm]{./imagenes/MarcoTeorico/RedesN/Radial.png}
%					\centering 
%					\caption{Diagrama Red Neuronal Radial Basis Function}
%				\end{figure}
%			\subsubsection{Multilayer Perception}
%				Esta red cuenta con tres o más capas, las cuales están completamente conectadas, ya que cada uno de los nodos de cada capa se encuentra conectado con todos los nodos de la capa siguiente. Esta red es usada para clasificar información que no pueda ser separada de forma lineal como lo son el \Gls{Reconocimiento} de voz o las traducciones automáticas.\cite{refTiposRedesNeu2}\par
%				\begin{figure}[H]
%					\includegraphics[width=12cm]{./imagenes/MarcoTeorico/RedesN/Multilayer.png}
%					\centering 
%					\caption{Diagrama Red Neuronal Multilayer Perception}
%				\end{figure}
%				\subsubsection{Convolutional Neural Network}
%				Es una variación de la Multilayer Perception. Contiene una o más capas convolucionales, las cuales pueden estar completamente interconectadas o agrupadas. Antes de pasar el resultado a la siguiente capa, la capa \Gls{Convolucional} hace uso de una operación convolucional en la entrada. Como resultado de esta operación convolucional la red puede ser más profunda, pero con menos parámetros.\par
%				Gracias a esto las redes neuronales convolucionales han demostrado buenos resultados en el \Gls{Reconocimiento} de imágenes y video, procesamiento de lenguaje natural, así como para realizar recomendaciones.\cite{refTiposRedesNeu1}\par
%				\begin{figure}[H]
%					\includegraphics[width=12cm]{./imagenes/MarcoTeorico/RedesN/Convolutional.png}
%					\centering 
%					\caption{Diagrama Red Neuronal Convolucional}
%				\end{figure}
				\subsection{Neuronas}
				\subsubsection{Neurona Natural}
				La neurona es un tipo de célula que representa la unidad estructural y funcional del sistema nervioso. Su función consiste en transmitir información a través de impulsos nerviosos o químicos, desde un lugar del cuerpo hacia otro.\par
				Este impulso viaja siempre en el mismo sentido, es decir, llega a la neurona a través de las dendritas, se procesa en el soma y posteriormente se transmite al axón, las neuronas no están en contacto entre sí, existe un espacio de separación entre ellas denominado sinapsis o espacio sináptico. Cuando el impulso nervioso llega al extremo del axón se liberan neurotransmisores al espacio sináptico transformando la señal eléctrica en otra química que penetra en la dendrita de la neurona contigua, desencadenando un impulso eléctrico en la neurona receptora y se repite el proceso con la contigua.\cite{refNeuronaNat}
				\begin{figure}[H] \caption{Estructura de una neurona}
					\includegraphics[width=12cm]{./imagenes/MarcoTeorico/Neurona.jpg}
					\centering 
				\end{figure}			
				\subsubsection{Neurona Artificial}
				La complejidad de las neuronas reales se abstrae mucho cuando se realiza el modelado de neuronas artificiales. Estas consisten básicamente en entradas, que se multiplican por pesos (fuerza de las respectivas señales), y luego calculada por una función matemática que determina la activación de la neurona. Otra función (que puede ser la identidad) calcula la salida de la neurona artificial (a veces en dependencia de un cierto umbral). Las redes neuronales combinan neuronas artificiales para procesar información.\cite{refNeuronaArt}
				\begin{figure}[H] \caption{Neurona Artificial}
					\includegraphics[width=12cm]{./imagenes/MarcoTeorico/NeuronaArt.png}
					\centering 
				\end{figure}
				\subsection{Aprendizaje de las Redes Neuronales}
				Si observamos la naturaleza, podemos ver que los sistemas que pueden aprender son altamente adaptables. En su búsqueda por adquirir conocimientos, estos sistemas utilizan información del mundo exterior y modifican la información que ya han recopilado o modifican su estructura interna. Eso es exactamente lo que hacen las redes neuronales. Adaptan y modifican su arquitectura para aprender. Para ser más precisos, las redes neuronales cambian los pesos de las conexiones según la entrada y la salida deseada.\par
				¿Por qué tienen un peso?, bueno, si observamos la estructura de las redes neuronales, hay algunos componentes que podríamos cambiar, si queremos modificar su arquitectura. Por ejemplo, podríamos crear nuevas conexiones entre neuronas, o eliminarlas, o agregar y eliminar neuronas. Incluso podríamos modificar la función de entrada o la función de activación. Resulta que cambiar los pesos es el enfoque más práctico. Además, la mayoría de los otros casos podrían cubrirse cambiando los pesos. La eliminación de una conexión, por ejemplo, se puede hacer estableciendo el peso en 0. Y una neurona se puede eliminar si establecemos los pesos de todas sus conexiones en cero.\cite{refAprendizajeRedes}\par
				El entrenamiento es un proceso necesario para toda red neuronal, y es un proceso en el que la red se familiariza con el problema que necesita resolver. En la práctica, generalmente se tienen algunos datos recopilados en función de los cuales necesitamos crear nuestras predicciones, clasificación, o cualquier otro procesamiento. Estos datos se denominan conjunto de entrenamiento. Según el comportamiento durante el entrenamiento y la naturaleza del conjunto de entrenamiento, tenemos algunos tipos de aprendizajes:
				\begin{itemize}
					\item \textbf{Aprendizaje no supervisado: }el conjunto de entrenamiento solo contiene entradas. La red intenta identificar entradas similares y clasificarlas en ciertas categorías.
					\item \textbf{Aprendizaje reforzado: }el conjunto de entrenamiento contiene entradas, pero la red también recibe información adicional durante la formación. Lo que sucede es que una vez que la red calcula la salida para una de las entradas, proporcionamos información que indica si el resultado fue correcto o incorrecto y, posiblemente, la naturaleza del error que cometió la red.
					\item \textbf{Aprendizaje supervisado: }el conjunto de entrenamiento contiene entradas y salidas deseadas. De esta manera, la red puede verificar su salida calculada con la salida deseada y tomar las acciones pertinentes para reformular su cálculo.
				\end{itemize}
				\subsection{Recurrent Neural Network – Long Short Term Memory}
				Una red neuronal \Gls{Recurrente} es un tipo de red neuronal artificial donde la salida de alguna capa en particular es salvada y sirve para retroalimentar la entrada de esta capa, lo cual ayuda a predecir futuras salidas de esta.\par
				La primera capa esta formada de la misma manera que la Feedforward Neural Network, es decir, solo pasa la información que entra a la siguiente capa inmediata, posteriormente la siguiente capa con el paso del tiempo comenzará a retroalimentarse, pero manteniendo la propagación frontal. Haciendo uso de esta retroalimentación la capa en futuras operaciones puede realizar predicciones, si estas predicciones no son los resultados esperados, el \Gls{Sistema} \Gls{Aprende} y trabaja para corregir sus futuras predicciones.\cite{refTiposRedesNeu1}\par
				Se distinguen por su “memoria”, ya que toman información de entradas anteriores para influir en la entrada y salida actuales. Mientras que las redes neuronales profundas tradicionales asumen que las entradas y salidas son independientes entre sí, la salida de las redes neuronales recurrentes depende de los elementos anteriores dentro de la secuencia. Si bien los eventos futuros también serían útiles para determinar la salida de una secuencia dada, las redes neuronales recurrentes unidireccionales no pueden tener en cuenta estos eventos en sus predicciones.\cite{refTiposRedesNeu4}\par
				Estos algoritmos de aprendizaje profundo se utilizan comúnmente para problemas relacionados con la traducción de idiomas, el procesamiento del lenguaje natural, el reconocimiento de voz y los subtítulos de imágenes.
				\newpage
				\begin{figure}[H] \caption{Diagrama Red Neuronal Recurrente}
					\includegraphics[width=12cm]{./imagenes/MarcoTeorico/RedesN/Recurrente.png}
					\centering 
				\end{figure}				
				\subsection{Bert}
				\subsection{GPT-2}
%				\subsubsection{Modular Neural Network}
%				Esta red está constituida por un numero variado de redes, las cuales funcionan de manera independiente y realizan subtareas. Las diferentes redes, en realidad no interactúan entre si durante la ejecución del proceso. Trabajan de manera independiente para lograr una misma salida.\par
%				Como resultado, un largo y complejo proceso puede ser ejecutado rápido, separándolo en componentes individuales.\cite{refTiposRedesNeu2}\par
%				\begin{figure}[H]
%					\includegraphics[width=12cm]{./imagenes/MarcoTeorico/RedesN/Modular.png}
%					\centering 
%					\caption{Diagrama Red Neuronal Modular}
%				\end{figure}
                

        \section{Base de datos}
        Una base de datos es una colección organizada de información o datos, los cuales pertenecen a un mismo contexto, se encuentran almacenados de forma física o digital con la finalidad de realizar alguna consulta futura, ingreso de nuevos datos, actualización o eliminación de estos.\par
		Las bases de datos se componen de una o más tablas las cuales se encargan de guardar un conjunto de datos, y se dividen en columnas y filas.\par
		Una base de datos generalmente es manejada por un \acrfull{dbms}. En conjunto, los datos y el \acrshort{dbms}, junto con las aplicaciones asociadas a ellos, se les conoce como un sistema de base de datos.\cite{refQueEsBD}\par
			\subsection{Base de datos NoSQL o no relacionales}
			Estas bases permiten que los datos no estructurados y/o semiestructurados se almacenen y manipulen, a diferencia de la base de datos relacional donde se define como deben de componerse todos los datos insertados en esta. Generalmente los registros de este tipo de base de datos suelen almacenarse como un documento de tipo JSON.\par
%			\subsection{Tipos de bases de datos}
%				\subsubsection{Bases de datos estáticas}
%				Estas son de sólo lectura, utilizadas principalmente para almacenar datos históricos los cuales posteriormente se pueden utilizar para estudiar el comportamiento de un conjunto datos a través del tiempo, realizar proyecciones o tomar decisiones.\par
%				\subsubsection{Bases de datos dinámicas}
%				Estas son las que la información almacenada puede ser modificada con el tiempo, permitiendo operaciones como adición, actualización y eliminación de datos.\par
%				\subsubsection{Bases de datos jerárquicas}
%				Como su nombre su dice, almacenan la información en una estructura jerárquica, en este modelo la organización es similar a un árbol visto al revés, de donde un nodo padre se pueden tener varios nodos hijos, el nodo padre es llamado nodo raíz, y a los nodos que no tiene hijos se les conoce como hojas.\par
%				Estas bases de datos son útiles cuando se anejan grandes cantidades de información y datos que requieren compartirse.\par
%				\subsubsection{Bases de datos relacionales}
%				En esta se almacenan y se proporcionan acceso a puntos de datos relacionados entre sí. Esto gracias a que cada fila de la tabla es un registro con un ID único, las columnas de la tabla contienes atributos de los datos, y cada registro generalmente cuenta con un valor para cada atributo, lo que facilita el establecimiento de las relaciones entre los puntos de datos.\par
%				\subsubsection{Base de datos NoSQL o no relacionales}
%				Estas bases permiten que los datos no estructurados y/o semiestructurados se almacenen y manipulen, a diferencia de la base de datos relacional donde se define como deben de componerse todos los datos insertados en esta. Generalmente los registros de este tipo de base de datos suelen almacenarse como un documento de tipo JSON.\par
%				\subsubsection{Base de datos orientada a objetos}
%				En esta se tratan de almacenar los objetos completos (\Gls{Estado} y comportamiento), incorpora todos los conceptos del paradigma orientado a objetos (encapsulamiento, herencia, polimorfismo)\par
		  \newpage
	    \section{Canción}
	    De manera resumida se puede decir que una canción es una composición literaria, generalmente en verso, la cual puede ser acompañada con música para poder ser cantada.\cite{refEstructuraCancion1}\par
			\subsection{Elementos de una canción}
			Los elementos que conforman una canción son los siguientes:
			\subsection{Introducción}
			Generalmente es una parte única la cual aparece al inicio de una canción, acompañada de una \Gls{Armonia} o \Gls{Melodia} compuesta solo para este inicio. El objetivo principal de la introducción es captar la atención y generar un ambiente.\cite{refEstructuraCancion2}\par
			\subsection{Verso}
			Está encargado de comenzar a desarrollar la idea a transmitir, trata de contarnos de que va a ser la canción, y ya cuenta con una \Gls{Armonia}  bien establecida.\par
			\subsection{Pre-estribillo}
			Es un arreglo que permite realizar una transición, su función principal es conectar el verso con el estribillo. También ayuda a evitar que el estribillo se estanque en la monotonía.\par
			\subsection{Estribillo}
			Es una \Gls{Estrofa} la cual se repite varias veces dentro de una composición. La función principal del estribillo es destacar la idea de la canción tanto en la letra como en lo musical. El estribillo es considerado la parte más importante de la canción y en algunas ocasiones es repetido al inicio y al final de la misma.\par
			\subsection{Puente}
			Es un interludio la cual conecta dos partes de una canción, permitiendo construir una \Gls{Armonia}  entre ellas, suele ser usado para llevar a la canción a su clímax, para prepararlo para el desarrollo final de la canción.\par
			\subsection{Cierre}
			Hay distintas formas de terminar o concluir una canción, puede ser un quiebre brusco generado por un silencio repentino o por una sucesión de \Gls{Acordes}. Pero la forma más común es haciendo uso de la repetición de un estribillo.\par
			\subsection{Estructura de una canción}
			La estructura mínima de una canción está compuesta de:\par
			\begin{itemize}
			  \item Verso
			  \item Estribillo
			  \item Verso
			  \item Estribillo
			\end{itemize}\par
			La estructura más usada en una canción es la siguiente
			\begin{itemize}
			  \item Introducción
			  \item Verso
			  \item Pre-estribillo
			  \item Estribillo
			  \item Verso
			  \item Estribillo
			  \item Puente
			  \item Cierre
			\end{itemize}
		\newpage
		\section{Cadenas de Markov}
	    Las cadenas de Markov son un \Gls{Sistema} matemático el cual experimenta con las transiciones de un \Gls{Estado} a otro de acuerdo con ciertas reglas probabilísticas. La característica que define a una cadena de Markov es que no importa cómo llegó el proceso a su estado actual, y sus posibles estados futuros son fijos. En otras palabras, la probabilidad de pasar a cualquier otro estado depende únicamente del estado actual y del tiempo transcurrido. El espacio de estados o conjunto de todos los posibles estados, pueden ser cualquier cosa: letras, números, puntuaciones de un partido, acciones, etc.\par
		Son procesos \Gls{Estocastico}s, con la diferencia de que estos deben ser “sin memoria”, es decir, la probabilidad de las acciones futuras no depende ni se ve afectada de los pasos que la condujeron al estado actual. A esto se le denomina una propiedad de Markov.\par
		En la teoría de probabilidad, el ejemplo más inmediato es el de una cadena de Markov homogénea en el tiempo, en la que la probabilidad de que ocurra cualquier transición de estado es independiente del tiempo.\par
		En el lenguaje de probabilidad condicional y variables aleatorias, una cadena de Markov es una secuencia X0, X1, X2, … de variables aleatorias que satisfacen la regla de independencia condicional.
		En otras palabras, el conocimiento del estado anterior es todo lo que se necesita para determinar la distribución de probabilidad del estado actual. Esta definición es más amplia que la explorada anteriormente, ya que permite probabilidades de transición no estacionarias y, por lo tanto, cadenas de Markov no homogéneas en el tiempo; es decir, a medida que pasa el tiempo los pasos aumentan y la probabilidad de pasar de un estado a otro puede cambiar. \cite{refMarkov}\par
		Las cadenas de Markov pueden ser modeladas mediante \Gls{Maquinas de Estados Finitos}.\par
		\begin{figure}[H]
			\includegraphics[width=12cm]{./imagenes/MarcoTeorico/Markov/CMarkov.png}
			\centering 
			\caption{Diagrama De Estados finitos De Una Cadena De Markov}
		\end{figure}

		\section{Flask}
		Flask es un marco (framework) web, esto es, un modulo de Python el cual permite desarrollar aplicaciones web. No cuenta con un Manejador de Objetos Realacionales u \acrshort{orm} por sus siglas en inglés, pero si cuenta con características como el enrutamiento de URLS y un motor de plantillas. En general es un marco de aplicación web WSGI.\cite{refFlask3}\par
		La \acrfull{wsgi} es una especificación que describe como se va a comunicar un servidor web con una aplicación web, y como se pueden llegar a enlazar distintas aplicaciones web para procesar una solicitud o una petición.\cite{refFlask2}\par
		Para poder empezar a trabajar con un proyecto o aplicación web Flask lo primero que se hace es usar virtualenv para poder manejar un ambiente virtual separado para el proyecto y así evitar conflictos entre dependencias de este. Virtualenv en una herramienta que nos ayuda a crear ambientes Python aislados (en forma de un directorio).\par
		Para poder hacer esto, en la terminal se escribe:\par
		\begin{center}
		virtualenv --no-site-packages myapp\par
		\end{center}
		Al usar ese comando se va a crear un directorio con la siguiente estructura:\par
		\begin{itemize}
		    \item myapp/
		    \begin{itemize}
		        \item bin/
		        \item include/
		        \item lib/
		    \end{itemize}
		\end{itemize}
		Si se trabaja en Windows la estructura es la siguiente:\par
		\begin{itemize}
		    \item myapp/
		    \begin{itemize}
		        \item lib/
		        \item Scripts/
		    \end{itemize}
		\end{itemize}
		Para poder trabajar con este ambiente, en la terminal se debe acceder a la carpeta bin o Scripts y activar el ambiente de la siguiente manera:\par
		\begin{center}
		source myapp/bin/activate\par
		source myapp/Scripts/activate
		\end{center}
		Posteriormente se procede a escribir el código de una aplicación, por ejemplo, el código mostrado a continuación es extraído de la documentación de flask.\cite{refFlask1}
		\begin{minted}{python}
		from flask import Flask  
		app = Flask(__name__) 

		@app.route('/')  
		def hello_world():  
		    return 'Hello, World!'  
		\end{minted}
		Se guarda como cualquier otra aplicación de Python y se ejecuta de la misma manera que una aplicación de Python. Una vez ejecutada aparecerá lo siguiente en terminal:\par
		\begin{center}
		* Running on http://127.0.0.1:5000/
		\end{center}
		Si se copia esa URL y se abre en un navegador se va a poder observar nuestra aplicación web en ejecución, en este caso, y habiendo hecho uso del ejemplo anterior, se mostrará en pantalla “Hello, World!”\par
		Como se hizo mención al inicio, una de las características de Flask es que se pueden manejar las rutas de la aplicación web usando el método route(), siempre que un usuario visite esa ruta se va a ejecutar el método adjuntado a este.\par
		Flask soporta diferentes tipos de estructuras de variables. Se pueden incluir cadenas personalizadas, números enteros y otros tipos de caracteres en las URLs.\par
		Si se requiere manejar solicitudes, Flask tiene soporte para solicitudes de tipo GET y POST, por defecto Flask solo admite peticiones de tipo GET, por lo que si se necesita hacer uso de peticiones tipo POST esta debe de estar especificada como parámetro de la función route(), por ejemplo:\par
		\begin{center}
		@app.route('/login', methods=['POST'])
		\end{center}
%	    A continuación en el cuadro \ref{cuadro:MetododsAutentificacion}, mostraremos algunas tablas comparativas que servirán para tener una mejor perspectiva de las ventajas, desventajas.
		
%		\begin{table}[H]
%			\centering
%			\resizebox{13cm}{!} {
%				\begin{tabular}{l|l|l|l|l|l|l|l|}
%					\cline{2-8}
%					& Recordar & \begin{tabular}[c]{@{}l@{}}Otros\\ dispositivos\end{tabular} & Acciones & Facilidad & Tiempo & Errores & Recuperación \\ \hline
%					\multicolumn{1}{|l|}{Contraseñas}                                                      & 1        & 3                                                            & 2        & 3         & 3      & 2       & 3            \\ \hline
%					\multicolumn{1}{|l|}{Otros recursos}                                                   & 2        & 3                                                            & 3        & 3         & 3      & 3       & 2            \\ \hline
%					\multicolumn{1}{|l|}{\begin{tabular}[c]{@{}l@{}}Contraseñas \\ gráficas\end{tabular}}  & 1        & 1                                                            & 2        & 3         & 3      & 2       & 3            \\ \hline
%					\multicolumn{1}{|l|}{\begin{tabular}[c]{@{}l@{}}Contraseñas \\ dinámicas\end{tabular}} & 1        & 3                                                            & 2        & 2         & 3      & 2       & 2            \\ \hline
%					\multicolumn{1}{|l|}{Tokens}                                                           & 3        & 1                                                            & 1        & 2         & 2      & 3       & 1            \\ \hline
%					\multicolumn{1}{|l|}{Multivariación}                                                   & 1        & 1                                                            & 1        & 3         & 2      & 2       & 1            \\ \hline
%					\multicolumn{1}{|l|}{Cryptografía}                                                     & 3        & 1                                                            & 1        & 1         & 1      & 2       & 1            \\ \hline
%					\multicolumn{1}{|l|}{Biométricos}                                                      & 3        & 3                                                            & 2        & 3         & 2      & 2       & 1            \\ \hline
%				\end{tabular}
%			}
%			\caption[Aplicación de los métodos de autentificación]{Tabla comparativa de la aplicación en los distintos métodos de autentificaci\'on}
%			\label{cuadro:MetododsAutentificacion}
%		\end{table}
%		La tabla anterior concentra las siguientes características:
%		
%		\begin{itemize}
%			\item Recordar: Hace referencia a que tan complicado es que un usuario se acuerde de los datos necesarios para la autentificaci\'on. 
%			\item Otros dispositivos: El usuario usa una entidad externa para facilitar su autentificaci\'on.
%			\item Acciones: Hace referencia a que tantas acciones adicionales se deben de realizar para autentificarse.
%			\item Facilidad: Simplicidad de tecnología.
%			\item Tiempo: Cantidad de recursos temporales que consume el método de autentificaci\'on.
%			\item Errores: Posibles errores durante la autentificaci\'on. 
%			\item Recuperación: Denota la dificultad de recuperar la clave de acceso en caso de pérdida.
%		\end{itemize}
		
%		En el cuadro \ref{cuadro:VentajasMetodos} se muestra una tabla comparativa del nivel de seguridad en los distintos métodos de autentificaci\'on, donde 1 - baja seguridad, 2 – media seguridad y 3 – alta seguridad.
		
%		\begin{table}[H]
%			\centering
%			\resizebox{10cm}{!} {
%				\begin{tabular}{l|l|l|l|l|}
%					\cline{2-5}
%					& \begin{tabular}[c]{@{}l@{}}Ataque por\\ fuerza bruta\end{tabular} & Observación & \begin{tabular}[c]{@{}l@{}}Hackeo\\ indirecto\end{tabular} & Phishing \\ \hline
%					\multicolumn{1}{|l|}{Contraseñas}                                                      & 1                                                                 & 1           & 1                                                          & 1        \\ \hline
%					\multicolumn{1}{|l|}{Otros recursos}                                                   & 2                                                                 & 2           & 3                                                          & 3        \\ \hline
%					\multicolumn{1}{|l|}{\begin{tabular}[c]{@{}l@{}}Contraseñas \\ gráficas\end{tabular}}  & 1                                                                 & 1           & 2                                                          & 2        \\ \hline
%					\multicolumn{1}{|l|}{\begin{tabular}[c]{@{}l@{}}Contraseñas \\ dinamicas\end{tabular}} & 2                                                                 & 3           & 2                                                          & 2        \\ \hline
%					\multicolumn{1}{|l|}{Tokens}                                                           & 3                                                                 & 3           & 3                                                          & 3        \\ \hline
%					\multicolumn{1}{|l|}{Multivariación}                                                   & 1                                                                 & 1           & 3                                                          & 3        \\ \hline
%					\multicolumn{1}{|l|}{Cryptografía}                                                     & 3                                                                 & 3           & 3                                                          & 3        \\ \hline
%					\multicolumn{1}{|l|}{Biométricos}                                                      & 3                                                                 & 3           & 1                                                          & 1        \\ \hline
%				\end{tabular}
%			}
%			\caption[Seguridad en los métodos de autentificación]{Tabla comparativa de la seguridad en los distintos métodos de autentificaci\'on}
%			\label{cuadro:VentajasMetodos}
%		\end{table}
%		La tabla se enfoca principalmente en los siguientes problemas de seguridad: 
		
%		\begin{itemize}
%			\item Ataque por fuerza bruta: Se descifra el método de autentificaci\'on con una gran cantidad de intentos, usualmente generados por un programa.
%			\item Observación: Cuando se intenta ver directamente los datos necesarios para la autentificaci\'on desde una distancia cercana hasta incluso usando binoculares, cámaras o algún otro dispositivo.
%			\item Hackeo indirecto: El usuario confía sus datos del método de autentificaci\'on a terceros quienes pueden ser atacados. 
%			\item Phishing: Hace referencia a programas que se hacen pasar por entidades confiables para interceptar los datos que desean.
%		\end{itemize}
		
%	    \textbf{Seguridad en internet}
%	    En la actualidad, el incremento constante de internet ha impactado directamente en la seguridad de la información que se maneja cotidianamente y por la mayoría de usuarios. Existen infinidad de sitios donde es aplicada la seguridad, ya que sin ésta, se verían afectados todos los usuarios en  sus cuentas, pudiendo verse afectados desde un posible \Gls{Palabra1} (Robo de identidad), hasta la perdida de dinero real dado que la base de algunas de éstas paginas son E-Commerce, estas paginas implican el manejo de tarjetas de crédito, paypal, etc.\\
		
%		Uno de los puntos más críticos de la seguridad en Internet son las herramientas que interactúan de forma directa con los usuarios. Es común escuchar sobre fallas en los sistemas de protección de los servidores más frecuentemente utilizados, por ejemplo Apache, NGINX, IIS, etc. O en los lenguajes de programación en que son escritas las aplicaciones \cite{refSeguridadWeb}. Sin embargo, la vulnerabilidad más grande dentro de un sistema, son los ataques directos a los usuarios finales durante la autentificación.\\
	    
%	    Una de las técnicas de autentificación que actualmente se usa es ''recordar la sesión'' usando las ''cookies'', en la siguiente sección, nos adentraremos en definir qué son las cookies y exponer sus vulnerabilidades.
	    
	    \section{Apache}
	 	Es un servidor web gratuito y de código abierto que permite que los dueños de sitios web muestren contenido en ellas, no es un servidor físico sino más bien un software que corre en un servidor.
	 	Su trabajo es establecer la conexión entre un servidor y los navegadores de los visitantes del sitio web mientras se mandan archivos de ida y vuelta entre ellos (estructura cliente-servidor). El servidor y el cliente se comunican mediante el protocolo HTTP, y el software de Apache es responsable por la comunicación fluida y segura entre las 2 máquinas.\par
	 	Apache trabaja sin problemas con muchos otros sistemas de gestión de contenido (Joomla, Drupal, etc.), marcos de trabajo web (Django, Laravel, etc.), y lenguajes de programación. Todo esto en conjunto lo vuelve una opción sólida al momento de escoger entre todos los tipos de plataformas de alojamiento web, como serían VPS o alojamiento compartido.\\
		
		\newpage
		
		\begin{table}[hbt!]\caption{Tabla comparativa de los distintos servidores web contemplados para nuestro proyecto}% title of Table
			\centering % used for centering table
			\resizebox{15cm}{!} {
				\begin{tabular}{c c c}% centered columns (3 columns)
					\hline\hline                        %inserts double horizontal lines
					Apache & NGINX & Tomcat \\ [0.5ex]% inserts table %heading
					\hline                  % inserts single horizontal line
					\begin{tabular}[c]{@{}l@{}}Apache viene con ventajas útiles\\ sobre Nginx, como sería su fácil \\configuración, muchos módulos, y \\un entorno amigable con los\\ primerizos.\end{tabular}
					& \begin{tabular}[c]{@{}l@{}}Nginx fue creado para resolver el\\ llamado “problema c10k”,\\ significando que un servidor web \\que usa hilos para manejar \\solicitudes del usuario es incapaz \\de gestionar más de 10,000\\ conexiones al mismo tiempo.\end{tabular}
					& \begin{tabular}[c]{@{}l@{}}Tomcat fue creado \\específicamente para aplicaciones \\Java, mientras que Apache es un \\servidor HTTP de propósito \\general.\end{tabular} \\% inserting body of the table 
					\hline
					\begin{tabular}[c]{@{}l@{}}De código abierto y gratuito,\\ incluso para su uso comercial.\\ Software estable y confiable.\\Frecuentemente actualizado \\incluyendo actualizaciones\\ regulares a los parches de \\seguridad.\end{tabular}
					& \begin{tabular}[c]{@{}l@{}}Nginx es uno de los servidores \\web que soluciona el problema \\c10k y probablemente el más \\exitoso al hacerlo, no crea un \\nuevo proceso por cada solicitud y \\en su lugar, maneja toda solicitud \\entrante en un único hilo.\end{tabular}
					& \begin{tabular}[c]{@{}l@{}}Tomcat es menos configurable \\comparado a otros servidores \\web. Por ejemplo, para correr \\WordPress, la mejor elección es \\un servidor HTTP de propósito \\general como Apache o Nginx.\end{tabular} \\
					\hline
					\begin{tabular}[c]{@{}l@{}}Funciona de forma intuitiva, con \\sitios web hechos con WordPress. \\Comunidad grande y posibilidad \\de contactar con soporte \\disponible de manera sencilla en \\caso de cualquier problema.\end{tabular}
					& \begin{tabular}[c]{@{}l@{}}El modelo basado en eventos de \\Nginx distribuye solicitudes de \\usuario entre procesos \\trabajadores de una manera \\eficiente, llevando con ello a una \\mejor escalabilidad.\end{tabular}
					&  \\
					\hline
					\begin{tabular}[c]{@{}l@{}}Problemas de rendimiento en \\sitios web con tráfico\\ extremadamente pesado. \\Tantas opciones de configuración \\pueden llevar a vulnerabilidades \\de seguridad.\end{tabular}
					& \begin{tabular}[c]{@{}l@{}}Si necesitas manejar un sitio web \\con un alto tráfico, Nginx es una \\excelente opción, puesto que \\puede hacerlo con un mínimo uso \\de los recursos.\end{tabular}
					&  \\
					[1ex]      % [1ex] adds vertical space
				\end{tabular}\label{table:VentajasServidor}% is used to refer this table in the text
			}
		\end{table}

	    \section{Servidor Web}
	        El trabajo de un servidor web es mostrar sitios web en internet. Para conseguir este objetivo, actúa como un intermediario entre el servidor y las máquinas del cliente. Obtiene el contenido del servidor en cada petición hecha por el usuario y lo entrega al sitio web.\\
	        Los servidores web procesan archivos escritos en diferentes lenguajes de programación como PHP, Python, Java, y otros.\\
	        Cuando se escuche la palabra ‘servidor web’, se puede pensar en ello como la herramienta responsable por la correcta comunicación entre el cliente y el servidor.

		 \section{Certificado SSL}
			Conocido como (Secure Sockets Layer) o (Capa de Conexión Segura) es un estándar de seguridad global que fue originalmente creado por Netscape en los 90's. SSL crea una conexión encriptada entre tu servidor web y el navegador web de tu visitante permitiendo que la información privada sea transmitida sin que ocurran problemas como serían espionaje, manipulación de la información, y falsificación de los datos del mensaje. Básicamente, la capa SSL permite que dos partes tengan una “conversación” privada.\par
			Para establecer esta conexión segura, se instala en un servidor web un certificado SSL (también llamado 'certificado digital') que cumple dos funciones:
			
			\begin{itemize}
				\item Autentificar la identidad del sitio web, garantizando a los visitantes que no están en un sitio falso. 
				\item Cifrar la informaci\'on transmitida.
			\end{itemize}
		
			Hay varios tipos de certificados SSL según la cantidad de nombres de dominio o subdominios que se tengan, como por ejemplo:
			
			\begin{itemize}
				\item \'Unico: Asegura un nombre de dominio o subdominio completo (FQDN por sus siglas en ingl\'es). 
				\item Comod\'in: Cubre un nombre de dominio y un n\'umero ilimitado de sus subdominios.
				\item Multidominio: Asegura varios nombres de dominio.
			\end{itemize}
		
		\newpage
		\section{Plataforma en la Nube de Aprendizaje Autom\'atico}
		
		El entrenamiento de aprendizaje automático y de modelos de aprendizaje profundo involucra miles de iteraciones. Se necesitn esta gran cantidad de iteraciones para producir el modelo más preciso.\par 
		El cómputo en la nube permite modelar capacidad de almacenamiento y manejar cargas a escala, o escalar el procesamiento a través de los nodos. Por ejemplo, AWS ofrece instancias de GPUs con capacidad de memoria que va de los 8Gb's a los 256Gb's, estas instancias son cobradas a ritmos por hora.\\
		Las GPUs son procesadores especializados diseñados para procesado complejo de imágenes. Azure de Microsift ofrece GPUs de alto rendimiento de la serie NC para aplicaciones o algoritmos de cómputo de alto rendimiento.
		
		\subsection{Amazon}
		Dentro de las bondades que ofrecen los Servicios Web de Amazon (AWS por sus siglas en ingl\'es) y que nos pueden ser de utilidad para las nacesidades de nuestro proyecto se pueden encontrar: 
		
		\begin{itemize}
			\item \textbf{SageMaker:} Es una plataforma de aprendizaje automático completamente administrado para científicos de datos y desarrolladores. La plataforma corre en Cómputo Elástico en la  Nube (EC2 por sus siglas en inglés), y permite construir modelos de aprendizaje automático, organizar la información y escalar sus operaciones. \\
			Algunas aplicaciones de aprendizaje automático en SageMaker van desde reconocimiento de voz hasta visión de computadora, e incluso recomendaciones basadas en el comportamiento aprendido del usuario.\par 
			El mercado de AWS ofrece modelos que se pueden usar en lugar de empezar desde cero, posterior a eso se puede entonces empezar a entrenar y optimizar el modelo; las elecciones más comunes son frameworks como Keras, TensorFlow, y PyTorch; Sagemaker puede optimizar y configurar estos frameworks automáticamente, o pueden ser entrenadas de manera personal.\par 
			Uno mismo puede incluso desarrollar su propio algoritmo construyéndolo en un contenedor de Docker o se puede hacer uso de una “Jupyter notebook” para construir un modelo propio de aprendizaje automático y visualizar su información usada para el entrenamiento del modelo siendo este punto lo que más nos interesa para nuestro proyecto.
			
		\end{itemize}
		
		\begin{comment}
		
			\item \textbf{Lex:} Fue diseñado para integrar chatbots en aplicaciones. Lex cuenta con capacidades de aprendizaje profundo basado en procesamiento de lenguaje natural (NLP) y reconocimiento automático de voz. La API puede reconocer texto hablado y escrito. 
			\item \textbf{Rekognition:} Es un servicio de visión computacional que simplifica el proceso de desarrollo para aplicaciones de reconocimiento de imagen y vídeo.
		
			\subsection{Azure}
			Comparado a AWS, las propuestas del aprendiza de máquina de Azure son más flexibles en términos de algoritmos out-of-the-box.
			
			\begin{itemize}
			\item \textbf{Servicios de Aprendizaje Automático:} Es una gran librería de algoritmos de aprendizaje automático pre-entrenados y pre-empacados, también provee de un ambiente para implementar estos algoritmos y aplicarlos a aplicaciones del mundo real. Se puede usar la interfaz de usuario para entrenar, probar y evaluar los modelos así como también provee soluciones para Inteligencia Artificial (IA) (Esto incluye visualización y otra información que puede ayudar a comprender el comportamiento del modelo y comparar algoritmos para encontrar la mejor opción). Algunas de las ofertas dentro de este apartado son:
			
			\begin{itemize}
			\item \textbf{Paquetes de Python:} Contiene funciones y librerías para visión computacional, análisis de texto, previsión, y aceleramiento de hardware. 
			\item \textbf{Gestión de Modelo:} Provee un entorno para acoger modelos, gestionar versiones, y monitorear modelos que corran en Azure o en soluciones propias de la casa.
			\item \textbf{Workbench:} Una simple línea de comandos y entorno de escritorio con dashboards y herramientas de seguimiento del desarrollo de modelos. 
			\end{itemize}
			
			\item \textbf{Frameworks de Servicio de Bots:} Provee de un entorno para la construcción, despliegue, y prueba  de bots usando diferentes lenguajes de programación.
			\end{itemize}
			
			\subsection{Google}
			Google provee servicios de aprendizaje automático e Inteligencia Artificial (IA) en dos niveles – Motor de Aprendizaje automático de Google Cloud para profesionales de datos experimentados y la plataforma para principiantes de Cloud AutoML (Aprendizaje Automático-Automático).
			
			\begin{itemize}
			\item \textbf{AutoML:} Una plataforma de aprendizaje automático basada en la nube desarrollada para usuarios sin experiencia. Se pueden cargar datasets, entrenar modelos, y desplegarlos en un sitio web. AutoML se integra con todos los servicios de Google y guarda la información en la nube.\\
			Se pueden desplegar modelos entrenados vía una interfaz de REST API. Se puede acceder a un cierto número de productos AutoML disponibles a través de una interfaz gráfica.
			\item \textbf{Motor de Aprendizaje Automático:} Se puede usar Google Cloud ML para entrenar un modelo complejo apalancando la infraestructura de la GPU y la Unidad de Procesamiento de Tensor (TPU por sus siglas en inglés). Incluso se puede usar el servicio para desplegar un modelo entrenado de manera externa.\\
			El Aprendizaje automático en la Nube automatiza todo el monitoreo y los procesos proveedores de recursos para ejecutar las tareas;  aparte de hospedar y entrenar, Cloud ML también puede llevar a cabo ajustes de hiperparámetros que influencian la precisión de las predicciones.
			\item \textbf{Tensorflow:} Es una librería de software open-source que usa gráficas de flujo de datos para operaciones numéricas. Las operaciones matemáticas en estas gráficas son representados por nodos, mientras que los bordes representan la información transferida de un nodo a otro. La información en TensorFlow es representada en forma de tensores, los cuales son arreglos multidimensionales.\\ Tensorflow es usualmente usado para investigación y práctica de aprendizaje profundo, y es multi-plataforma por lo que se puede correr en GPU\'s, CPU\'s, TPU\'s y plataformas móviles.
			\end{itemize}
		\end{comment}
		
		\begin{table}[hbt!]\caption{Tabla comparativa de las diversas plataformas contempladas}% title of Table
			\centering % used for centering table
			\resizebox{15cm}{!} {
				\begin{tabular}{c c c c}% centered columns (3 columns)
					\hline\hline                        %inserts double horizontal lines
					Característica & GCP & AWS & Azure \\ [0.5ex]% inserts table %heading
					\hline                  % inserts single horizontal line
					\begin{tabular}[c]{@{}l@{}}Jupyter Notebook\\alojado de \\manera local\\ o remota\\\end{tabular}
					& \begin{tabular}[c]{@{}l@{}}Plataforma de IA\end{tabular}
					& \begin{tabular}[c]{@{}l@{}}SageMaker Studio IDE\end{tabular}
					& \begin{tabular}[c]{@{}l@{}}- Estudio de Notebooks\\ de Azure de Aprendizaje\\ Autom\'atico\\- Databrick de Azure\end{tabular} \\% inserting body of the table 
					\hline
					Entrenamiento\\Distribuido
					& \begin{tabular}[c]{@{}l@{}}Si\end{tabular}
					& \begin{tabular}[c]{@{}l@{}}Si\end{tabular}
					& \begin{tabular}[c]{@{}l@{}}Si\end{tabular}\\
					\hline
					Versionado de Modelos
					& Si
					& \begin{tabular}[c]{@{}l@{}}Si\end{tabular}
					& \begin{tabular}[c]{@{}l@{}}Si\end{tabular}\\
					\hline
					Seguimiento de \\Experimentos
					& \begin{tabular}[c]{@{}l@{}}Si\end{tabular}
					& Si
					& \begin{tabular}[c]{@{}l@{}}Si\end{tabular}\\
					\hline
					AutoML\\(UI y API) 
					& \begin{tabular}[c]{@{}l@{}}Tabla de AutoML\end{tabular}
					& \begin{tabular}[c]{@{}l@{}}Autopiloto de\\SageMaker\end{tabular}
					& \begin{tabular}[c]{@{}l@{}}AutoML\end{tabular}\\
					\hline
					Análisis de Errores
					& \begin{tabular}[c]{@{}l@{}}Tabla de AutoML\\con BigQuery\end{tabular}
					& \begin{tabular}[c]{@{}l@{}}Debugger de\\SageMaker\end{tabular}
					& \begin{tabular}[c]{@{}l@{}}Aprendizaje Profundo\\de Azure\end{tabular}\\
					[1ex]      % [1ex] adds vertical space
				\end{tabular}\label{table:VentajasServidor}% is used to refer this table in the text
			}
		\end{table}
	
		\newpage
		Los 3 proveedores de la tabla anterior han alojado servicios de “Jupyter Notebook” (contienen tanto código de computadora como elementos ricos en texto como ecuaciones, figuras, etc.), experimentando seguimientos y control de versiones, y métodos de despliegue sencillos.\\
		\\
		Dentro de las características únicas de cada una de las plataformas podemos encontrar que la Plataforma de la Nube de Google (GCP) usa un paquete llamado “what if tool” el cual se puede integrar junto a un “Jupyter Notebook” y de esa manera jugar con el modelo cambiando el umbral o un valor característico de un ejemplo dado, esto permite checar como es que ciertos cambios afectan el resultado predicho con anterioridad previo al cambio.\\
		\\
		El debbuger de SageMaker de AWS permite analizar cómo es que la ingeniería de características y el refinado del modelo son hechos, o de forma más concisa, permite ver qué sucede durante el entrenamiento del modelo.\\
		\\
		Azure, por su parte, provee un módulo propio en su SDK el cual parece tener la mejor integración de entre las tres plataformas.\\
		\\
		\textbf{Costo y Rendimiento del Modelo}\\
		\\
		Azure y GCP puntúan ligeramente mejor que AWS en términos de rendimiento, esto no necesariamente significa que una plataforma es mejor que otra.\\
		\\
		El costo, por otro lado, resulta ser más interesante. El costo de AWS fue considerablemente menor que GCP y Azure. Por supuesto, esta no es estrictamente una comparación de manzanas-manzanas.\\
		\\
%		Si se tienen problemas con el presupuesto y el gasto por el 5 por ciento extra de mejora en el rendimiento no justifica el retorno, la historia se vuelve completamente distinta.\\
		\\
		El diseñador de aprendizaje automático de Azure cuenta con una interfaz de arrastar y soltar el cual es bastante amigable con aquellos nuevos en la ciencia de datos, con menos codeo ni antecedentes técnicos. AWS y GCP parecen ser más enfocados a desarrolladores. Aunque puede resultar en un poco más de trabajo ensamblar una pipeline (o cadena de procesos), son más personalizables con los diferentes componentes disponibles. Estos componentes y la conexión de la pipeline son usualmente desarrollados usando código y configuraciones, en vez de haciendo uso de una interfaz.\\
		\\
		Tanto GCP como AWS ofrecen un modelo de pago “pay-as-you-go”. Este modelo es el mejor para aquellos individuos que puedan llegar a esperar un uso intermitente de la nube, ya que permite un enfoque flexible para añadir y remover servicios cuando se necesite. Por supuesto, este nivel de flexibilidad tiene un costo, haciendo al modelo “pay-as-you-go” el más caro por hora.
		
		\begin{table}[hbt!]\caption{Tabla comparativa GCP vs AWS (en dólares)}% title of Table
			\centering % used for centering table
			\resizebox{13cm}{!} {
				\begin{tabular}{c c c}% centered columns (3 columns)
					\hline\hline                        %inserts double horizontal lines
					Tipo de Instancia & Precio de EC2 (por hora) & Precio de Google (por hora) \\ [0.5ex]% inserts table %heading
					\hline                  % inserts single horizontal line
					\begin{tabular}[c]{@{}l@{}}Propósito\\General\end{tabular}
					& \begin{tabular}[c]{@{}l@{}}\$0.134\end{tabular}
					& \begin{tabular}[c]{@{}l@{}}\$0.15\end{tabular} \\% inserting body of the table 
					\hline
					Cómputo\\Optimizado
					& \begin{tabular}[c]{@{}l@{}}\$0.136\end{tabular}
					& \begin{tabular}[c]{@{}l@{}}\$0.188\end{tabular}\\
					\hline
					Optimizado de\\Memoria
					& \begin{tabular}[c]{@{}l@{}}\$0.201\end{tabular}
					& \begin{tabular}[c]{@{}l@{}}\$0.295\end{tabular}\\
					\hline
					GPU
					& \begin{tabular}[c]{@{}l@{}}\$0.526\end{tabular}
					& \begin{tabular}[c]{@{}l@{}}\$1.4\end{tabular}\\
					[1ex]      % [1ex] adds vertical space
				\end{tabular}\label{table:ComparacionPlataformas}% is used to refer this table in the text
			}
		\end{table}
	
		Habiendo analizado todo lo anterior, resulta claro que la opción ideal para las necesidades de nuestro proyecto es AWS debido a que es más barato que las otras dos plataformas y que posee la herramienta de SageMaker, siendo que la competencia no cuenta con alguna similar, y la cual nos permitirá saber cómo es que se está entrenando el modelo.
		
		
            \begin{comment}
           \section{Inteligencia Artificial}
           \subsection{¿Qu\'e es la Inteligencia Artificial?}
           
            \subsection{Historia}
            
            \subsection{Objetivos de la Inteligencia Artificial.}
            \subsection{¿Cómo funciona?}   
            \subsection{Usos de la Inteligencia Artificial.}
                
                
            \subsection{Inteligencia Artificial Supervisada y no supervisada.}
               
                
                \subsubsection{Inteligencia Artificial Supervisada.}
                
        		
        		\subsubsection{Inteligencia Artificial No Supervisada.}
        		
    		\subsection{Procesamiento de Lenguaje Natural.}\\
                
                \subsubsection{SSL/TLS.}
                Secure Sockets Layer (SSL) provee servicios de seguridad entre la capa TCP y las aplicaciones que hacer uso de esa capa. Actualmente, la sucesora de SSL es TLS (Transport Layer Service), sin embargo, lo acuñado que está el término SSL hace que se use indistintamente para referirse a TLS, aunado a ello, las diferencias entre la última versión de SSL (SSL3.0) y la primera versión de TLS (TLSv1) son menores, por lo que en el desarrollo de este reporte, utilizaremos el término SSL/TLS.\\
                
                SSL/TLS provee entonces confidencialidad, lográndola con criptografía asimétrica y controlando la integridad de los datos utilizando un MAC (Message Authentication Code).\\
                
                El proceso de comunicación del protocolo establece, como primer paso, la negociación de ambas partes de los algoritmos a utilizar. Luego, procede al intercambio de llaves públicas y a la autentificación basada en certificados digitales para, finalmente, cifrar de manera simétrica los datos o información a transferir \cite{refCriptografia}.\\
                
                En nuestro trabajo, usaremos SSL/TLS para brindar confidencialidad al canal de comunicación entre la extensión y el servidor autentificador. Esto se explicará más a detalle en el análisis del Componente II.
                
                \subsubsection{OpenSSL}
	            OpenSSL es un proyecto de código abierto que implementa funciones criptográficas sin limitaciones dentro de una librería y que provee diversas herramientas útiles. OpenSSL actualmente implementa SSL2.0, SSL3.0 y TLSv \cite{refCriptografia}.\\
	            
	            Dentro de las herramientas que tiene se encuentran: 
	            \begin{itemize}
	                \item Crear y manejar llaves privadas, públicas y parámetros.
	                \item Realizar operaciones criptográficas de llave pública.
	                \item Calcular hash de algún mensaje.
	                \item Cifrar y descifrar con algoritmos simétricos.
	                \item Crear certificados X.509 (CSRs y CRLs).
	            \end{itemize}
	            
	            Esta última herramienta, es la que usaremos para este trabajo terminal. Se creará un certificado de cada usuario a partir de sus datos de inicio de sesión para autentificarlo en los servicios web en donde quiera acceder. Esto se explicará más a detalle en el análisis del Componente II.
	            
         Para nuestro trabajo terminal usaremos \textit{Chaffing and Winnowing}, proponiendo así un nuevo método de autentificación para servicios web. \\
	\end{comment}

    
%   %   %   %   %   %   %   %   %   %
%		        Capítulo 3					%
%   			ANÁLISIS 					%
%                               				  %
%   %   %   %   %   %   %   %   %	%   
	\chapter{\textcolor{azulescom}{Análisis.}}
	    
    	\section{Estudio de Factibilidad.}
    	El estudio de factibilidad es un instrumento que sirve para orientar la toma de decisiones, así como para determinar la posibilidad de desarrollar un negocio o un proyecto; corresponde a la última fase de la etapa pre-operativa del ciclo del proyecto. Se formula con base en información que tiene la menor incertidumbre posible para medir las posibilidades de éxito o fracaso de un proyecto, apoyándose en el resultado se tomará la decisión de proceder o no con su implementación.\par 
    	Este estudio establecerá la viabilidad, si existe, del trabajo planteado previamente.
    	\begin{itemize}
    		\item \textbf{Factibilidad Técnica:} Este aspecto evalúa que la infraestructura, es decir, los equipos, el software, el conocimiento, la experiencia, etc., que se posean son los necesarios para efectuar las actividades requeridas para la realización del Trabajo Terminal.
    		\item \textbf{Factibilidad Operativa:} Analiza si el personal posee las competencias necesarias para el desarrollo del proyecto.
    		\item \textbf{Factibilidad Econ\'omica:} Consiste en el análisis de los recursos financieros necesarios para llevar a cabo la elaboraci\'on del Trabajo Terminal.
    	\end{itemize}
    	\subsection{Factibilidad Técnica}
    	Dentro de este apartado se explican detalladamente las tecnologías que se utilizarán. La elección de estas herramientas estuvo basada tanto en las tecnologías que más se usan en la actualidad como en las que cuentan con el mayor soporte para su trabajo en la nube, a su vez se detallan las características de los equipos de c\'omputo con los que se cuenta actualmente.
    	\begin{table}[H]
    		\begin{tabular}{ |p{3.5cm}|p{9.5cm}|}
    			\hline
    			\rowcolor{guindapoli}
    			\multicolumn{2}{|c|}{\textbf{\textcolor{white}{Factibilidad T\'ecnica}}}\\
    			\hline
    			\cellcolor{azulclaro}Sistema Operativo & 
    			Linux, Mac, Windows \\ 
    			\hline
    			\cellcolor{azulclaro}Navegador Web &
    			Google Chrome\\
    			\hline
    			\cellcolor{azulclaro}Lenguaje de Programaci\'on &
    			Python\\
    			\hline
    			\cellcolor{azulclaro}Servidor &
    			Apache y Gunicorn\\
    			\hline
    			\cellcolor{azulclaro}Servicio Nube &
    			Amazon Web Services\\
    			\hline
    			
    		\end{tabular}
    		\caption[Herramientas de Software]{Herramientas de Software a utilizar}
    	\end{table}
    	Además de las herramientas de software a utilizar, es necesario mencionar el equipo de hardware que se utiliza, tanto para desarrollar, como para probar e implementar cada uno de los prototipos a lo largo de este Trabajo Terminal.
            	    \begin{table}[H]
    	\begin{tabular}{|p{3.5cm}||p{10cm}|}
    		\rowcolor{guindapoli}
    		\multicolumn{2}{|c|}{\textbf{\textcolor{white}{Equipo de hardware utilizado. [1]}}}\\
    		\hline
    		\rowcolor{azulclaro}Procesador & Ryzen 5 3600\\
    		\hline
    		\rowcolor{white}Tarjeta de video & Amd Radeon Rx580\\
    		\hline
    		\rowcolor{azulclaro}Memoria RAM & 32 Gb\\
    		\hline
    		\rowcolor{white}Disco duro & 1Tb HDD y 512Gb SSD\\
    		\hline
    	\end{tabular}
    \end{table}
        	    \begin{table}[H]
    	\begin{tabular}{|p{3.5cm}||p{10cm}|}
    		\rowcolor{guindapoli}
    		\multicolumn{2}{|c|}{\textbf{\textcolor{white}{Equipo de hardware utilizado. [2]}}}\\
    		\hline
    		\rowcolor{azulclaro}Marca & Apple\\
    		\hline
    		\rowcolor{white}Modelo & iMac Late 2012\\
    		\hline
    		\rowcolor{azulclaro}Procesador & Intel Core i5\\
    		\hline
    		\rowcolor{white}Tarjeta de video & NVIDIA GeForce GT 640M 512 Mb\\
    		\hline
    		\rowcolor{azulclaro}Memoria RAM & 8 Gb\\
    		\hline
    		\rowcolor{white}Disco duro & 1Tb y 256 Gb SSD\\
    		\hline
    	\end{tabular}
    \end{table}
        	    \begin{table}[H]
	\begin{tabular}{|p{3.5cm}||p{10cm}|}
		\rowcolor{guindapoli}
		\multicolumn{2}{|c|}{\textbf{\textcolor{white}{Equipo de hardware utilizado. [3]}}}\\
		\hline
		\rowcolor{azulclaro}Procesador & Amd FX-8350\\
		\hline
		\rowcolor{white}Tarjeta de video & Nvidia Geforce 1050ti\\
		\hline
		\rowcolor{azulclaro}Memoria RAM & 16 Gb\\
		\hline
		\rowcolor{white}Disco duro & 1Tb HDD\\
		\hline
	\end{tabular}
\end{table}
\newpage
    	Junto con las herramientas de hardware y software a utilizar es necesario mencionar los servicios b\'asicos que son relevantes para el desarrollo de este trabajo terminal como lo son:
    	\begin{itemize}
    		\item Luz Eléctrica
    		\item Agua Potable
    		\item Internet
    	\end{itemize}
    	Estos servicios forman parte de la factibilidad técnica ya que sin ellos no se podría realizar este proyecto y por eso mismo generan un costo, dicho costo se menciona en la Factibilidad Económica.
    	
    	\subsection{Factibilidad Operativa}
    	Los recursos operativos del presente Trabajo Terminal se calcularon con base en los recursos humanos con los que se cuenta actualmente y un análisis de las horas en las que el personal estará en operación, lo cual se muestra a continuación:
    	
    	\begin{table}[h!]
    		\begin{tabular}{|p{2cm}|p{1.4cm}|p{2.2cm}|p{1.6cm}|p{2.2cm}|p{1.6cm}|}
    			\hline    			
    			\rowcolor{guindapoli}
    			\multicolumn{6}{|c|}{\textbf{\textcolor{white}{Horas a trabajar en el desarrollo del trabajo terminal}}}\\
    			\hline
    			Mes & No. de Días & Sábado y Domingo & Días hábiles & Horas de trabajo por día & Horas Totales \\
    			\hline
    			Marzo & 31 & 8 & 22 & 2 & 44 \\ 
    			\hline
    			Abril & 30 & 8 & 15 & 2 & 30 \\ 
    			\hline
    			Mayo & 31 & 10 & 19 & 2 & 38 \\
    			\hline
    			Junio & 30 & 10 & 17 & 2 & 34 \\
    			\hline
    			Agosto & 31 & 10 & 18 & 2 & 24 \\
    			\hline
    			Septiembre & 30 & 8 & 20 & 2 & 40 \\ 
    			\hline
    			Octubre & 31 & 10 & 20 & 2 & 40 \\ 
    			\hline
    			Noviembre & 31 & 8 & 18 & 2 & 36 \\ 
    			\hline
    		\end{tabular}
    		\caption[Horas de trabajo]{Relación de horas de trabajo estimadas para la realización de este trabajo terminal}
    	\end{table}
    	Con esto podemos concluir que contamos con 286 horas, suficiente tiempo para el desarrollo de este Trabajo Terminal, ya que las horas totales de trabajo están contempladas para cada uno de los integrantes del equipo
    	
    	\subsection{Factibilidad Económica}
    	Luego de haber realizado el estudio de factibilidad técnica así como el operacional es necesario tomar en cuenta un estudio de factibilidad económica el cual desglosará todo el gasto económico realizado para la elaboración de este Trabajo Terminal:
    	\begin{itemize}
    		\item \textbf{Capital Humano:} Se tienen contemplados aproximadamente 36 días laborales, es decir, 288 horas para la elaboración de este trabajo terminal en el cual participaremos los tres integrantes
    		\item \textbf{Capital Técnico:} Se cuenta con las viviendas de cada uno de los integrantes así como con el principal equipo de cómputo de cada uno.
    	\end{itemize}
    	Respecto a los costos monetarios de todo el proyecto se tomo a consideración lo siguiente:
    	\begin{itemize}
    		\item \textbf{Servicios}\\
    		Se considera un gasto mensual aproximado de \$1,600.00 que al ser multiplicado por todo el tiempo de elaboración comtemplado nos da un total de \$ 14,400.00.
    		\item \textbf{Software} \\
    		Durante algunos periodos se va a hacer uso principalmente de herramientas gratuitas y de software libre, en cuanto al servicio de la nube de pretende hacer uso de AWS y trabajar inicialmente con los planes gratuitos que ofrece, y si se llegaran a consumir los recursos de ese plan entonces se procederá a cambiarnos a otro plan superior donde se cobra 0.15 centavos de dólar por hora de uso.
    		\item \textbf{Hardware}\\
    		Se utilizarán los equipos de cómputo personal de cada integrante, lo que da un costo total aproximado de \$ 24,012.00 aplicando los parámetros de vida útil de un equipo de cómputo y su depreciación anual. 
    		\item \textbf{Recursos Humanos}\\
    		Se estima un gasto aproximado de \$15,000.00 por cada integrante del equipo para la elaboración del proyecto con lo que se generará un gasto total de \$45,000.00
    	\end{itemize}
    	El costo final del desarrollo de este Trabajo Terminal una vez contemplados los puntos anteriores es de: \\
    	\begin{center}
    		\$83,420.00
    	\end{center}
    	\textbf{Conclusión} 
    	Tras realizar el análisis el estudio de factibilidad de este proyecto es pertinente decir que los integrantes no cuentan con el apoyo financiero y que el hardware mencionado ya es propiedad de los integrantes, por lo que el Trabajo Terminal se califica como \textit{“Viable”} iniciando de esta manera su implementación acorde con las fechas mencionadas.
    	    
    	\section{Herramientas a usar.}
			\subsection{Software.}
			Para el desarrollo de software de este prototipo, es necesario hacer mención de algunas de las siguientes herramientas, para tener una idea clara sobre qué herramientas estamos utilizando y porque es que las estamos utilizando:
			
			\paragraph{HTML5.}
			HTML comenzó mucho tiempo atrás con una simple versión propuesta para crear la estructura básica de páginas web, organizar su contenido y compartir información, todo esto tenía la intención de comunicar información por medio de texto. El limitado objetivo de html motivó a varias compañías a desarrollar nuevos lenguajes y programas para agregar características a la web nunca antes implementadas.   \\\\
			Dos de las opciones propuestas fueron Java y Javascript; ambas fueron muy aceptadas y consideradas como el objetivo de la internet, sin embargo, con el crecimiento exponencial del internet, éste dejó de ser únicamente para los aficionados de los computadores y pasó a ser usado como un campo estratégico para los negocios y para la interacción social, ciertas limitaciones presentes en ambas tecnologías probaron ser una sentencia de muerte. Esta falta de integración resultó ser crítica y preparó el camino para la evaluación de un lenguaje del cual hablaremos un poco más a detalle después: JavaScript. Sin embargo, pese a su gran impacto, el mercado no terminó de adoptarlo plenamente y rápidamente su popularidad fue declinando, y el mercado terminó enfocando su atención a Flash. No fue hasta que los navegadores mejoraron su intérprete para JavaScript y la gente se empezaba a dar cuenta de las limitaciones que ofrecía Flash, que JavaScript fue implementado y comenzó a innovar la forma en la que se programaba la web. Al cabo de unos años, JavaScript, html y css eran considerados como la más perfecta combinación para evolucionar la Web. \\\\
			HTML5 es una mejora de esta combinación, lo que unió todos estos elementos. HTML5 propone estándares para cada aspecto de la Web y también un propósito claro para cada una de las tecnologías involucradas. A partir de esto, html provee los elementos estructurales, CSS se concentra en como volver esta estructura utilizable y atractiva a la vista, y JavaScript tiene todo lo necesario para brindar dinamismo y construir aplicaciones web completamente funcionales. Cabe mencionar que HTML5 funciona diferente dependiendo del navegador y la versión en la que se esté trabajando, algunos soportan más carcterísticas o diferentes funcionalidades que otros.
			
			\paragraph{CSS.\\}
			HTML5 fue evolucionando a un grado de combinación de estructura y diseño, sin embargo, la web demanda diseño y funcionalidad, no solamente organización estructural o definición de secciones, la función de CSS se concentra en volver la estructura de HTML utilizable y atractivo a la vista.\\
			
			Oficialmente CSS no tiene nada que ver con HTML4, no es parte de la especificación, es de hecho, un complemento desarrollado para superar las limitaciones y reducir la complejidad de HTML. Al principio, atributos dentro de las etiquetas HTML proveóan estilos esenciales para cada elemento, pero a medida que HTML evolucionó, la escritura de códigos se volvió más compleja y html por sí mismo no pudo satisfacer más las demandas de los diseñadores.\\
			
			En consecuencia a esta demanda, CSS fue adoptado como la forma de separar la estructura de la presentación, y ha ido creciendo y ganando importancia, pero siempre desarrollado en paralelo enfocado en las necesidades de los diseñadores y apartado de la estructura de HTML.\\
			
			La versión 3 de CSS sigue el mismo camino, pero esta vez con un mayor compromiso. La especificación de HTML5 fue desarrollada considerando CSS a cargo del diseño, Debido a esta consideración, la integración entre HTML y CSS es ahora vital para el desarrollo web y esta razón por la que cada vez que mencionamos HTML5 también estamos haciendo referencia a CSS3, aunque oficialmente se trate de dos tecnologías completamente separadas. Las nuevas características incorporadas en CSS3 están siendo implementadas e incluidas junto al resto de la especificación en navegadores compatibles con HTML5 \cite{refElGranLibro}.
			
			\paragraph {JavaScript. \\}
			JavaScript es considerado como el lenguaje de programación de html y de la web. Es un lenguaje de programación fácil de usar y muy versátil para el ámbito de la comunicación en redes. Los programas, llamados “scripts”, se ejecutan en el navegador (Mozilla, Google Chrome, Internet Explorer, etc.) normalmente consisten en unas funciones que son llamadas desde el propio html cuando algún evento sucede.\\\\
			Su primera aproximación a un uso real, fue en mayor parte para "dar vida a una página web", como dar animaciones a un botón, interacciones en tiempo real, entre otras más. 
			JavaScript fue desarrollado por Brendan Eich, a partir del lenguaje Java, que en ese momento tenía mucho auge y popularidad, y su principal diferencia es que JavaScript sólo "funciona" dentro de una página html.\\
			JavaScript fue declarado como estándar del European Computer Manufacturers Association (ECMA) en 1997, y poco después, también fue estandarizado por ISO \cite{refJavaScript}. \\ 
			JavaScript es un lenguaje interpretado, usado mayormente como complemento de ciertos objetivos específicos, sin embargo, uno de las innovaciones que ayudó a JavaScript fue el desarrollo de nuevos motores de interpretación, creados para acelerar el procesamiento del código. La clave de los motores más exitosos fue transformar el código de Javascript en código máquina para obtener una velocidad de ejecución mejor que antes. Esto a la vez permitió superar viejas limitaciones de rendimiento y confirmar el lenguaje JavaScript como la mejor opción para la Web.\\
			
			Para aprovechar esta prometedora plataforma de trabajo ofrecida por los nuevos navegadores, JavaScript fue expandido en cuestión de portabilidad e integración, a la vez, interfaces de programación de aplicaciones (APIs) fueron incorporando por defecto con cada navegador para asistir a JavaScript en funciones elementales. El objetivo de esto, fue principalmente hacer disponible funciones a través de técnicas de programación sencillas y estándares, expandiendo el alcance del lenguaje y facilitando la creación de programas útiles para la Web \cite{refElGranLibro}.
		    
		    \paragraph{OpenSSL.\\}
		    OpenSSL es un proyecto de software libre basado en SSLeay, desarrollado por Eric Young y Tim Hudson.
            
            Consiste en un robusto paquete de herramientas de administración y bibliotecas relacionadas con la criptografía, que suministran funciones criptográficas a otros paquetes como OpenSSH y navegadores web (para acceso seguro a sitios HTTPS). Estas herramientas ayudan al sistema a implementar el Secure Sockets Layer (SSL), así como otros protocolos relacionados con la seguridad, como el Transport Layer Security (TLS). OpenSSL también permite crear certificados digitales que pueden aplicarse a un servidor, por ejemplo Apache \cite{opensslmillones}.
      
            \paragraph{MongoDB. \\}
            MongoDB es un sistema de base de datos multiplataforma orientado a documentos, de esquema libre, esto significa que cada entrada o registro puede tener un esquema de datos diferente, con atributos o ''columnas'' que no tienen por qué repetirse de un registro a otro. \\
            Las características más destacadas son su velocidad y su sencillo sistema de consulta de los contenidos de la base de datos. Alcanzando así un balance perfecto entre rendimiento y funcionalidad. 
            MongoDB utiliza un modelo NoSQL el cual es un modelo de agregación que se basan en la noción de agregado, entendiendo el agregado como una colección de objetos relacionados que se desean tratar de forma semántica e independiente \cite{refMongoDB}.
            
            Las ventajas que ofrece MongoDB como herramienta de desarrollo de base de datos no relacionales son:
            \begin{itemize}
                \item La base de datos no tiene un esquema de datos predefinido.
                \item El esquema puede variar para instancias de datos que pertenecen a una misma entidad.
                \item En ocasiones el gestor de la base de datos no es consciente del esquema de la base de datos.
                \item Permite reducir los problemas de concordancia entre estructuras de datos usadas por las aplicaciones y la base de datos.
                \item Frecuentemente se aplican técnicas de desnormalización de los datos.
            \end{itemize}
                    
            \paragraph{MySQL. \\}
            Es un sistema de gestión de bases de datos relacional desarrollado bajo licencia dual: Licencia pública general/Licencia comercial por Oracle Corporation y está considerada como la base de datos de código abierto más popular del mundo, y una de las más populares en general junto a Oracle y Microsoft SQL Server, sobre todo para entornos de desarrollo web \cite{refmysql}.\\
            
            El modelo relacional, para el modelado y la gestión de bases de datos, es un modelo de datos basado en la lógica de predicados y en la teoría de conjuntos.
            Su idea fundamental es el uso de relaciones. Estas relaciones podrían considerarse en forma lógica como conjuntos de datos llamados tuplas. Es el modelo más utilizado en la actualidad para modelar problemas reales y administrar datos dinámicamente \cite{refER}.\\
            
            Las principales ventajas son: 
            \begin{itemize}
                \item Provee herramientas que garantizan evitar la duplicidad de registros.
                \item Garantiza la integridad referencial, así, al eliminar un registro elimina todos los registros relacionados dependientes.
                \item Favorece la normalización por ser más comprensible y aplicable.
            \end{itemize}

            Mientras que las principales desventajas son: 
            \begin{itemize}
                \item Presentan deficiencias con datos gráficos, multimedia, CAD y sistemas de información geográfica.
                \item No se manipulan de forma eficiente los bloques de texto como tipo de dato.
            \end{itemize}
			
			\paragraph{Flask}
			\paragraph{Gunicorn}
			\paragraph{Python}
			\paragraph{Bert}
			\paragraph{GPT-2}
			\paragraph{EC2}
			\paragraph{SageMaker}
			\paragraph{CLI}
			\paragraph{SageMaker}
			\paragraph{S3}
			\paragraph{Apache}
			
			
		\subsection{Hardware.}
			Se usarán los equipos de cómputo con los que los integrantes del equipo contamos actualmente, los cuales se especifican a continuación: 
            	    \begin{table}[H]
	\begin{tabular}{|p{3.5cm}||p{10cm}|}
		\rowcolor{guindapoli}
		\multicolumn{2}{|c|}{\textbf{\textcolor{white}{Equipo de hardware utilizado. [1]}}}\\
		\hline
		\rowcolor{azulclaro}Procesador & Ryzen 5 3600\\
		\hline
		\rowcolor{white}Tarjeta de video & Amd Radeon Rx580\\
		\hline
		\rowcolor{azulclaro}Memoria RAM & 32 Gb\\
		\hline
		\rowcolor{white}Disco duro & 1Tb HDD y 512Gb SSD\\
		\hline
	\end{tabular}
\end{table}
\begin{table}[H]
	\begin{tabular}{|p{3.5cm}||p{10cm}|}
		\rowcolor{guindapoli}
		\multicolumn{2}{|c|}{\textbf{\textcolor{white}{Equipo de hardware utilizado. [2]}}}\\
		\hline
		\rowcolor{azulclaro}Marca & Apple\\
		\hline
		\rowcolor{white}Modelo & iMac Late 2012\\
		\hline
		\rowcolor{azulclaro}Procesador & Intel Core i5\\
		\hline
		\rowcolor{white}Tarjeta de video & NVIDIA GeForce GT 640M 512 Mb\\
		\hline
		\rowcolor{azulclaro}Memoria RAM & 8 Gb\\
		\hline
		\rowcolor{white}Disco duro & 1Tb y 256 Gb SSD\\
		\hline
	\end{tabular}
\end{table}
\begin{table}[H]
	\begin{tabular}{|p{3.5cm}||p{10cm}|}
		\rowcolor{guindapoli}
		\multicolumn{2}{|c|}{\textbf{\textcolor{white}{Equipo de hardware utilizado. [3]}}}\\
		\hline
		\rowcolor{azulclaro}Procesador & Amd FX-8350\\
		\hline
		\rowcolor{white}Tarjeta de video & Nvidia Geforce 1050ti\\
		\hline
		\rowcolor{azulclaro}Memoria RAM & 16 Gb\\
		\hline
		\rowcolor{white}Disco duro & 1Tb HDD\\
		\hline
	\end{tabular}
\end{table}
\newpage
%%%%%%%%%%%%%%%%%%%%%%%%%%%%%%%%%%%%%%%%%%%%%%%%%%%%%%%%%
%                                                       %
%                                                       %
%               Arquitectura de Sistema   %
%                                                       %
%                                                       %
%%%%%%%%%%%%%%%%%%%%%%%%%%%%%%%%%%%%%%%%%%%%%%%%%%%%%%%%%
\section{Arquitectura del sistema.}
	        \subsection{Descripci\'on de la arquitectura del sistema.}
	        \subsection{Aspectos Econ\'omicos}
	        Aqui va lo que vienen siendo las licencias, descripción, software a utilizar, etc.
	        \subsection{Aspectos Legales}
	         
\section{Diagrama de casos de uso general.}
% Cada caso de uso llevará un Formato tipo:
% \section{Nombre Caso de uso.} % Nombre de caso de uso
% \section{Descripción de Caso de Uso.} % Nombre de caso de uso
% \subsection{Estudio de requerimientos.} % Describir a detalle el caso de uso, 5-10 renglones
% \subsubsection{Requerimientos Funcionales.} % Todos los Requerimientos funcionales para el caso de uso
% \subsubsection{Requerimientos no Funcionales.} % Todos los Requerimientos funcionales para el caso de uso
% \subsection{Reglas del negocio.} % Martin y Odell (1998) y Russel (1995) proponen que una regla de negocio es un restricción que opera sobre el sistema para este caso de uso
.
%   %   %   %   %   %   %   %   %	%	%
%		       											%
%   				Iteración 1 					  %
%                               				  		%
%   %   %   %   %   %   %   %   %	%   %     
\chapter{\textcolor{azulescom}{Iteración 1.}}
\section{Descripción.}
\section{Base de Datos} % Explicar como se sacó la Base de Datos
\section{Género Musical} % Explicar por qué se sacó de este género musical
\section{Resultados}
%   %   %   %   %   %   %   %   %	%	%
%		       											%
%   				Iteración 2 				   %
%                               				  		%
%   %   %   %   %   %   %   %   %	%   %     
\chapter{\textcolor{azulescom}{Iteración 2.}}
\section{Descripción.}
\section{Limpieza} % Explicar cómo se limpió la base de datos
\section{Regular Expressions} % Explicar qué es una regular expression, cómo se genera y para qué sirve
\section{Técnicas de Scrapping} % Explicar la técnica que se usó para scrappear la base de datos
\section{Implementación} % Explicar como se obtuvo en código la base de datos limpia
\section{Resultados} % Explicar los Resultados
% Para agregar una cita en el documento se usa \cite{refKey}, por automático los ordena conforme se van agregando
\newpage

\begin{thebibliography}{20}
	\bibitem{refRAE}  		
	Real Academia Española (2020, noviembre), Diccionario de la lengua española, [En línea]. Disponible: https://dle.rae.es [Último acceso: 10 de diciembre del 2020].
	
	\bibitem{refOxfordLex}  		
	Oxford Lexico (2020, noviembre), Definitions, Meanings, Synonyms, and Grammar by Oxford, [En línea]. Disponible: https://www.lexico.com [Último acceso: 2 de diciembre del 2020].
	
	%  Introducción
	\bibitem{Bourreau_and_Gensollen}  		
	Chaney, D. (2012). The Music Industry in the Digital Age: Consumer Participation in Value Creation. International Journal of Arts Management, (1), pp. 15.
	
	\bibitem{Generating_Text_with_RNN}  	
	Sutskever, I., Martens, J., Hinton, G. E. (2011, January). Generating text with recurrent neural networks. In ICML.
	
	\bibitem{CNN_for_Sentence_Classification}  	
	Kim, Y. (2014). Convolutional neural networks for sentence classification. arXiv preprint arXiv:1408.5882.
	
	\bibitem{Automatic_Generation_of_Melodic_Accompaniments_for_Lyrics}  	
	Monteith, K., Martinez, T. R., \& Ventura, D. (2012, May). Automatic Generation of Melodic Accompaniments for Lyrics. In ICCC, pp. 87-94.
	
	\bibitem{Conditional_Rap_Lyrics_Generation}  	
	Nikolov, N. I., Malmi, E., Northcutt, C. G., Parisi, L. (2020). Conditional Rap Lyrics Generation with Denoising Autoencoders. arXiv preprint arXiv:2004.03965.
	
	\bibitem{What_about_the_music}  	
	Baker, F. A. (2015). What about the music? Music therapists’ perspectives on the role of music in the therapeutic songwriting process. Psychology of Music, 43(1), pp. 122-139.
	
	\bibitem{genero_musical_en_la_musica_popular}  	
	Guerrero, J. (2012). El género musical en la música popular: algunos problemas para su caracterización. Trans. Revista transcultural de música, (16), pp. 1-22.
	
	\bibitem{Bert}  	
	Wang, A., \& Cho, K. (2019). Bert has a mouth, and it must speak: Bert as a markov random field language model. arXiv preprint arXiv:1902.04094.
	
	\bibitem{spaCy}  	
	Honnibal, M.,\& Montani, I. (2017). spaCy 2: Natural language understanding with Bloom embeddings, convolutional neural networks and incremental parsing.
	
	\bibitem{refSistema}  		
	O. Jaramillo, Universidad Nacional Autonoma de México (2007, mayo 03), El concepto de Sistema, [En línea]. Disponible: https://www.ier.unam.mx/~ojs/pub/Termodinamica/node9.html [Último acceso: 2 de diciembre del 2020].
	
	\bibitem{refMaquinasFinitas}  		
	J. A. Gutiérrez Orozco, Escuela Superior de Cómputo (2008, septiembre 15), Máquinas de Estados Finitos, [En línea]. Disponible: http://delta.cs.cinvestav.mx/~mcintosh/cellularautomata/Summer\_Research\_files/maquinasef.pdf [Último acceso: 15 de diciembre del 2020].
	
	\bibitem{refInteligencia_Artificial2}
	IBM Corporation (2021, marzo 31), Artificial Intelligence (AI), [En línea]. Disponible: https://www.ibm.com/cloud/learn/what-is-artificial-intelligence [Último acceso: 6 de abril del 2021].
	
	\bibitem{refQueSonRedesNeu}  		
	L. Hardesty  (2017, abril), Explained: Neural networks, [En línea]. Disponible: https://news.mit.edu/2017/explained-neural-networks-deep-learning-0414 [Último acceso: 15 de noviembre del 2020].
	
		
	\bibitem{refNeuronaNat}
	amBientech (2019, julio 30), ¿Qué es la neurona?, [En línea]. Disponible: https://www.uv.mx/mia/files/2012/10/Artificial-Neural-Networks-for-Beginners.pdf [Último acceso: 6 de abril del 2021].
	
	\bibitem{refNeuronaArt}
	C. Gershenson (2012, octubre), Artificial Neural Networks for Beginners, [En línea]. Disponible: https://ambientech.org/la-neurona [Último acceso: 6 de abril del 2021].
	
	\bibitem{refAprendizajeRedes}  		
	Rubik's Code  (2018, febrero), How do Artificial Neural Networks learn?, [En línea]. Disponible: https://rubikscode.net/2018/01/15/how-artificial-neural-networks-learn/ [Último acceso: 7 de abril del 2021].	
	
	\bibitem{refTiposRedesNeu1}  		
	S. Leijnen, F. van Veen (2020, mayo), The Neural Network Zoo, [En línea]. Disponible: https://www.researchgate.net/publication/341373030\_The\_Neural\_Network\_Zoo [Último acceso: 20 de noviembre del 2020].
	
	\bibitem{refTiposRedesNeu2}  		
	A. Mehta (2019, enero 25), A Comprehensive Guide to Types of Neural Networks, [En línea]. Disponible: https://www.digitalvidya.com/blog/types-of-neural-networks/ [Último acceso: 15 de noviembre del 2020].
	
	\bibitem{refTiposRedesNeu3}  		
	P. Shukla, R. Iriondo (2020, agosto 11), Main Types of Neural Networks and its Applications, [En línea]. Disponible: https://medium.com/towards-artificial-intelligence/main-types-of-neural-networks-and-its-applications-tutorial-734480d7ec8e [Último acceso: 20 de noviembre del 2020].
	
	\bibitem{refTiposRedesNeu4} 
	IBM Cloud Education (2020, septiembre 14), What are Recurrent Neural Networks?, [En línea]. Disponible: https://www.ibm.com/cloud/learn/recurrent-neural-networks [Último acceso: 26 de marzo del 2021].
	
	\bibitem{refQueEsBD}  		
	Oracle México (2020, noviembre), ¿Qué es una base de datos?, [En línea]. Disponible: https://www.oracle.com/mx/database/what-is-database/ [Último acceso: 20 de noviembre del 2020].
	
	\bibitem{refEstructuraCancion1}  		
	Escribir Canciones (2008), Estructura y elementos de una canción, [En línea]. Disponible: https://dle.rae.es [Último acceso: 2 de diciembre del 2020].
	
	\bibitem{refEstructuraCancion2}  		
	Swing this Music (2008), ¿QUÉ SECCIONES PUEDE TENER UNA CANCIÓN?española, [En línea]. Disponible: https://sites.google.com/site/swingthismusiccast/interpretacio/estructura-cancion/secciones-de-una-cancion [Último acceso: 2 de diciembre del 2020].
	
	\bibitem{refMarkov}  		
	J. R. Norris (1997), Markov Chains, [En línea]. Disponible: https://cape.fcfm.buap.mx/jdzf/cursos/procesos/libros/norris.pdf [Último acceso: 15 de diciembre del 2020].
	
	\bibitem{refFlask1}  		
	Flask (2019, julio 04), Flask´s Documentation, [En línea]. Disponible: https://flask.palletsprojects.com/en/1.0.x/ [Último acceso: 16 de diciembre del 2020].
	
	\bibitem{refFlask2}  		
	Gajesh (2019, julio 17), The complete Flask beginner tutorial, [En línea]. Disponible: https://dev.to/gajesh/the-complete-flask-beginner-tutorial-124i [Último acceso: 16 de diciembre del 2020].
	
	\bibitem{refFlask3}  		
	A. Abdelaal (2019), Deploying a Flask Application to Heroku, [En línea]. Disponible: https://stackabuse.com/deploying-a-flask-application-to-heroku/ [Último acceso: 16 de diciembre del 2020].
	
	% Apache
	\bibitem{Apache}  		
	Documentation Group. (n.d.). Welcome! - The Apache HTTP Server Project. Apache. Retrieved April 6, 2021, from https://httpd.apache.org/
	\bibitem{what_is_apache}  		
	G., D. (2021, March 9). What is Apache? An In-Depth Overview of Apache Web Server. Hostinger Tutorials. https://www.hostinger.com/tutorials/what-is-apache
	% Servidor Web
	\bibitem{What_is_a_web_server}  		
	What is a web server? - Learn web development | MDN. (2021, January 27). MDN Web Docs. https://developer.mozilla.org/en-US/docs/Learn/Common\_questions/What\_is\_a\_web\_server
	% Certificado SSL
	\bibitem{SSL}  		
	https://letsencrypt.org/es/
	\bibitem{what_is_SSL} 
	What is SSL? (2017, May 15). SSLSHOPPER. https://www.sslshopper.com/what-is-ssl.html
	\bibitem{ssl_certificates} 
	Verisign. (2015, August 30). ¿Qué es un certificado SSL? – Verisign. Verisign NameStudio. https://www.verisign.com/es\_LA/website-presence/online/ssl-certificates/index.xhtml
	% Plataforma en la nube de ML 
	\bibitem{data_science_in_the_cloud} 
	Bavati, I. (2020, September 22). Data Science in the Cloud - Towards Data Science. Medium. https://towardsdatascience.com/data-science-in-the-cloud-239b795a5792 (September 2020)
	\bibitem{comparing-google-cloud-platform-aws-and-azure} 
	G. (2021, March 5). Comparing Google Cloud Platform, AWS and Azure - Georgian Impact Blog. Medium. https://medium.com/georgian-impact-blog/comparing-google-cloud-platform-aws-and-azure-d4a52a3adbd2
	\bibitem{cloud_vs_onprem} 
	J. (2020, July 30). Choosing Your Deep Learning Infrastructure: The Cloud vs. On-Prem Debate. Determined AI. https://determined.ai/blog/cloud-v-onprem
	\bibitem{what_is_jupyter} 
	I., A. (2015, October 2). Jupyter/IPython Notebook Quick Start Guide. Jupyter Notebook Beginner Guide. https://jupyter-notebook-beginner-guide.readthedocs.io/en/latest/what\_is\_jupyter.html
	\bibitem{google_cloud_vs_aws} 
	Jones, E. (2021, March 25). Google Cloud vs AWS in 2021 (Comparing the Giants). Kinsta. https://kinsta.com/blog/google-cloud-vs-aws/
	
	
	%---------------------------Referencias Originales ↓----------------------------%
	
	
	\bibitem{refMongoDB}
	MongoDB. (2019), ¿Qué es MongoDB?, [En línea]. Disponible: https://www.mongodb.com/es. [Último acceso: 26 Marzo 2021].
	
	\bibitem{refCookies}
	Universitat de Valéncia (2016), ¿Qué son las cookies? [En línea]. Disponible:           https://www.uv.es/uvweb/universidad/es/politica-privacidad/politica-cookies/son-cookies--1285919089226.html [Último acceso: 26 Marzo 2021]
	
	\bibitem{refNodeOpenSSL}
	Noteworthy Programming Masterpiece (2019), openssl-nodejs [En línea]. Disponible: https://www.npmjs.com/package/openssl-nodejs [Último acceso: 26 Marzo 2021]
	
	\bibitem{refTomcat}
	The Apache Software Foundation (2019), Apache Tomcat [En línea]. Disponible: http://tomcat.apache.org/index.html [Último acceso: 26 Marzo 2021]
	
	\bibitem{refmysql}
	Oracle (2018), MySQL [En línea]. Disponible: https://www.oracle.com/mysql/ [Último acceso: 26 Marzo 2021]
	
\end{thebibliography}	
\end{document}
% ESTO SI VAAAAAAAAAAAAAAAA EN TT1, tomarlo como referencia
    	%%%%%%%%%%%%%%%%%%%%%%%%%%%%%%%%%%%%%%%%%%%%%%%%%%%%%%%%%
		%                                                       %
		%                                                       %
		%               Arquitectura de Sistema                 %
		%                                                       %
		%                                                       %
		%%%%%%%%%%%%%%%%%%%%%%%%%%%%%%%%%%%%%%%%%%%%%%%%%%%%%%%%%
    	% ESTO SI VAAAAAAA
    	\begin{comment} 
    	\section{Arquitectura del sistema.}
            \begin{figure}[H]
        		\begin{center}
        		\includegraphics[width=15cm]{./imagenes/Analisis/ArquitecturaSistema.png}
        		\caption{Arquitectura General del Sistema}
	            \end{center}
	        \end{figure}    
	        \subsection{Descripci\'on de la arquitectura del sistema.}
	            El sistema se compone de 3 grandes bloques los cuales se comunicar\'an v\'ia red:
	            \begin{enumerate}
	                \item \textbf{Navegador Chrome con la Extensi\'on instalada}: Este primer bloque es el que se encuentra interactuando directamente con el usuario de nuestro sistema, consiste en la extensión creada por nosotros y el navegador en el que el usuario realiza peticiones a diferentes servicios en la web.
	                \item \textbf{Servidor autentificador:} Este bloque va ser el encargado de generar los certificados para cada usuario que se registre en la extensión y enviarlos a la extensión. Para la generación de dichos certificados utilizaremos una autoridad certificadora con lo que garantizamos la seguridad de estos mismos. Por otro lado para almacenar los datos de nuestros usuarios contaremos con una tabla que contenga como principales campos:
	                    \begin{itemize}
	                        \item Usuario
	                        \item Contraseña
	                        \item Certificado
	                    \end{itemize}
	                \item \textbf{Servidor web con API instalada: }En este módulo el servicio web contar\'a con una API, que se encargará de reconocer las peticiones que se reciban con nuestro método de autenticación y ser\'a la encargada de interpretar los datos y facilitarle la informaci\'on de autenticaci\'on al servicio.
	                Es importante destacar que el servicio almacenar\'a el certificado en cuesti\'on para que el usuario pueda autenticarse la próxima vez de forma automática.
	            \end{enumerate}
	        
	        Es importante mencionar que la comunicaci\'on entre cada uno de los bloques se realizar\'a mediante t\'ecnicas que permitan la confidencialidad de los datos, por un lado la comunicación del certificado que viajará entre la extensión y el servicio web se encontrará oculto mediante Chaffing and Winnowing y el patr\'on necesario para el m\'etodo se encontrar\'a cifrado mediante RSA. La comunicación entre el Servidor autentificador y la Extensión se encontrar\'a oculto mediante un socket seguro. 
	        
	        ESTA PARTE ES IMPORTANTE
	        \subsection{Aspectos Econ\'omicos}
	        Aqui va lo que vienen siendo las licencias, descripción, software a utilizar, etc.
	        \subsection{Aspectos Legales}
    	contenidos...
    	\section{Diagrama BPMN}
        	\begin{figure}[H]
            		\begin{center}
            		\includegraphics[width=14cm]{./imagenes/Analisis/BPMN.png}
            		\caption{Diagrama de Modelo y Notación de Procesos de Negocio}
    	            \end{center}
    	    \end{figure}   
    	    
    	    \subsection{Diagrama BPMN Proceso: Inicio de Sesión}
        	    \begin{figure}[H]
                		\begin{center}
                		\includegraphics[width=12cm]{./imagenes/Analisis/BPMN_IS.png}
                		\caption{Diagrama BPMN del proceso Iniciar Sesión}
        	            \end{center}
        	    \end{figure}   
    	        
	    \section{Diagrama de casos de uso general.}
	        \begin{figure}[H]
        		\begin{center}
        		\includegraphics[width=13cm]{./imagenes/Analisis/UCD_General.png}
        		\caption{Diagrama de casos de uso general del sistema}
	            \end{center}
            \end{figure}   
	        
	        NOTA: En la sección 4 "Diseño", cada prototipo describirá sus casos de uso correspondientes para su funcionamiento.
	
		%%%%
		%
		% El caso de uso de "registrarse en el servidor", "revocar" autentificador sería de la autoridad o de la extensión o ambos
		%
		%
		%%%%
		
		\section{Componente I. Extensión.}
		    \subsection{Descripción.}
		    
		    Este componente permite a la extensión poder interceptar peticiones hechas por el usuario a través del navegador de Google Chrome.\\
		    Una vez que se intercepte la petición, ésta podrá ser modificada. La modificación se hará sólo mientras la extensión esté habilitada, y tiene como objetivo inyectar el certificado autentificador en el encabezado del protocolo una vez que se haya llevado a cabo el proceso de Chaffing. Cuando dicho certificado sea inyectado, la extensión deberá liberar la petición para que salga a red.\\
		    Este certificado será único por cada usuario y será obtenido desde el Componente II cuando el usuario inicie sesión en la extensión. Si el usuario no puede iniciar sesión debido a que no tiene una cuenta, desde la extensión se podrá registrar para poder obtener su certificado.\\
		    Además, el usuario podrá cerrar la sesión en la extensión si es que así lo desea, lo que eliminaría el certificado de la máquina local del usuario. Por otro lado, si el usuario desea revocar su certificado, lo podrá hacer también desde este componente.\\
		    %Para la creación de dicho certificado, se usará el algoritmo de hashing SHA-1, tomando como argumentos de entrada usuario y contraseña, generando así, una cadena resultante de 160 bits, sin embargo, para los siguientes prototipos, se manejará una autoridad certificadora la cual recibirá los datos del inicio de sesión para que genere el certificado del usuario. 
		    
		    El propósito de este prototipo es utilizar la técnica de \textit{Chaffing and Winnowing} en este nuevo método de autentificación, para evitarle al usuario la tediosa tarea de ingresar sus credenciales cada vez que accede al servicio y brindarle la seguridad necesaria al iniciar sesión.\\
		    
		    Para inyectar el certificado autentificador, es necesario crear un \textit{''patrón de chaffing''}, este patrón lo generaremos aleatoriamente para después mandarlo junto con la petición HTTP.
		    Dicho patrón irá cifrado con la clave pública del Componente III (API). El objetivo de mandar el patrón junto con el protocolo HTTP, es que el servidor pueda descifrar el patrón con su clave privada y con él realizar la etapa de \textit{winnowing} para extraer el certificado.\\
		    %Por el momento, dicho patrón no irá cifrado, ya que para esto es necesario la implementación del servidor del servicio web de prueba, para poder conocer su llave pública y cifrar el patrón. Sin embargo, en el prototipo 1 del subsistema 3 se implementará el servidor y con ello su cifrado asimétrico. El objetivo de mandar el patrón junto con el protocolo HTTP, es que el servidor pueda leer el patrón y realizar la etapa de \textit{winnowing}, para extraer el certificado.\\
            \end{comment}


			%%%%%%%%%%%%%%%%%%%%%%%%%%%%%%%%%%%%%%%%%%%%%%%%%%%%%%%%%
			%                                                       %
			%                                                       %
			%                REQUERIMIENTOS PII                     %
			%                                                       %
			%                                                       %
			%%%%%%%%%%%%%%%%%%%%%%%%%%%%%%%%%%%%%%%%%%%%%%%%%%%%%%%%%ESTO SI VA
			\begin{comment}
		    \subsection{Estudio de requerimientos.}
				\subsubsection{Requerimientos Funcionales.}
				{\setlength{\parindent}{12pt}
				
			    \textbf{CI\_RF1. Interceptar petición \acrshort{Palabra1}.} La extensión deberá interceptar la petición \acrshort{Palabra1} del navegador, en cuanto el usuario realice alguna a través de éste.\\

				\textbf{CI\_RF2. Deshabilitar extensión.} El usuario podrá deshabilitar la extensión, para que ésta no vigile su actividad en el navegador.\\
				
				\textbf{CI\_RF3. Habilitar extensión.} El usuario podrá habilitar la extensión, para que ésta vigile las peticiones \acrshort{Palabra1}.\\
				
				\textbf{CI\_RF4. Validar petición.} La extensión deberá analizar la petición previamente interceptada, y validar si ésta es HTTP(S) o no.\\
				
				\textbf{CI\_RF5. Interceptar petición.} La extensión deberá evitar que la petición salga a red, deteniéndola para aplicar la etapa de \textit{Chaffing}.\\ 
				
				\textbf{CI\_RF6. Inicio de sesi\'on en la extensi\'on.} La extensi\'on contar\'a con una interfaz para el ingreso de datos del usuario, donde ingresará un usuario y contraseña. \\  
				
                \textbf{CI\_RF7. Obtención del certificado autentificador.} La extensión, mediante el inicio de sesión del usuario, se conectará al Componente II (Servidor autentificador) para obtener el certificado autentificador.\\ 
                
                \textbf{CI\_RF8. Almacenamiento del certificado autentificador.} La extensi\'on deberá almacenar el certificado autentificador devuelto por la autoridad certificadora.\\
                
                \textbf{CI\_RF9. Generación de patrón de \textit{Chaffing}.} La extensión generará un patrón para poder implementar el método de \textit{Chaffing}. Este patrón será generado al azar, y es aquel que se usará para introducir el código autentificador en el protocolo HTTP.\\
                
				\textbf{CI\_RF10. Etapa de \textit{Chaffing}.} Por medio del m\'etodo \textit{Chaffing} se agregará al encabezado HTTP el certificado del usuario, utilizando el patrón de \textit{Chaffing} del requerimiento funcional CI\_RF9. Generación de patrón de \textit{Chaffing}\\
                    
                \textbf{CI\_RF11. Liberaci\'on de Petición.} Se liberar\'a el bloqueo a la petici\'on HTTP impuesto por el requerimiento funcional CI\_RF5. Interceptar petición.\\
                
                \textbf{CI\_RF12. Cierre de sesión en la extensión.} El usuario podrá cerrar sesión en la extensión para que ésta no siga guardando su certificado.\\
                
                \textbf{CI\_RF13. Registro en servidor autentificador.} La extensión contará con una interfaz para el ingreso de datos del usuario, para que éste se registre en el servidor y pueda obtener un certificado. Los datos a ingresar seŕan: correo electrónico (email), y contraseña.\\
                \label{CI_RF13}
                
                \textbf{CI\_RF14. Revocar certificado.} El usuario podrá, mediante la interfaz de la extensión, revocar su certificado en caso de que así lo deseé. La interfaz requerirá que se introduzca el email correspondiente al usuario y la contraseña para validar la identidad del usuario. \\
                
                \textbf{CI\_RF15. Ver contraseña.} Si el usuario así lo desea, podrá habilitar la función ver contraseña cuando escribe. Este requerimiento se usara en los requerimientos CI\_RF6. Inicio de sesión en la extensión y CI\_RF13. Registro en servidor autentificador.\\
                
                
		        }
				
				\subsubsection{Requerimientos no Funcionales.}
				{\setlength{\parindent}{12pt}
				
				\textbf{CI\_RNF1. Plataforma de implementación.} La extensión será implementada en el navegador Google Chrome Desktop.\\
				
				\textbf{CI\_RNF2. Versión del navegador} La extensión funcionará a partir de la versión 28.0.\\
				
				\textbf{CI\_RNF3. Tecnologías para la interfaz de usuario} Para el sistema se hará uso de HTML5, JavaScript, CSS3, JSON.\\
				
				\textbf{CI\_RNF4. Codificación en base64 del Chaffing.}Se codificará en base64 la cadena de chaffing para permitir el envío de los datos, debido a que el algoritmo nos devuelve ciertos caracteres no permitidos para enviar.\\
				\label{CI_RNF4}
				
				\textbf{CI\_RNF5. Conexión a internet.} Para el funcionamiento de la extensión, no es necesario que se tenga conexión a internet.\\
				
				\textbf{CI\_RNF6. Tamaño del código autentificador.} El tamaño del código autentificador es de aproximadamente 10496 bits pero puede variar dependiendo de la información introducida para su elaboración.\\
				
				\textbf{CI\_RNF7. Almacenado de archivo en la extensión.} Se necesita tener almacenado el archivo en la extensión de Google Chrome. Específicamente, utilizamos el \textbf{Storage} para almacenarlo. \\
				
			    }
		    
		    \subsection{Reglas del negocio.}
		    %Martin y Odell (1998) y Russel (1995) proponen que una regla de negocio es un restricción que opera sobre el sistema.
		    
		    \textbf{CI\_RN1. Extensión habilitada.} En cuanto el usuario lo indique por medio de la \acrlong{acronimo1}, la extensión deberá vigilar la actividad que éste realice en el navegador para interceptar una petición.\\
			\label{CI_RN1}
			
			\textbf{CI\_RN2. Extensión deshabilitada.} En cuanto el usuario lo indique por medio de la \acrlong{acronimo1}, la extensión deberá dejar de vigilar la actividad que éste realice en el navegador.\\
			\label{CI_RN2}
			
            \textbf{CI\_RN3. Petición válida.} La extensión modificará la petición siempre y cuando se trate de una petición válida HTTP .\\
            \label{CI_RN3}
            
            \textbf{CI\_RN4. Inicio de sesión de extensión por usuario.} Cada usuario que desee utilizar la extensión sólo deberá tener una cuenta con un correo electrónico y una contraseña respectiva a este usuario. \\
            \label{CI_RN4}
            
            \textbf{CI\_RN5. Acceso a internet.} Se debe de contar con acceso a internet para que la extensión pueda enviar al servidor la petición modificada.\\
            \label{CI_RN5}
				
			\textbf{CI\_RN6. Longitud de código autentificador.} La longitud del código autentificador es de aproximadamente 10496 bits pero puede variar dependiendo de la información introducida para su elaboración.\\
			\label{CI_RN6}
			
			\textbf{CI\_RN7. Longitud del campo de usuario.} La longitud del usuario no debe pasar de 20 caracteres, y mínimo será de 3 caracteres.\\
			\label{CI_RN7}
			
			\textbf{CI\_RN8. Longitud del campo contraseña.}
			La longitud de la contraseña no debe pasar de 16 caracteres, y mínimo sera de 8 caracteres.\\
			\label{CI_RN8}
			
			\textbf{CI\_RN9. Caracteres permitidos en campo usuario.} Los caracteres permitidos en el campo de usuario son únicamente símbolos alfanuméricos y guion bajo.\\
			\label{CI_RN9}
			
			\textbf{CI\_RN10. Caracteres permitidos en campo contraseña.} Los caracteres permitidos en el campo de contraseña son únicamente símbolos alfanuméricos así como los s\'imbolos \_ @ / \# ? .\\
			\label{CI_RN10}
			
			\textbf{CI\_RN11. Formato de contraseña.} La contraseña debe tener al menos un número y una letra mayúscula.\\
			\label{CI_RN11}
			
			\textbf{CI\_RN12. Longitud del campo email.}
			La longitud del email no debe pasar de 100 caracteres, y mínimo sera de 7 caracteres.\\
			\label{CI_RN12}
			
			\textbf{CI\_RN13. Caracteres permitidos en campo email.}
			Los caracteres permitidos en el campo email son únicamente símbolos alfanuméricos así como los símbolos '.' '@' '\_' '-' ',' '()\{\}[]'.\\
			\label{CI_RN13}
				
	    %%%%%%%%%%%%%%%%%%%%%%%%%%%%%%%%%%%%%%%%%%%%%%%%%%%%%%%%%
		%                                                       %
		%                                                       %
		%       Análisis CII: SERVIDOR AUTENTIFICADOR           %
		%                                                       %
		%                                                       %
		%%%%%%%%%%%%%%%%%%%%%%%%%%%%%%%%%%%%%%%%%%%%%%%%%%%%%%%%%
	        
	    \section{Componente II: Servidor autentificador.}
	        \subsection{Descripción.}
	        Para este componente, implementaremos un servidor autentificador en el cual se crearán y almacenarán los certificados de los usuarios. En este servidor, los usuarios tendrán registrada una cuenta a la cual accederán desde el Componente I. Una vez que el usuario inicie sesión en la extensión, este servidor autentificador regresará como respuesta el certificado generado para que la extensión lo almacene. 
	        Para la creación del certificado autentificador se utilizará la herramienta OpenSSL, además, la comunicación entre extensión y servidor se hará bajo SSL/TLS.\\
	        
	        La principal función de este componente es gestionar las cuentas y certificados en una base de datos, de tal forma que el Componente 1 se pueda comunicar con este componente ya sea para que le genere un certificado a una cuenta (en el caso de que el usuario se esté registrando en el servidor autentificador por primera vez), para mandarle su certificado si es que ya se encuentra registrado en el servidor, o bien, para revocar un certificado. Así mismo, este componente también estará comunicado con el Componente III, específicamente la API, para controlar la revocación de certificados.\\
	        
	        El propósito de este componente es poder crear y utilizar un certificado real, generado por una autoridad certificadora de confianza y con el cual se pueda autentificar a un usuario en un servicio web.\\
	       
	        \subsection{Estudio de requerimientos.}
	            \subsubsection{Requerimientos funcionales.}
	            {\setlength{\parindent}{12pt}
				
				\textbf{CII\_RF1. Creación de nuevo usuario.} La autoridad certificadora podrá crear una nueva instancia en la base de datos de acuerdo a los datos recuperados por la extensión (Usuario y contraseña), si es que estos no se encuentran guardados en la base de datos.\\

				\textbf{CII\_RF2. Generar certificado.} La autoridad certificadora deberá generar un certificado diferente para cada usuario que se registre en la extensión con los datos proporcionados por la misma.\\
				
				\textbf{CII\_RF3. Asignar certificado.} La autoridad certificadora deberá asignar el certificado generado al usuario correspondiente en la base de datos.\\
				
				\textbf{CII\_RF4. Devolver certificado.} La autoridad certificadora deberá enviar el certificado correspondiente al usuario quien realiza la solicitud para obtener el certificado.\\
				
				\textbf{CII\_RF5. Actualizar certificado.} La autoridad certificadora deberá actualizar el certificado cuanto éste haya caducado.\\
				
				\textbf{CII\_RF6. Revocar certificado.} La autoridad certificadora deberá revocar un certificado de acuerdo a la petición dada por el usuario, si esta petición tiene los datos correctos del usuario, su certificado ligado al mismo se eliminará, se creará uno nuevo y se le asignará a dicho usuario.\\
				
				\textbf{CII\_RF7. Comprobar certificado certificado.} La autoridad certificadora deberá poder comprobar la existencia de un certificado basándose en el código hash recibido por parte del Componente III. 
                
		        }
		        
	            \subsubsection{Requerimientos no funcionales.}
	            {\setlength{\parindent}{12pt}
				
				\textbf{CII\_RNF1. Plataforma de implementación.} La autoridad certificadora será implementada en NodeJs.\\
				
				\textbf{CII\_RNF2. Version de SSL/TLS.} La autoridad certificadora usará como conexión, así como la generación de certificado, OpenSSL con la versión 1.1.1.\\
				
				\textbf{CII\_RNF3. Base de datos.} La autoridad certificadora se conectará a una base de datos MongoDB 4.0.10.\\
				
				\textbf{CII\_RNF4. Formato de certificado.} El servidor autentificador deberá crear certificados X509.
			}
	            
	            
	        \subsection{Reglas del negocio.}
	        
	        \textbf{CII\_RN1. Acceso a internet.} Se debe de contar con acceso a internet para que la autoridad certificadora pueda enviar a la extensión el certificado generado.\\
            \label{CII_RN1}
            
            \textbf{CII\_RN2. Conexión a la base de datos.} La autoridad certificadora cuenta con un pull de conexiones a la base de datos para poder insertar u obtener certificados de los usuarios.\\
            \label{CII_RN2}
            
            \textbf{CII\_RN3. Conexión SSL.} La autoridad certificadora cuenta con un certificado auto firmado, el cual le permitirá tener una conexión SSL con la extensión para poder así enviar el certificado de dicho usuario de manera segura y cifrada.\\
            \label{CII_RN3}
          
            \textbf{CII\_RN4. Parámetros POST.} La autoridad certificadora recibirá mediante método POST peticiones las cuales tendrán como datos: Correo electronico (email) y password (Con hash 256).\\
            \label{CII_RN4}
            
            \textbf{CII\_RN5. Parámetros GET.} La autoridad certificadora enviará una pagina como respuesta si el usuario aprueba el uso del certificado de la autoridad certificadora.\\
            \label{CII_RN5}
            
            \textbf{CII\_RN6. Respuestas a peticiones.} La autoridad certificadora podrá devolver un certificado creado por ella misma. Así como también devolver códigos de respuestas, como lo son: 200 (Se realizo petición correctamente) ó 404 (No se encontró usuario).\\
            \label{CII_RN6}
			
			%%%%%%%%%%%%%%%%%%%%%%%%%%%%%%%%%%%%%%%%%%%%%%%%%%%%%%%%%
    		%                                                       %
    		%                                                       %
    		%            Análisis CIII: API                %
    		%                                                       %
    		%                                                       %
    		%%%%%%%%%%%%%%%%%%%%%%%%%%%%%%%%%%%%%%%%%%%%%%%%%%%%%%%%%
	       
        \section{Componente III: API.}
            
            \subsection{Descripción.}
	        
    	        Para el componente III, se utilizará un servicio web de prueba para mostrar la funcionalidad del inicio de sesión por este método propuesto, por lo cual dicho servicio no se ve reflejado en el análisis desarrollo del Componente III.
    	        
    	        Se realizará la etapa de \textit{winnowing} a la petición HTTP para obtener el certificado. Esto último lo realizará una API dedicada exclusivamente a ello, con conexión al Componente II para checar el estatus y validez del certificado. 
    	        
    	        El servicio web sólo tendrá que conectarse al API para obtener el certificado y su validez, con base en ello realizar la lógica del negocio para iniciar sesión.
    	        
    	        Así mismo, el API implementará un cifrado asimétrico (RSA), esto con la finalidad de que el Componente I pueda tomar la llave pública del servidor y cifrar el \textit{''patrón de chaffing''} que se mandará en la petición. Una vez que el patrón y el chaffing llegue al API, sólo ella podrá descifrar el patrón con su llave privada, dandole la integridad necesaria al patrón para viajar a través de la red.\\
                
                El propósito de este componente es poder obtener el certificado que identificará a cada usuario, para poder validarlo y saber si se debe de dar o negar el acceso. Cabe mencionar que, la primera vez que el servidor reciba el certificado autentificador en el servicio, éste pedirá al usuario que se inicie sesión con la finalidad de poder asociar este certificado a una cuenta del servicio; para las peticiones posteriores, el certificado ya contará con una cuenta asociada a la cual podrá dar acceso siempre y cuando el certificado sea válido.\\
            
	        \subsection{Estudio de requerimientos.}
	            \subsubsection{Requerimientos funcionales.}
	            
	                %% Involucran al servicio
	            
    	            \textbf{CIII\_RF1. Primer inicio de sesión con chaffing.} El servicio deberá guardar el certificado en caso de que sea la primera vez que le llega y si es válido, asociandolo a un usuario por medio de un inicio de sesión.\\
    	            \label{CIII_RF1}
    	            
    	            \textbf{CIII\_RF2. Inicio de sesión con chaffing.} El servicio web, en caso de que el certificado sea válido, deberá de dar acceso a la cuenta del usuario sin la necesidad de tener que pedir las credenciales del mismo, a excepción de lo establecido en CIII\_RF1. Primer inicio de sesión con chaffing.\\
    	            \label{CIII_RF2}
    	            
    	            \textbf{CIII\_RF3. Negación del inicio de sesión.} En caso de que el certificado de usuario no sea válido, el servicio deberá negar el acceso al usuario.\\
    	            \label{CIII_RF3}
    	            
    	            \textbf{CIII\_RF4. Envió de chaffing y patrón de chaffing.} El servicio web deberá mandar, en caso de que los reciba, el chaffing y el patrón de chaffing al API para que ésta pueda realizar el winnowing sobre el chaffing.\\
    	            \label{CIII_RF4}
    	            
    	            %% Involucran al API
    	            
    	            \textbf{CIII\_RF5. Comunicación con autoridad certificadora.} El API desarrollada en este componente, deberá tener comunicación con el Componente II. Servidor Autentificador para verificar la validez del certificado.\\
    	            \label{CIII_RF5}
    	            
    	            \textbf{CIII\_RF6. Generación de llaves.} El API deberá tener su par de llaves, pública y privada, para poder realizar el cifrado asimétrico al \textit{'patrón de chaffing'}.\\
    	            \label{CIII_RF6}
    	            
    	            \textbf{CIII\_RF7. Descifrado del patrón de chaffing.} El API deberá ser capaz de descifrar el patrón de chaffing con su llave privada, para poder realizar el winnowing sobre el chaffing.\\
    	            \label{CIII_RF7}
    	            
    	            \textbf{CIII\_RF8. Realización de la etapa de winnowing.} El API deberá de poder realizar la etapa de winnowing para poder obtener el certificado.\\
    	            \label{CIII_RF8}
    	            
    	            \textbf{CIII\_RF9. Validación de firma.} Posteriormente a la etapa de winnowing del CIII\_RF8, este tiene que validar que la firma del certificado, para saber si este fue emitido por nuestro mismo servidor autentificador. \\
	                \label{CIII_RF9}
	                
	                \textbf{CIII\_RF10. Retorno de certificado.} La API deberá de retornarle el certificado y su estatus al servicio para que este pueda realizar la lógica del negocio necesaria para el inicio de sesión.\\
	                \label{CIII_RF10}
	                
	                \textbf{CIII\_RF11. Descifrado AES al patrón.} La API deberá de descifrar el patrón utilizando AES.\\
	                \label{CIII_RF11}
	       
	            \subsubsection{Requerimientos no funcionales.}
	                
	                \textbf{CIII\_RNF1. Plataforma de implementación.} Las pruebas del servidor web serán realizadas en el servidor Apache Tomcat.\\
                    \label{CIII_RNF1}

	                \textbf{CIII\_RNF2. Base de datos.} Se utilizará una base de datos en MySQL para guardar la información de los usuarios.\\
	                \label{CIII_RNF2}
	                
	                \textbf{CIII\_RNF3. Tamaño de llaves.} Se utilizarán llaves de un tamaño de 2048 bits para el cifrado de tipo RSA.\\
	                \label{CIII_RNF3}
	                
	                \textbf{CIII\_RNF4. Framework de desarrollo.} Se utilizará Spring Framework Core 5 para implementar la API con JAVA 8.\\
	                \label{CIII_RNF4}
	       
            \subsection{Reglas del negocio.}
	            \textbf{CIII\_RN1. El servicio debe contar con una correcta conexi\'on.} Se debe de contar con una conexión estable tanto a internet para que los usuarios finales puedan acceder como al servidor autentificador necesario para el correcto funcionamiento.\\
	            \label{CIII_RN1}
	            
	            \textbf{CIII\_RN2. Conexión a base de datos.} El servidor debe tener una correcta conexión a la base de datos tanto del negocio como la modificación necesaria para que el inicio de sesión implementado funcione correctamente.\\
	            \label{CIII_RN2}
	            
	            \textbf{CIII\_RN3. API en el servidor.} El servidor debe de contar con la API para que pueda realizar el procesos de \textit{winnowing}.\\
	            \label{CIII_RN3}
	            
	            \textbf{CIII\_RN4. Inicio de Sesi\'on.} El servidor debe contar con dos inicios de sesión, uno propio del negocio y otro propio del incio de sesión con \textit{Chaffing and Winnowing}.\\
	            \label{CIII_RN4}
	            \end{comment}
	            
	    %   %   %   %   %   %   %   %   %   %
	    %		        Capítulo 4					%
	    %   			DISEÑO 						%
	    %                               				  %
	    %   %   %   %   %   %   %   %   %	%
	    \begin{comment}
	\chapter{\textcolor{azulescom}{Diseño.}}
			
		\section{Componente I: Extensión.}
		 
		    \subsection{Diagrama de casos de uso}
			    
			    \begin{figure}[H]
		            \begin{center}			                \includegraphics[width=14cm]{./imagenes/Disenio/Componente_1/CI_UCD.png}
						\caption{Diagrama de casos de uso del Componente I.}
                    \end{center}    					
				\end{figure}
			    
	        \subsubsection{Descripción de casos de uso.}
			    
			    %DESCRIPCIÓN CI_CU1
			    \begin{table}[H]
				\begin{tabular}{ |p{3.5cm}||p{9.5cm}|}
					\hline
					\rowcolor{guindapoli}
					\multicolumn{2}{|c|}{\textbf{\textcolor{white}{Caso de uso: CI\_CU1. Realizar petición.}}}\\
					\hline
					\rowcolor{azulfuerte}Concepto & Descripción\\
					\hline
					\cellcolor{azulclaro}Actor & 
					Navegador de Google Chrome.\\ 
					\hline
					\cellcolor{azulclaro}Propósito &
					Este caso de uso permite al navegador realizar una petición, ordenada por el usuario o un sistema externo.\\
					\hline
					\cellcolor{azulclaro}Entradas &
					URL del servicio web solicitado.\\
					\hline
					\cellcolor{azulclaro}Salidas &
					Petición.\\
					\hline
					\cellcolor{azulclaro}Pre-condiciones&
					Algún agente externo (Sistema o usuario) ha ordenado al navegador mandar una petición.\\
					\hline
					\cellcolor{azulclaro}Post-condiciones&
					Creación de la petición HTTP.\\
					\hline
					\cellcolor{azulclaro}Reglas del negocio&
					-\\
					\hline
					\cellcolor{azulclaro}Errores &
					La petición no se pudo realizar. \newline La petición no es tipo \acrshort{Palabra1}.\\					
					\hline
				\end{tabular}
				\caption[DCU: CI\_CU1]{Descripción CU: CI\_CU1}
				\end{table}
				
				\paragraph{... Trayectoria Principal ...}
				\begin{enumerate}
					\item \textbf{\textit{El Usuario}} o \textbf{\textit{El Sistema Externo}} realiza una petición \acrshort{Palabra1} en el navegador Google Chrome.\\
					\item \textbf{\textit{El navegador}} realiza la petición.\\
				\end{enumerate}
				\paragraph{... Fin de la Trayectoria Principal ...}
				
				\paragraph{... Trayectoria Alternativa 1 ...}
				\begin{enumerate}
					\item \textbf{\textit{El Usuario}} o \textbf{\textit{El Sistema Externo}} realiza una petición que no es \acrshort{Palabra1} en el navegador Google Chrome.\\
					\item \textbf{\textit{El navegador}} realiza la petición.					
				\end{enumerate}
				\paragraph{... Fin de la Trayectoria Alternativa 1 ...}
				\newpage
				
				%DESCRIPCIÓN CI_CU2
				\begin{table}[H]
				\begin{center}
				\begin{tabular}{ |p{3.5cm}||p{9.5cm}|}
					\hline
					\rowcolor{guindapoli}
					\multicolumn{2}{|c|}{\textbf{\textcolor{white}{Caso de uso: CI\_CU2. Habilitar extensión.}}}\\
					\hline
					\rowcolor{azulfuerte}Concepto & Descripción\\
					\hline
					\cellcolor{azulclaro}Actor & 
					Usuario.\\ 
					\hline
					\cellcolor{azulclaro}Propósito &
					Este caso de uso, permite al usuario habilitar la extensión, para que ésta sea capaz de ver todas las peticiones que realiza el navegador.\\
					\hline
					\cellcolor{azulclaro}Entradas &
					Indicación de habilitar extensión, mediante interfaz de usuario.\\
					\hline
					\cellcolor{azulclaro}Salidas &
					-\\
					\hline
					\cellcolor{azulclaro}Pre-condiciones&
					{\textbf{CI\_CU3.}}\\
					\hline
					\cellcolor{azulclaro}Post-condiciones&
					-\\
					\hline
					\cellcolor{azulclaro}Reglas del negocio&
					\hyperref[CI_RN1]{\textbf{CI\_RN1.}}\\
					\hline
					\cellcolor{azulclaro}Errores &
					No se puede habiliar la extensión.\\
					\hline
				\end{tabular}
				\caption[DCU: CI\_CU2]{Descripción CU: CI\_CU2}
				\end{center}
				\end{table}
				
				\paragraph{... Trayectoria Principal ...}
				\begin{enumerate}
					
					\item \textbf{\textit{El usuario}} da click en el ícono de la extensión.
					
					\item \textbf{\textit{El usuario}} da click en el botón 'Activar'.
					
					\item \textbf{\textit{La extensión}} empieza a vigilar las peticiones que se realicen a través del navegador.
					
				\end{enumerate}
				\paragraph{... Fin de la Trayectoria Principal ...}
				
				\paragraph{... Trayectoria Alternativa 1 ...}
				\begin{enumerate}
				
				    \item \textbf{\textit{La extensión}} no muestra el botón 'Activar', por ende el usuario no puede dar click.
				
				\end{enumerate}
				\paragraph{... Fin de la Trayectoria Alternativa 1 ...}
				\newpage
				
				%DESCRIPCIÓN CI_CU3
				\begin{table}[H]
				\begin{center}
				\begin{tabular}{ |p{3.5cm}||p{9.5cm}|}
					\hline
					\rowcolor{guindapoli}
					\multicolumn{2}{|c|}{\textbf{\textcolor{white}{Caso de uso: CI\_CU3. Deshabilitar extensión.}}}\\
					\hline
					\rowcolor{azulfuerte}Concepto & Descripción\\
					\hline
					\cellcolor{azulclaro}Actor & 
					Usuario.\\ 
					\hline
					\cellcolor{azulclaro}Propósito &
					Este caso de uso permite al usuario deshabilitar la extensión, para que ésta ignore todas las peticiones que se realicen por medio del navegador.\\
					\hline
					\cellcolor{azulclaro}Entradas &
					Indicación de deshabilitar extensión, mediante interfaz de usuario.\\
					\hline
					\cellcolor{azulclaro}Salidas &
					-\\
					\hline
					\cellcolor{azulclaro}Pre-condiciones&
				    {\textbf{CI\_CU2.}}\\
					\hline
					\cellcolor{azulclaro}Post-condiciones&	-\\
					\hline
					\cellcolor{azulclaro}Reglas del negocio&
					\hyperref[CI_RN2]{\textbf{CI\_RN2.}}\\
					\hline
					\cellcolor{azulclaro}Errores &
					No se puede deshabilitar la extensión.\\
					\hline
				\end{tabular}
				\caption[DCU: CI\_CU3]{Descripción CU: CI\_CU3}
				\end{center}
				\end{table}
			
				\paragraph{... Trayectoria Principal ...}
				\begin{enumerate}
					
					\item \textbf{\textit{El usuario}} da click en el ícono de la extensión.
					
					\item \textbf{\textit{El usuario}} da click en el botón.
					
					\item \textbf{\textit{La extensión}} deja de vigilar las peticiones que se realicen a través del navegador.
				
				\end{enumerate}
				\paragraph{... Fin de la Trayectoria Principal ...}
				
		        \paragraph{... Trayectoria Alternativa 1 ...}
				\begin{enumerate}
				     \item \textbf{\textit{La extensión}} no muestra el botón, por ende el usuario no puede dar click.
				\end{enumerate}
				\paragraph{... Fin de la Trayectoria Alternativa 1 ...}
				\newpage 
				
				
				%%DESCRIPCIÓN CI_CU4
				\begin{table}[H]
    				\begin{tabular}{ |p{3.5cm}||p{9.5cm}|}
    					\hline
    					\rowcolor{guindapoli}
    					\multicolumn{2}{|c|}{\textbf{\textcolor{white}{Caso de uso: CI\_CU4. Inicio de sesión en la extensión}}}\\
    					\hline
    					\rowcolor{azulfuerte}Concepto & Descripción\\
    					\hline
    					\cellcolor{azulclaro}Actor & 
    					Usuario\\ 
    					\hline
    					\cellcolor{azulclaro}Propósito &
    					Este caso de uso permite al usuario poder iniciar sesión en la extensión para posteriormente obtener un código autentificador, es decir, el certificado del usuario, al cual se le aplicará \textit{Chaffing}. \newline
    					Sólo se requerirá iniciar sesión una sola vez ya que después de ello el certificado se guardará en la extensión.\\
    					\hline
    					\cellcolor{azulclaro}Entradas &
    					Usuario y Contraseña\\
    					\hline
    					\cellcolor{azulclaro}Salidas &
    					Certificado del usuario que acaba de iniciar sesión.\\
    					\hline
    					\cellcolor{azulclaro}Pre-condiciones&
    					Haber instalado la extensión en el navegador Google Chrome y haberla habilitado.\\
    					\hline
    					\cellcolor{azulclaro}Post-condiciones&
    					Por medio de los datos introducidos por el usuario, se obtendrá el certificado del mismo para que se pueda realizar la etapa de chaffing al momento que deseé identificarse en un servicio web.\\
    					\hline
    					\cellcolor{azulclaro}Reglas del negocio&
    					\hyperref[CI_RN2]{\textbf{CI\_RN1.}} \newline \hyperref[CI_RN3]{\textbf{CI\_RN4.}} \newline
    					\hyperref[CI_RN6]{\textbf{CI\_RN7.}} \newline
    					\hyperref[CI_RN7]{\textbf{CI\_RN8.}} \newline
    					\hyperref[CI_RN8]{\textbf{CI\_RN9.}} \newline
    					\hyperref[CI_RN9]{\textbf{CI\_RN10.}} \newline
    					\hyperref[CI_RN10]{\textbf{CI\_RN11.}} \\
    					\hline
    					\cellcolor{azulclaro}Errores &
    					No se encuentra usuario registrado. \newline 
    					No se puede obtener el código autentificador.\newline
    					No se pudo guardar el código autentificador.\newline
    					Contraseña no válida.\newline
    					Email no válido.\\					
    					\hline
    				\end{tabular}
				\caption[DCU: CI\_CU4]{Descripción CU: CI\_CU4}
				\end{table}
				
				\paragraph{... Trayectoria Principal ...}
				\begin{enumerate}
					
					\item \textbf{\textit{El usuario}} da click en el ícono de la extensión.
					
					\item \textbf{\textit{EL usuario}} da click en el botón 'Iniciar sesión'.
					
					\item \textbf{\textit{La extensión}} abre en una nueva pestaña del navegador la página principal de la extensión.
					
					\item \textbf{\textit{El usuario}} llena correctamente los campos del formulario, los cuales son: 'Email' y 'Contraseña'.
					
					\item \textbf{\textit{EL usuario}} da click en el botón 'Iniciar sesión'.
					
					\item \textbf{\textit{La extensión}} despliega un popup con la información de la obtención del certificado.
					
					\item \textbf{\textit{El usuario}} da click en el botón 'Aceptar'.
				
				\end{enumerate}
				\paragraph{... Fin de la Trayectoria Principal ...}
				
				
				\paragraph{... Trayectoria Alternativa 1 ...}
				\begin{enumerate}
				
					\item \textbf{\textit{El usuario}} da click en el ícono de la extensión.
					
					\item \textbf{\textit{EL usuario}} da click en el botón 'Iniciar sesión'.
					
					\item \textbf{\textit{La extensión}} abre en una nueva pestaña del navegador la página principal de la extensión.
					
					\item \textbf{\textit{El usuario}} da click en el botón 'Iniciar sesión'.
					
					\item \textbf{\textit{La extensión}} abre la página de inicio de sesión.
					
					\item \textbf{\textit{El usuario}} llena correctamente los campos del formulario, los cuales son: 'Email' y 'Contraseña'.
					
					\item \textbf{\textit{EL usuario}} da click en el botón 'Iniciar sesión'.
					
					\item \textbf{\textit{La extensión}} despliega un popup con la información de la obtención del certificado.
					
				\end{enumerate}
				\paragraph{... Fin de la Trayectoria Alternativa 1 ...}
			    
			    
			    \paragraph{... Trayectoria Alternativa 2 ...}
				\begin{enumerate}
					
					\item \textbf{\textit{El usuario}} da click en el ícono de la extensión.
					
					\item \textbf{\textit{EL usuario}} da click en el botón 'Iniciar sesión'.
					
					\item \textbf{\textit{La extensión}} abre en una nueva pestaña del navegador la página principal de la extensión.
					
					\item \textbf{\textit{El usuario}} llena incorrectamente al menos un campo del formulario.
					
					\item \textbf{\textit{EL usuario}} da click en el botón 'Iniciar sesión'.
					
					\item \textbf{\textit{La extensión}} despliega un popup en donde informa cuál campo está llenado de forma incorrecta.
					
					\item \textbf{\textit{El usuario}} da click en el botón 'Aceptar'.
					
					\item \textbf{\textit{La extensión}} retorna al formulario sin realizar ninguna otra acción.
					
				\end{enumerate}
				\paragraph{... Fin de la Trayectoria Alternativa 2 ...}
			    
			    
			    \paragraph{... Trayectoria Alternativa 3 ...}
				\begin{enumerate}
				
					\item \textbf{\textit{El usuario}} da click en el ícono de la extensión.
					
					\item \textbf{\textit{EL usuario}} da click en el botón 'Iniciar sesión'.
					
					\item \textbf{\textit{La extensión}} abre en una nueva pestaña del navegador la página principal de la extensión.
					
					\item \textbf{\textit{El usuario}} da click en el botón 'Iniciar sesión'.
					
					\item \textbf{\textit{La extensión}} abre la página de inicio de sesión.
					
					\item \textbf{\textit{El usuario}} llena incorrectamente al menos un campo del formulario.
					
					\item \textbf{\textit{EL usuario}} da click en el botón 'Iniciar sesión'.
					
				    \item \textbf{\textit{La extensión}} despliega un popup en donde informa cuál campo está llenado de forma incorrecta.
					
					\item \textbf{\textit{El usuario}} da click en el botón 'Aceptar'.
					
					\item \textbf{\textit{La extensión}} retorna al formulario sin realizar ninguna otra acción.
					
				\end{enumerate}
				\paragraph{... Fin de la Trayectoria Alternativa 3 ...}
				\newpage
			    
			    
				%DESCRIPCIÓN CI_CU5
				\begin{table}[H]
    				\begin{tabular}{ |p{3.5cm}||p{9.5cm}|}
    					\hline
    					\rowcolor{guindapoli}
    					\multicolumn{2}{|c|}{\textbf{\textcolor{white}{Caso de uso: CI\_CU5. Cerrar sesión en la extensión.}}}\\
    					\hline
    					\rowcolor{azulfuerte}Concepto & Descripción\\
    					\hline
    					\cellcolor{azulclaro}Actor & 
    					Usuario\\ 
    					\hline
    					\cellcolor{azulclaro}Propósito &
    					Este caso de uso permite al usuario cerrar su sesión en el ordenador que se encuentre en ese momento, siempre y cuando haya iniciado sesión anteriormente. El objetivo es que se elimine su certificado de la extensión.\\
    					\hline
    					\cellcolor{azulclaro}Entradas &
    					-\\
    					\hline
    					\cellcolor{azulclaro}Salidas &
    					Sesión cerrada, en este momento la extensión se encuentra sin usuario logueado.\\
    					\hline
    					\cellcolor{azulclaro}Pre-condiciones&
    					\textbf{CI\_CU4}.\\
    					\hline
    					\cellcolor{azulclaro}Post-condiciones&
    					La extensión debe mantenerse habilitada para su correcto funcionamiento.\\
    					\hline
    					\cellcolor{azulclaro}Reglas del negocio&
    					\hyperref[CI_RN1]{\textbf{CI\_RN1.}} \newline \hyperref[CI_RN4]{\textbf{CI\_RN4.}} \newline
    					\hyperref[CI_RN5]{\textbf{CI\_RN5.}}\\
    					\hline
    					\cellcolor{azulclaro}Errores &
    					No se puede cerrar sesión.
    					\\					
    					\hline
    				\end{tabular}
				\caption[DCU: CI\_CU5]{Descripción CU: CI\_CU5}
				\end{table}
				
				\paragraph{... Trayectoria Principal ...}
				\begin{enumerate}
				
					\item \textbf{\textit{El usuario}} da click en el ícono de la extensión.
					
					\item \textbf{\textit{El usuario}} da click en el botón 'Cerrar sesión'.
					
					\item \textbf{\textit{La extensión}} cierra la sesión del usuario.
					
				\end{enumerate}
				\paragraph{... Fin de la Trayectoria Principal ...}
				
				\paragraph{... Trayectoria Alternativa 1 ...}
				\begin{enumerate}
				
					\item \textbf{\textit{El usuario}} da click en el ícono de la extensión.
					
					\item \textbf{\textit{El usuario}} da click en el botón 'Cerrar sesión'.
					
					\item \textbf{\textit{La extensión}} no puede cerrar la sesión y lo informa al usuario por medio de un popup.
					
				\end{enumerate}
				\paragraph{... Fin de la Trayectoria Alternativa 1 ...}
				\newpage
				
				
				
				% DESCRIPCIÓN CI_CU6
				\begin{table}[H]
    				\begin{tabular}{ |p{3.5cm}||p{9.5cm}|}
    					\hline
    					\rowcolor{guindapoli}
    					\multicolumn{2}{|c|}{\textbf{\textcolor{white}{Caso de uso: CI\_CU6. Registrarse en servidor autentificador.}}}\\
    					\hline
    					\rowcolor{azulfuerte}Concepto & Descripción\\
    					\hline
    					\cellcolor{azulclaro}Actor & 
    					 Usuario \\ 
    					\hline
    					\cellcolor{azulclaro}Propósito &
    					Este caso de uso permitirá al usuario poder registrarse en el servidor autentificador si es que no tiene una cuenta con la cual poder iniciar sesión.\\
    					\hline
    					\cellcolor{azulclaro}Entradas &
    					email y contraseña ingresados por el usuario en la extensión.\\
    					\hline
    					\cellcolor{azulclaro}Salidas &
					    Estatus de operación.\\
    					\hline
    					\cellcolor{azulclaro}Pre-condiciones&
    					-\\
    					\hline
    					\cellcolor{azulclaro}Post-condiciones&
    					 -\\
    					\hline
    					\cellcolor{azulclaro}Reglas del negocio&
    					\hyperref[CI_RN5]{\textbf{CI\_RN5.}} \newline
    				    \hyperref[CI_RN7]{\textbf{CI\_RN7.}} \newline
    				    \hyperref[CI_RN8]{\textbf{CI\_RN8.}} \newline
    				    \hyperref[CI_RN9]{\textbf{CI\_RN9.}} \newline
    				    \hyperref[CI_RN10]{\textbf{CI\_RN10.}} \newline
    				    \hyperref[CI_RN11]{\textbf{CI\_RN11.}} \newline
    				    \hyperref[CI_RN12]{\textbf{CI\_RN12.}} \newline
    				    \hyperref[CI_RN13]{\textbf{CI\_RN13.}} \\
    					\hline
    					\cellcolor{azulclaro}Errores &
    					Error en el nombre de usuario. \newline
    					Error en el email.\newline
    					Error en la contraseña.\newline
    					No se pudo registrar al usuario.\\
    					\hline
    				\end{tabular}
				\caption[DCU: CI\_CU6]{Descripción CU: CI\_CU6}
				\end{table}
				\label{CI_CU6}
				
				\paragraph{... Trayectoria Principal ...}
				\begin{enumerate}
				    
				    \item \textbf{\textit{El usuario}} da click en el ícono de la extensión.
				    
				    \item \textbf{\textit{La extensión}} despliega un popup.
					
					\item \textbf{\textit{El usuario}} da click en el botón 'Iniciar Sesión'.
					
					\item \textbf{\textit{La extensión}} abre una página en otra pestaña del navegador.
					
					\item \textbf{\textit{El usuario}} da click en el botón 'Registrarse'.
					
					\item \textbf{\textit{La extensión}} muestra la página para registrarse.
					
					\item \textbf{\textit{El usuario}} llena los datos del formulario correctamente, los cuales son: 'Email', 'Contraseña' y 'Repetir contraseña'.
					
					\item \textbf{\textit{El usuario}} da click en el botón 'Crear Usuario'.
					
					\item \textbf{\textit{La extensión}} despliega un mensaje de confirmación.
					
					\item \textbf{\textit{El usuario}} da click en el botón '!Si, continuar!'.
					
			        \item \textbf{\textit{La extensión}} despliega el estado de la operación.
			        
			        \item \textbf{\textit{El usuario}} da click en el botón 'Iniciar sesión'.
			        
			        \item \textbf{\textit{La extensión}} redirige a la página de inicio de sesión.
					
				\end{enumerate}
				\paragraph{... Fin de la Trayectoria Principal ...}
				
				
				\paragraph{... Trayectoria Alternativa 1 ...}
				\begin{enumerate}
				    
				    \item \textbf{\textit{El usuario}} da click en el ícono de la extensión.
				    
				    \item \textbf{\textit{La extensión}} despliega un popup.
					
					\item \textbf{\textit{El usuario}} da click en el botón 'Iniciar Sesión'.
					
					\item \textbf{\textit{La extensión}} abre una página en otra pestaña del navegador.
					
					\item \textbf{\textit{El usuario}} da click en el botón 'Registrarse'.
					
					\item \textbf{\textit{La extensión}} muestra la página para registrarse.
					
					\item \textbf{\textit{El usuario}} llena al menos un campo del formulario incorrectamente.
					
					\item \textbf{\textit{El usuario}} da click en el botón 'Crear Usuario'.
					
					\item \textbf{\textit{La extensión}} despliega un popup en donde informa cuál campo está llenado de forma incorrecta.
					
					\item \textbf{\textit{El usuario}} da click en el botón 'Aceptar'.
					
					\item \textbf{\textit{La extensión}} retorna al formulario sin realizar ninguna otra acción.
				\end{enumerate}
				\paragraph{... Fin de la Trayectoria Alternativa 1 ...}
				
				
				\paragraph{... Trayectoria Alternativa 2 ...}
				\begin{enumerate}
				    
				    \item \textbf{\textit{El usuario}} da click en el ícono de la extensión.
				    
				    \item \textbf{\textit{La extensión}} despliega un popup.
					
					\item \textbf{\textit{El usuario}} da click en el botón 'Iniciar Sesión'.
					
					\item \textbf{\textit{La extensión}} abre una página en otra pestaña del navegador.
					
					\item \textbf{\textit{El usuario}} da click en el botón 'Registrarse'.
					
					\item \textbf{\textit{La extensión}} muestra la página para registrarse.
					
					\item \textbf{\textit{El usuario}} llena los datos del formulario correctamente, los cuales son: 'Nombre de usuario', 'Email', 'Contraseña' y 'Repetir contraseña'.
					
					\item \textbf{\textit{El usuario}} da click en el botón 'Crear Usuario'.
					
					\item \textbf{\textit{La extensión}} despliega un popup de confirmación.
					
					\item \textbf{\textit{El usuario}} da click en el botón 'Cancelar'.
					
			        \item \textbf{\textit{La extensión}} retorna al formulario sin realizar ninguna otra acción.
					
				\end{enumerate}
				\paragraph{... Fin de la Trayectoria Alternativa 2 ...}
				\newpage

			
			    % DESCRIPCIÓN CI_CU7
				\begin{table}[H]
    				\begin{tabular}{ |p{3.5cm}||p{9.5cm}|}
    					\hline
    					\rowcolor{guindapoli}  					\multicolumn{2}{|c|}{\textbf{\textcolor{white}{Caso de uso: CI\_CU7. Revocar certificado.}}}\\
    					\hline
    					\rowcolor{azulfuerte}Concepto & Descripción\\
    					\hline
    					\cellcolor{azulclaro}Actor & 
    					Usuario\\ 
    					\hline
    					\cellcolor{azulclaro}Propósito &
    					El usuario tendrá la opción de actualizar su certificado generando uno nuevo, con la finalidad de que el usuario tenga un mejor control de sus credenciales.\\
    					\hline
    					\cellcolor{azulclaro}Entradas &
    					Certificado autentificador.\\
    					\hline
    					\cellcolor{azulclaro}Salidas &
    					Certificado autentificador.\\
    					\hline
    					\cellcolor{azulclaro}Pre-condiciones&
    					\textbf{{CI\_CU4}}, \textbf{CI\_CU6}. \\
    					\hline
    					\cellcolor{azulclaro}Post-condiciones&
    					\textbf{CI\_CU4}\\
    					\hline
    					\cellcolor{azulclaro}Reglas del negocio&
    			        \hyperref[CI_RN4]{\textbf{CI\_RN4.}}\newline
    			        \hyperref[CI_RN5]{\textbf{CI\_RN5.}}\newline	
    				    \hyperref[CI_RN6]{\textbf{CI\_RN6.}}\\
    				    \hline
    				    
    					\cellcolor{azulclaro}Errores &
    				    No se puede almacenar el certificado autentificador.\\
    					\hline
    				\end{tabular}
				\caption[DCU: CI\_CU7]{Descripción CU: CI\_CU7}
				\end{table}
				
				\paragraph{... Trayectoria Principal ...}
				\begin{enumerate}
				    \item \textbf{\textit{La extensión}} ha creado el certificado autentificador.
				    \item \textbf{\textit{La extensión}} actualiza las credenciales de su nuevo certificado.
				     %\ref{fig:UI_certSavedInStorage}
				\end{enumerate}
				\paragraph{... Fin de la Trayectoria Principal ...}
				
				\paragraph{... Trayectoria Alternativa 1 ...}
				\begin{enumerate}
				    \item \textbf{\textit{La extensión}} no ha creado el certificado autentificador.
					\item \textbf{\textit{La extensión}} no guarda el certificado autentificador en storage. 
					 \ref{fig:UI_certNotSavedInStorage}
				\end{enumerate}
				\paragraph{... Fin de la Trayectoria Alternativa 1 ...}
				
				\paragraph{... Trayectoria Alternativa 2 ...}
				\begin{enumerate}
				    \item \textbf{\textit{La extensión}} ha creado  el certificado autentificador.
					\item \textbf{\textit{La extensión}} no puede guardar el certificado autentificador en storage. 
					\item \textbf{\textit{La extensión}} muestra al usuario el siguiente mensaje ''No se pudo guardar el certificado en el Storage'' en la figura \ref{fig:UI_certNotSavedInStorage}
				\end{enumerate}
				\paragraph{... Fin de la Trayectoria Alternativa 2 ...}
				
				\paragraph{... Trayectoria Alternativa 3 ...}
				\begin{enumerate}
				    \item \textbf{\textit{La extensión}} ha creado  el certificado autentificador.
					\item \textbf{\textit{La extensión}} no puede eliminar el certificado anterior.
					\item \textbf{\textit{La extensión}} muestra al usuario el siguiente mensaje ''No se pudo guardar el certificado en el Storage'' en la figura
					 \ref{fig:UI_certNotSavedInStorage}
				\end{enumerate}
				\paragraph{... Fin de la Trayectoria Alternativa 2 ...}
				
				\newpage
		
			\subsection{Diagrama de flujo.}
			    
			    \begin{figure}[H]
					\begin{center}			    	\includegraphics[height=16cm]{./imagenes/Disenio/Componente_1/CI_DF.png}
						\caption{Diagrama de flujo del Componente I.}
					\end{center}
				\end{figure}
				
			    \subsubsection{Descripción diagrama de flujo.}
			        Para el caso de este diagrama se inicia con una petición realizada por el navegador web para luego analizar si la extensión se encuentra activada, en caso de que no no se realiza ninguna acción, pero si sí se encuentra, se analizará si ya se inició sesión, en caso de que no ya no se realiza ninguna acción pero en caso afirmativo se obtiene el certificado guardado en storage para luego interceptar el patrón y realizar el proceso de chaffing para finalmente liberar la petición modificada.
				
			\subsection{Diagrama de flujo de datos.}
			    \begin{figure}[H]
					\begin{center}			        	    \includegraphics[height=16cm]{./imagenes/Disenio/Componente_1/CI_DFD.png}
						\caption{Diagrama de flujo de datos del Componente I.}
					\end{center}
				\end{figure}
			    
			    \subsubsection{Descripción diagrama de flujo de datos.}
			        En este caso contamos con dos entidades externas, el usuario por una parte debe iniciar sesión para que se pueda generar un \textbf{Certificado} el cual se utilizará para generar el chaffing y por otro lado el Navegador web que intercepta una \textbf{petición HTTP} que de igual manera se utilizará para generar el chaffing. Como resultado de la mezcla de estos dos se genera una \textbf{petición modificada} la cuál mas tarde se liberará para que pueda viajar hacia el servidor.
			   
			   
			   
			\subsection{Diagrama de clases.}
			
    			\begin{figure}[H]
    				\begin{center}	\includegraphics[width=14cm]{./imagenes/Disenio/Componente_1/CI_CD.png}
    				\caption{Diagrama de clases de Componente I.}
    				\end{center}
    			\end{figure}
			    
			    \subsubsection{Descripción de diagrama de clases}
			    
			       	\textbf{\textcolor{guindapoli}{Clase: \textit{Extension}}}\\
                    
                    \textbf{Atributos}
                    \begin{enumerate}
    		            \item \textbf{request} : Variable que almacena la petición HTTP.
        		        \begin{itemize}
        		            \item Tipo de dato: \textbf{Request}.
        		        \end{itemize}
        		        \item \textbf{status} : Variable para saber si la extensión está activada o no.
        		        \begin{itemize}
        		            \item Tipo de dato: \textbf{boolean}.
        		        \end{itemize}
        		        \item \textbf{chaffing} : instancia para ejecutar el proceso de Chaffing.
        		        \begin{itemize}
        		            \item Tipo de dato: \textbf{ChaffingUtil}.
        		        \end{itemize}
        		        \item \textbf{cert} : Variable para almacenar el certificado del usuario.
        		        \begin{itemize}
        		            \item Tipo de dato: \textbf{Certificate}.
        		        \end{itemize}
                    \end{enumerate}
                    
		            \textbf{Métodos}
    		        \begin{enumerate}
    		            \item \textbf{catchRequest()}: Este método permite interceptar las peticiones del usuario.
    		            \begin{itemize}
    		                \item Tipo de dato de retorno: \textbf{void}
    		            \end{itemize}
    		            \item \textbf{login(String user, String pass)}: Este método permite iniciar sesión para obtener el certificado del usuario.
    		            \begin{itemize}
    		                \item Tipo de dato de retorno: \textbf{int}
    		            \end{itemize}
    		            \item \textbf{logout(String pass, String email)}: Este método permite cerrar sesión en la extensión y eliminar el certificado almacenado.
    		            \begin{itemize}
    		                \item Tipo de dato de retorno: \textbf{void}
    		            \end{itemize}
    		            \item \textbf{revokeCert(String user, String pass)}: Este método permite revocar el certificado del usuario.
    		            \begin{itemize}
    		                \item Tipo de dato de retorno: \textbf{void}
    		            \end{itemize}
    		            \item \textbf{setFreeRequest(Request req)}: Este método permite liberar la petición nueva con chaffing.
    		            \begin{itemize}
    		                \item Tipo de dato de retorno: \textbf{void}
    		            \end{itemize}
    		        \end{enumerate}
    		        
    		        
    		        \textbf{\textcolor{guindapoli}{Clase: \textit{ChaffingUtil}}}\\
                  
		            \textbf{Métodos}
    		        \begin{enumerate}
    		            \item \textbf{doChaffing(Certificate certificate)}: Este método permite realizar la etapa de chaffing.
    		            \begin{itemize}
    		                \item Tipo de dato de retorno: \textbf{String}
    		            \end{itemize}
    		        \end{enumerate}
    		        
    		        
    		        \textbf{\textcolor{guindapoli}{Clase: \textit{HTTPHeader}}}\\
                  
		            \textbf{Atributos}
    		        \begin{enumerate}
    		            \item \textbf{name}: variable para guardar el nombre del header el protocolo.
    		            \begin{itemize}
    		                \item Tipo de dato de retorno: \textbf{String}
    		            \end{itemize}
    		             \item \textbf{value}: variable para guardar el contenido del header el protocolo.
    		            \begin{itemize}
    		                \item Tipo de dato de retorno: \textbf{String}
    		            \end{itemize}
    		        \end{enumerate}
    		        
    		        
    		        \textbf{\textcolor{guindapoli}{Clase: \textit{HTTPRequest}}}\\
                  
		            \textbf{Atributos}
    		        \begin{enumerate}
    		            \item \textbf{headers}: variable para guardar los header de una petición HTTP.
    		            \begin{itemize}
    		                \item Tipo de dato de retorno: \textbf{HTTPHeader}
    		            \end{itemize}
    		        \end{enumerate}
    		        
    		        
    		        \textbf{\textcolor{guindapoli}{Clase: \textit{Certificate}}}\\
                  
		            \textbf{Atributos}
    		        \begin{enumerate}
    		            \item \textbf{cert}: variable para guardar el certificado del usuario.
    		            \begin{itemize}
    		                \item Tipo de dato de retorno: \textbf{ByteArray}
    		            \end{itemize}
    		        \end{enumerate}
    		        
    		        \textbf{Métodos}
    		        \begin{enumerate}
    		            \item \textbf{saveCertificate()}: variable para guardar el certificado del usuario en el Storage.
    		            \begin{itemize}
    		                \item Tipo de dato de retorno: \textbf{ByteArray}
    		            \end{itemize}
    		        \end{enumerate}
			   
			   
			   
			   

            \subsection{Diagramas de secuencia.}       
          	
              	\subsubsection{Diagrama de secuencia: Realizar petición.}
                	\begin{figure}[H]
        			    \begin{center} \includegraphics[width=14cm]{./imagenes/Disenio/Componente_1/CI_SD_UC1.png}
        			    \caption[Diagrama de secuencia 1 del Componente I]{Diagrama de secuencia del CI\_CU1. Realizar petición.}
        		        \end{center}
        		    \end{figure}
        		    
        		    \paragraph{Descripción:}
        		    Para realizar una petición, es necesario seguir la siguiente secuencia. Primeramente, el usuario ingresa un URL al navegador, para que este construya la petición que mandará. Una vez que ha construido la petición realiza la petición, es decir sale a red la petición.
    		    
    		    
    		     \subsubsection{Diagrama de secuencia: Habilitar extensión.}
    			    \begin{figure}[H]
    				    \begin{center} \includegraphics[width=14cm]{./imagenes/Disenio/Componente_1/CI_SD_UC2.png}
    				    \caption[Diagrama de secuencia 2 del Componente I]{Diagrama de secuencia del CI\_CU2. Habilitar extensión.}
    			        \end{center}
    			    \end{figure}
    			    
    			    \paragraph{Descripción:}
    			    Para habilitar la extensión, el usuario necesita primero abrir la extensión. Una vez abierta, debe de dar click en el botón 'Activar' para que la extensión ejecute la orden y se mantenga habilitada.
    			    
    			
    			\subsubsection{Diagrama de secuencia: Deshabilitar extensión.}
    			    \begin{figure}[H]
    				    \begin{center} \includegraphics[width=14cm]{./imagenes/Disenio/Componente_1/CI_SD_UC3.png}
    				    \caption[Diagrama de secuencia 3 del Componente I]{Diagrama de secuencia del CI\_CU3. Deshabilitar extensión.}
    			        \end{center}
    			    \end{figure}
    			    
    			    \paragraph{Descripción:}
    			    Para deshabilitar la extensión, el usuario necesita primero abrir la extensión. Una vez abierta la extensión, debe de dar click en el botón 'Desactivar', para que la extensión ejecute la orden y se mantenga deshabilitada.
    			   
    			
    			\subsubsection{Diagrama de secuencia: Inicio de sesión}
    			    \begin{figure}[H]
    				    \begin{center} \includegraphics[width=14cm]{./imagenes/Disenio/Componente_1/CI_SD_UC4.png}
    				    \caption[Diagrama de secuencia 4 del Componente I]{Diagrama de secuencia del CI\_CU4. Inicio de sesión en la extensión.}
    			        \end{center}
    			    \end{figure}
    			    
    			    \paragraph{Descripción:}
        		    Como primeros dos pasos de la secuencia para iniciar sesión en la extensión, el usuario ingresa su nombre de usuario y su contraseña, para después iniciar sesión en la extensión. Una vez que el usuario ha dado la orden de iniciar sesión, la extensión valida el nombre de usuario y la contraseña, retornando un mensaje en caso de que alguno de estos dos valores esté mal, si no es el caso, la extensión retornará un mensaje con el estatus del inicio de sesión. 
    			
    			
    			\subsubsection{Diagrama de secuencia: Cerrar sesión.}
        			\begin{figure}[H]
        				\begin{center}    		    	\includegraphics[width=14cm]{./imagenes/Disenio/Componente_1/CI_SD_UC5.png}
        				\caption[Diagrama de secuencia 5 del Componente I]{Diagrama de secuencia del CI\_CU5. Cerrar sesión en la extensión.}
        				\end{center}
        			\end{figure}
        			
        			\paragraph{Descripción:}
        		        Como primer paso de este diagrama de secuencia se tiene que abrir la extensión para después dar click en el botón de 'Cerrar Sesión'.
        		
        		
        		\subsubsection{Diagrama de secuencia: Registrarse en servidor autentificador.}
        			\begin{figure}[H]
        				\begin{center}    		    	\includegraphics[height=14cm]{./imagenes/Disenio/Componente_1/CI_SD_UC6.png}
        				\caption[Diagrama de secuencia 6 del Componente I]{Diagrama de secuencia del CI\_CU6. Registrarse en servidor autentificador.}
        				\end{center}
        			\end{figure}
        			
        			\paragraph{Descripción:}	
    			         Como primeros pasos de este diagrama de secuencia, la extensión se tiene que abrir para después dar click en el botón de 'Iniciar sesión' para desplegar la página principal y, una vez ahí, dar click en el botón de 'Registrarse'. Una vez en la página de registro, el usuario, como siguiente paso, llena los datos requeridos por el formulario para después dar click en el botón 'Crear Usuario'. La extensión validará los campos retornando un mensaje de error en caso de que alguno se llene mal, si no es el caso, retornará un aviso con el estatus del registro.
    			
    			\subsubsection{Diagrama de secuencia: Revocar certificado.}
        			\begin{figure}[H]
        				\begin{center}    		    	\includegraphics[width=14cm]{./imagenes/Disenio/Componente_1/CI_SD_UC7.png}
        				\caption[Diagrama de secuencia 7 del Componente I]{Diagrama de secuencia del CI\_CU7. Revocar certificado.}
        				\end{center}
        			\end{figure}
        			
        			\paragraph{Descripción:}
        		
    		
		    \subsection{Diagrama de actividades}
    		    \begin{figure}[H]
    				\begin{center}	\includegraphics[width=13cm]{./imagenes/Disenio/Componente_1/CI_DA.png}
    					\caption{Diagrama de actividades del Prototipo 2.}
    				\end{center}
    			\end{figure}
    		
        		\subsubsection{Descripción del diagrama de actividades.}
        		El componente cuenta con los siguientes pasos.
            	Uno de los primeros pasos que debe hacer el usuario es iniciar sesión, esto con el fin de generar un certificado de acuerdo a los datos del usuario. Dicho certificado se almacena en el \textbf{Storage} de Google Chrome. \\
            	El otro camino a seguir por el usuario es realizar una búsqueda en el navegador, en la cual el navegador realiza la petición y la extensión recibe la misma. Modificándola siempre y cuando se encuentre un certificado guardado en el Storage. Si este proceso es válido, se genera el proceso de Chaffing y se libera la petición.
			
    		\subsection{Interfaz de usuario.}
                
                \subsubsection{Pantalla Inicial.}
                Esta pantalla es la primer vista que el usuario tiene el sistema. Aparece al darle click a la extensión en su icono correspondiente. 
                
                \begin{figure}[H]
    				\begin{center}	\includegraphics[width=9cm]{./imagenes/Disenio/Componente_1/UI_extension.PNG}
    				\caption{Pantalla de interfaz de la extensión.}
    				\label{fig:CI_PantallaExtension}
    				\end{center}
    			\end{figure}
                
                En ella, el usuario puede activar y desactivar la extensión para empezar a interceptar peticiones, ademas cuenta con un botón de inicio de sesión \textit{(Iniciar Sesión)}, el cual al darle click abrirá la pantalla 'Tab de la extensión' (Figura \ref{fig:CI_PantallaInicial}). La cual se muestra a continuación.
                
    		    \begin{figure}[H]
    				\begin{center}	\includegraphics[width=15cm]{./imagenes/Disenio/Componente_1/UI_inicio.PNG}
    				\caption{Pantalla inicial.}
    		        \label{fig:CI_PantallaInicial}
    				\end{center}
    			\end{figure}
    			
    			El navegador Google Chrome tiene ciertas restricciones con los certificados que no son firmados por una autoridad certificadora de confianza, y debido a que nuestros certificados son autofirmados. nos aparecerá un mensaje de alerta por parte del navegador, para evitar que esto sea un problema, en la extensión vamos a seleccionar la opción de \textit{Aviso}, donde nos abrirá una pestaña nueva de un aviso de seguridad, vamos a escoger la opción de \textit{opciones avanzadas}, y se desplegará información adicional con la opción de \textit{Proceder a la ip x.x.x.x}, como se muestra en la siguiente imagen: 
    			
    			\begin{figure}[H]
    				\begin{center}	\includegraphics[width=12cm]{./imagenes/Disenio/Componente_1/Ext_aviso.png}
    				\caption{Pantalla de aviso.}
    		        \label{fig:Ext_aviso}
    				\end{center}
    			\end{figure}
    			
    			\subsubsection{Tab de la extensión.}
    			Esta pantalla es la interfaz que se le brinda al usuario para que inicie sesión y obtenga su certificado autentificador (Figura \ref{fig:CI_Tab}). En ella se aprecian únicamente dos campos para introducir texto ('Ingrese email' e 'Ingrese contraseña'), y un botón 'Iniciar Sesión'.
    			  
    		    \begin{figure}[H]
    				\begin{center}	\includegraphics[width=12cm]{./imagenes/Disenio/Componente_1/UI_webpage.PNG}
    					\caption{Tab de la extensión. Permite inicio de sesión}
    					\label{fig:CI_Tab}
    				\end{center}
    			\end{figure}
    			
    			Durante el inicio de sesión el usuario puede visualizar mensajes de éxito o error. La Figura \ref{fig:UI_certSavedInStorage} se le muestra al usuario tras haber obtenido el certificado desde la autoridad y guardado con éxito en el Storage de Google Chrome. La Figura \ref{fig:UI_certNotSavedInStorage} notifica al usuario que existió un error al obtener el certificado de la autoridad certificadora. 
    			
    			\begin{figure}[H]
    				\begin{center}	\includegraphics[width=10cm]{./imagenes/Disenio/Componente_1/UI_certSavedInStorage.PNG}
    					\caption[Mensaje de éxito]{Mensaje de éxito tras obtener el certificado de la autoridad y guardarlo en el storage de Google Chrome.}
    				\label{fig:UI_certSavedInStorage}
    				\end{center}
    			\end{figure}
    			
    			\begin{figure}[H]
    				\begin{center}	\includegraphics[width=10cm]{./imagenes/Disenio/Componente_1/UI_certNotSavedInStorage.PNG}
    					\caption[Mensaje de error]{Mensaje de error al tratar de obtener certificado de la autoridad (Usuario y/o contraseña incorrectos).}
    					\label{fig:UI_certNotSavedInStorage}
    				\end{center}
    			\end{figure}
    			
    			
    			Por otra parte, se le brinda al usuario una interfaz donde pueda registrarse, para ello basta con escoger la opción de \textit{Registrarse} en la parte superior de la interfaz, al acceder a esta interfaz, aparecerán cuatro campos los cuales el usuario puede llenar para crear una cuenta y registrarse en la extensión. Estos campos son correo electrónico y contraseña.
    			
    			\begin{figure}[H]
    				\begin{center}	\includegraphics[width=10cm]{./imagenes/Disenio/Componente_1/UI_registro1.PNG}
    					\caption{Tab de la extensión. Permite registrarse}
    					\label{fig:CI_TabRegistro}
    				\end{center}
    			\end{figure}
    			
    			Al llenar los campos y enviar la petición a la autoridad certificadora de registrase, se le notificará al usuario si pudo generarse el registro del usuario con éxito u ocurrió un error tras crear el registro. Cabe resaltar que la autoridad creará un certificado para dicho usuario, el cual el usuario deberá obtener posteriormente con un inicio de sesión, éste proceso se explicará en el componente 2.\\
    			
    			Como mensaje de éxito se muestra la figura \ref{fig:UI_nuevoUsuarioRegistrado} la cual significa que el usuario fue creado con éxito y se le generó un certificado para dicho usuario.
    			
    			\begin{figure}[H]
    				\begin{center}	\includegraphics[width=10cm]{./imagenes/Disenio/Componente_1/UI_nuevoUsuarioRegistrado.PNG}
    					\caption[Mensaje de éxito]{Mensaje de éxito tras obtener el certificado de la autoridad y guardarlo en el storage de Google Chrome.}
    				\label{fig:UI_nuevoUsuarioRegistrado}
    				\end{center}
    			\end{figure}
    			
    			Como mensajes de errores se muestran distintos tipos de mensajes para diferentes situaciones, las cuales se explican a continuación.
    			
    			La figura \ref{fig:UI_emailIncorrecto} notifica al usuario que el email introducido no es valido, es decir, no cuenta con una estructura valida de un correo electrónico (email).
    			
    			\begin{figure}[H]
    				\begin{center}	\includegraphics[width=10cm]{./imagenes/Disenio/Componente_1/UI_emailIncorrecto.PNG}
    					\caption[Mensaje de error]{Mensaje de error tras detectar que no se cumplen los requisitos para el correo ingresado.}
    				\label{fig:UI_emailIncorrecto}
    				\end{center}
    			\end{figure}
    			
    			La figura \ref{fig:UI_contraseniaIncorrecto} muestra un error tras no cumplir los requisitos con la contraseña introducida.
    			
    			\begin{figure}[H]
    				\begin{center}	\includegraphics[width=10cm]{./imagenes/Disenio/Componente_1/UI_contraseniaIncorrecto.PNG}
    					\caption[Mensaje de error]{Mensaje de error tras detectar que no se cumplen los requisitos para la contraseña ingresada.}
    				\label{fig:UI_contraseniaIncorrecto}
    				\end{center}
    			\end{figure}
    			
    			La figura \ref{fig:UI_contraseniasNoIguales} muestra un error tras no coincidir las contraseñas introducidas.
    			
    			\begin{figure}[H]
    				\begin{center}	\includegraphics[width=10cm]{./imagenes/Disenio/Componente_1/UI_contraseniasNoIguales.PNG}
    					\caption[Mensaje de error]{Mensaje de error tras detectar que no coinciden el par de contraseñas ingresadas.}
    				\label{fig:UI_contraseniasNoIguales}
    				\end{center}
    			\end{figure}
    			
    			Y por ultimo se muestra en la figura \ref{fig:UI_usuarioNoRegistrado} el error que se obtiene al tratar de registrarse con un usuario que ya tiene una cuenta en la extensión.
    			
    			\begin{figure}[H]
    				\begin{center}	\includegraphics[width=10cm]{./imagenes/Disenio/Componente_1/UI_usuarioNoRegistrado.PNG}
    					\caption[Mensaje de error]{Mensaje de error tras detectar que el usuario ya se encuentra registrado.}
    				\label{fig:UI_usuarioNoRegistrado}
    				\end{center}
    			\end{figure}
    			
    			
			El usuario tendrá también la posibilidad de revocar su certificado. permitiéndole asi generar uno por la autoridad certificadora. La figura \ref{fig:UI_revocarCertificado} muestra la interfaz que se le proporciona al usuario para realizar la revocación de su certificado.
			
		    \begin{figure}[H]
				\begin{center}	\includegraphics[width=10cm]{./imagenes/Disenio/Componente_1/UI_revocarCertificado.PNG}
					\caption[Interfaz]{Interfaz para revocar certificado.}
				\label{fig:UI_revocarCertificado}
				\end{center}
			\end{figure}
			
			
			La figura \ref{fig:UI_revocacionExito} muestra un mensaje de éxito, el cual se le da a conocer al usuario que su certificado fue revocado exitosamente.
    			
    			\begin{figure}[H]
    				\begin{center}	\includegraphics[width=10cm]{./imagenes/Disenio/Componente_1/UI_revocacionExito.PNG}
    					\caption[Mensaje de éxito]{Mensaje de éxito al revocar el certificado de un usuario.}
    				\label{fig:UI_revocacionExito}
    				\end{center}
    			\end{figure}
			
			Si el usuario ingresa sus credenciales (correo electrónico y contraseña) incorrectas, la interfaz despliega la información de error de la figura \ref{fig:UI_revocacionFallo}.
    			
    			\begin{figure}[H]
    				\begin{center}	\includegraphics[width=10cm]{./imagenes/Disenio/Componente_1/UI_revocacionFallo.PNG}
    					\caption[Mensaje de error]{Mensaje de error al tratar de revocar el certificado de un usuario con credenciales incorrectas.}
    				\label{fig:UI_revocacionFallo}
    				\end{center}
    			\end{figure}
    			
			El usuario visualizará y tendrá interacción con este componente para obtener el certificado y llevar a cabo el proceso de Chaffing. Este proceso se lleva a cabo en segundo plano de la extensión (background) el cual cacha la petición e inyecta el resultado del Chaffing en el header de la petición hecha previamente. Una vez inyectado el código, la extensión libera la petición y ésta viaja en red al servidor de prueba.
			
			%Se ha analizado con un analizador de protocolos (Wireshark), para poder ver la información que viaja en red, y así poder ver el contenido del header de la petición. En la figura \ref{fig:wiresharkAnalizandoInterfaz} se muestra esta petición capturada con el analizador de protocolos Wireshark.
			
			%\begin{figure}[H]
        	%	\begin{center}	\includegraphics[width=11cm]{./imagenes/Desarrollo/Componente_1/Prototipo_1/wiresharkFull.jpeg}
        	%		\caption{Wireshark analizando en red WI-FI filtrando analisis por \textit{http}.}
        	%		\label{fig:wiresharkAnalizandoInterfaz}
        	%	\end{center}
        	%\end{figure}
	
        	%Para un análisis más específico, se muestra la petición en una herramienta que proporciona wireshark, \textit{Show Packet Bytes...}, la cual muestra el encabezado completo, teniendo una herramienta para seleccionar la codificación del texto (Figura \ref{fig:analisisPeticionInterfaz}). Para este caso se ha optado por elegir la codificación ISO8859-1, ya que en ésta nos permite ver de manera legible el encabezado.
	
        	%\begin{figure}[H]
        	%	\begin{center}	\includegraphics[width=11cm]{./imagenes/Desarrollo/Componente_1/Prototipo_1/wiresharkFrame_ISO_header.jpeg}
        	%		\caption{Análisis de petición en \textit{Show Packet Bytes...}}
        	%		\label{fig:analisisPeticionInterfaz}
        	%	\end{center}
        	%\end{figure}
    	   
    	    \subsection{Requisitos de diseño.}
    			   En este apartado, se especificarán los requisitos de diseño para que el prototipo opere de forma correcta, de igualmanera se tiene por entendido que son necesarios los Requisitos del diseño del primer prototipo.
    			   \subsubsection{Requisitos de ejecución de la extensi\'on}
    			        Para poder ejecutar el prototipo dos es necesario contar con todo lo requerido para el prototipo uno y agregarle la interacci\'on con \textbf{jQuery 3.3.1 } y \textbf{Bootstrap} que son dos frameworks que corren sobre JavaScript y que nos permiten interactuar un poco mejor con el usuario haciendo la interfaz de la extensi\'on más amigable y entendible, estos ya se encuentran insertados en la extensión por lo que no es necesario que el usuario realice acción alguna.
    			        Otros de los requisitos para su ejecución es contar con un \textbf{usuario y contraseña} válidos ya que ser\'an de suma importancia para el correcto funcionamiento del prototipo dos. Para deshabilitar la extensión, el usuario necesita primero abrir la extensión. Una vez abierta la extensión, debe de dar click en el botón ''deshabilitar'', para que la extensión ejecute la orden y se mantenga deshabilitada.

    			     \subsubsection{Requisitos para el envío del código autentificador}   
    			        Para este prototipo se considera necesaria los siguientes requisitos de diseño:
    			        \begin{itemize}
    			            \item \textbf{Código Autentificador} Archivo indispenzable en este prototito debido a que sin este es imposible acompletar el proceso.
                            
    			            \item \textbf{Iniciar Sesión} Este botón generará un certificado de prueba y se almacenará en el Storage del navegador Google Chrome. Esto con el fin, que al momento de interceptar la petición, la extensión de encargará de inyectar el certificado almacenado modificado con el método \textit{Chaffing} en el encabezado de protocolo HTTP.
    			           
    			        \end{itemize}
    			        \newpage
    			        
    			        
    		\subsection{Algoritmos.}
		        Los algoritmos necesarios para hacer el \textit{chaffing} son expuestos a continuación:\\
                \begin{algorithm}[H] % Algoritmo para obtener el patrón de Chaffing
                    \SetAlgoLined
                    \KwData{lenCert, lenAccept, Cert}
                    \KwResult{patternChaffing}
                    $lenPattern \longleftarrow lenCert + lenAccept$\;
                    %Indicar que el arreglo inicia en ceros
                    $Pattern[lenPattern*8] \longleftarrow 0$\;
                    %función get_ones devuelve el número de unos en bits que tiene el parámetro
                    $unosCert \longleftarrow getOnes(Cert)$\;
                    \For{$i=1$ \KwTo unosCert}{
                        \Repeat{$Pattern[random] != 0$}{
                            $random \longleftarrow secure\_random(0,lenPattern)$\;
                        }
                        $Pattern[random] \longleftarrow '1'$\;
                    }
                    \SetAlgorithmName{Algoritmo}{algoritmo}{Algoritmos}
                    \caption{getPattern: Generación de patrón de chaffing}
                \end{algorithm}
                
                \newpage
                
                \begin{algorithm}[H]
                    \SetAlgoLined
                    \KwData{Cert, Accept}
                    \KwResult{Chaffing}
                    $lenCert \longleftarrow lenght(Cert)$\;
                    $lenAccept \longleftarrow lenght(Accept)$\;
                    %Aquí se llama el algoritmo de arriba, para obtener el patrón.
                    $Pattern \longleftarrow getPattern(lenCert, lenAccept, Cert)$\;
                    $Chaffing \longleftarrow " "$\;
                    %Contadores para agarrar el siguiente byte del cert o Accept.
                    $countCert \longleftarrow 0$\;
                    $countAccept \longleftarrow 0$\;
                    %Para cada elemento de como resultó el patrón.
                    $certBits \longleftarrow getBits(Cert)$\;
                    $acceptBits \longleftarrow getBits(Accept)$\;
                    \ForEach{ $ i $ in Pattern}{
                        %Si es un 1, se coloca el siguiente byte del certificado.
                        \eIf{$i == '1'$}{
                        $Chaffing \longleftarrow Chaffing + certBits[countCert]$\;
                        $countCert \longleftarrow countCert + 1$\;
                        }{%Else, es un 0, entonces se coloca el siguiente byte del Accept.
                        $Chaffing \longleftarrow Chaffing + acceptBits[countAccept]$\;
                        $countAccept \longleftarrow countAccept + 1$\;
                        }
                    }
                    $Chaffing \longleftarrow getBytes(Chaffing)$\;
                    \SetAlgorithmName{Algoritmo}{algoritmo}{Algoritmos}
                    \caption{getChaff: Generación de chaffing}
                \end{algorithm}
                
    		   \paragraph{}
    		   Primero vamos a analizar el algoritmo del \textit{patrón de chaffing} con un pequeño ejemplo (utilizaremos datos de variables cortos); supongamos que nuestro certificado es el siguiente: 
    		    \begin{center}
    		        $C_k = MITZ057abZ251$
    		    \end{center}
    		    y utilizaremos el campo \textit{Accept} del header de la petición como \textit{chaff}, lo cual tendrá lo siguiente:
    		    \begin{center}
    		        $P_{HTTP} = Mozilla5.5/Chrome8.1/Safari$
    		    \end{center}
    		    Entonces, nuestro patrón de chaffing terminará teniendo un tamaño de la longitud de los caracteres de nuestro certificado más la longitud de los datos de la petición HTTP, esta variable se llamará \textit{lenPattern}. A continuación vamos a utilizar un arreglo de banderas de ese tamaño múltiplicado por 8, para que así podamos llenar de ceros y unos este arereglo y cada una de las posiciones nos represente en el chaffing un bit, este arreglo inicia con 0 todos sus valores, y que al final será nuestro patrón, esta variable será \textit{Pattern}.\\
    		    
    		    Este algoritmo nos permite generar posiciones aleatorias dentro del rango del tamaño de \textit{lenPattern}, y tantas veces como se tengan caracteres en el certiicado, se pondrá un 1 en dicha posición si es que hay un 0, en caso de que no se encuentre un 0 se repetirá el procedimiento obteniendo una posición aleatoria diferente, esto nos generará nuestro patrón para el Chaffing. Utilizaremos un contador para conseguir esto, aumentándolo cada vez que se obtenga un número aleatorio válido. En nuestro algoritmo, nuestra variable \textit{random} contiene la posición aleatoria donde se validará si en esa posición se puede poner un 1 o no.\\
    		    
        		
        		Este algoritmo nos permite llenar de unos el mismo número de veces que el número de unos contenga nuestro certificado en bits, lo que nos asegura que cada posición le corresponde a un byte ya sea del certificado o de la petición a enviar. Supongamos que al terminar este algoritmo, nuestro arreglo queda algo parecido a lo siguiente:
        		
        		\begin{figure}[H]
        			\begin{center}	                  \includegraphics[width=13cm]{./imagenes/Disenio/Componente_1/algorithm1_3.png}
    				\caption{Arreglo del patrón de chaffing final}
        			\end{center}
        		\end{figure}
    		    
    		    El segundo algoritmo realizará la etapa de Chaffing, utilizando el patrón generado en el algoritmo anterior, recorriendo cada posición de este arreglo, contaremos con dos contadores para saber que caracter se debe de poner a continuación en el proceso del algoritmo, \textit{countCert} y \textit{countAccept}, contador para el certificado y para el encabezado Accept respectivamente. \\
    		    
    		    Cuando se detecte un 1 en el arreglo del patrón, vamos a proceder a colcoar el byte siguiente correspondiente al arreglo del certificado. En caso análogo, si se encuentra con un 0, entonces se colocorá el siguiente caracter del encabezado Accept, se llevará el control de ambos mediante sus respectivos contadores.\\
    		    
    		    Por último, es importante mencionar que se mandará el Chaffing en base64 para evitar problemas con los caractéres no válidos que se puedan generar. Seguido del Chaffing en base64, vamos a enviar el patrón con un cifrado híbrido, cifrando primero el patrón con AES, la llave que utilizaremos para realizar este cifrado la cifraremos con RSA utilizando la llave pública del servidor. 

                \subsubsection{Complejidad computacional.}
                    Haciendo un analisis de los algoritmos anteriormente mostrados, deducimos que la complejidad del algoritmo de \textit{chaffing} es:
                    $O(n)$
                    Donde $n$ es la longitud de carácteres del certificado.

        \section{Componente II: Servidor Autentificador.}
			
    		\subsection{Diagrama de casos de uso.}
    		
    			\begin{figure}[H]
                	\begin{center}	\includegraphics[width=13cm]{./imagenes/Disenio/Componente_2/CII_UCD.png}
                	\caption{Diagrama de casos de uso del Componente II.}
                	\end{center}
        		\end{figure}
		    
		    \subsubsection{Descripción de casos de uso.}
		
		    \begin{table}[H]
    			\begin{tabular}{ |p{3.5cm}||p{9.5cm}|}
    				\hline
    				\rowcolor{guindapoli}
    				\multicolumn{2}{|c|}{\textbf{\textcolor{white}{Caso de uso: CII\_CU1. Crear nuevo usuario.}}}\\
    				\hline
    				\rowcolor{azulfuerte}Concepto & Descripción\\
    				\hline
    				\cellcolor{azulclaro}Actor & 
    				Componente I. Extensión.\\ 
    				\hline
    				\cellcolor{azulclaro}Propósito &
    				Guardar un nuevo usuario en el Componente I para que éste pueda hacer uso de este método de autentificación.\\
    				\hline
    				\cellcolor{azulclaro}Entradas &
    				Correo electrónico y contraseña.\\
    				\hline
    				\cellcolor{azulclaro}Salidas &
    				Status de operación.\\
    				\hline
    				\cellcolor{azulclaro}Pre-condiciones&
    				-\\
    				\hline
    				\cellcolor{azulclaro}Post-condiciones&
    				Creación de un certificado.\\
    				\hline
    				\cellcolor{azulclaro}Reglas del negocio&
    				\hyperref[CII_RN1]{\textbf{CII\_RN1}} \newline
    				\hyperref[CII_RN2]{\textbf{CII\_RN2}} \\
    				\hline
    				\cellcolor{azulclaro}Errores &
    				No se pudo registrar el usuario en la base de datos.\\
    				\hline
    		    \end{tabular}
    		    \caption[DCU: CII\_CU1]{Descripción CU: CII\_CU1}
		    \end{table}
		
    		\paragraph{... Trayectoria Principal ...}
    			\begin{enumerate}
    				\item \textbf{\textit{La extensión}} solicita al  servidor autentificador registrar un nuevo usuario enviando los datos necesarios.
    				
    				\item \textbf{\textit{El servidor autentificador}} recibe los parámetros que le envió la extensión.
    				
    				\item \textbf{\textit{El servidor autentificador }} guarda al usuario en la base de datos.
    				
    				\item \textbf{\textit{El servidor autentificador}} genera un certificado de acuerdo a los datos dados por el usuario.
    				
    				\item \textbf{\textit{El servidor autentificador}} comunica a la extensión que el usuario ha sido creado.
    				
    				\item \textbf{\textit{La extensión}} despliega un mensaje de confirmación al usuario. 
    			\end{enumerate}
    		\paragraph{... Fin de la Trayectoria Principal ...}
		
    		\paragraph{... Trayectoria alternativa 1 ...}
    		\begin{enumerate}
    		    \item \textbf{\textit{La extensión}} solicita al servidor autentificador a registrar un nuevo usuario enviando los datos necesarios.
    		    
    		    \item \textbf{\textit{El servidor autentificador}} no recibe correctamente los datos.
    		    
    		    \item \textbf{\textit{La extensión}} despliega un mensaje de error.
    		\end{enumerate}
    		\paragraph{... Fin de Trayectoria alternativa 1 ...}
		
    		\paragraph{... Trayectoria alternativa 2 ...}
    		\begin{enumerate}
    		    \item \textbf{\textit{La extensión}} solicita al servidor autentificador a registrar un nuevo usuario enviando los datos necesarios.
    		    
    		    \item \textbf{\textit{El servidor autentificador}} recibe los parámetros que le envió la extensión.
    		    
    		    \item \textbf{\textit{El serivdor autentificador}} no puede registrar el nuevo usuario en su base de datos.
    		    
    		    \item \textbf{\textit{La extensión}} despliega un mensaje de error.
    		\end{enumerate}
    		\paragraph{... Fin de Trayectoria alternativa 2 ...}
    		\newpage
		
		
		
    		\begin{table}[H]
    			\begin{tabular}{ |p{3.5cm}||p{9.5cm}|}
    				\hline
    				\rowcolor{guindapoli}
    				\multicolumn{2}{|c|}{\textbf{\textcolor{white}{Caso de uso: CII\_CU2. Revocar certificado.}}}\\
    				\hline
    				\rowcolor{azulfuerte}Concepto & Descripción\\
    				\hline
    				\cellcolor{azulclaro}Actor & 
    				Componente I. Extensión.\\ 
    				\hline
    				\cellcolor{azulclaro}Propósito &
    				Revocar el certificado que se encuentra vinculado a cierto usuario, cambiándolo por uno nuevo.\\
    				\hline
    				\cellcolor{azulclaro}Entradas &
    				Usuario y contraseña.\\
    				\hline
    				\cellcolor{azulclaro}Salidas &
    				Certificado nuevo.\\
    				\hline
    				\cellcolor{azulclaro}Pre-condiciones&
    				Usuario previamente registrado en el servidor autentificador.\\
    				\hline
    				\cellcolor{azulclaro}Post-condiciones&
    				-\\
    				\hline
    				\cellcolor{azulclaro}Reglas del negocio&
    				\hyperref[CII_RN1]{\textbf{CII\_RN1}} \newline
    				\hyperref[CII_RN2]{\textbf{CII\_RN2}} \newline
    				\hyperref[CII_RN4]{\textbf{CII\_RN4}} \newline
    				\hyperref[CII_RN6]{\textbf{CII\_RN6}} \\
    				\hline
    				\cellcolor{azulclaro}Errores &
    				El certificado no se pudo crear correctamente. \newline
    				No se pudo asignar el certificado al usuario.\\
    				\hline
    		    \end{tabular}
    		    \caption[DCU: CII\_CU2]{Descripción CU: CII\_CU2}
    		\end{table}
    		
    		\paragraph{... Trayectoria principal ...}
    		\begin{enumerate}
    		    \item \textbf{\textit{La extensión}} solicita revocar certificado al usuario actual enviando los datos necesarios.
    		    
    		    \item \textbf{\textit{El servidor autentificador}} valida si se encuentra registrado el usuario.
    		    
    		    \item \textbf{\textit{El servidor autentificador}} genera un nuevo certificado.
    		    
    		    \item \textbf{\textit{El servidor autentificador}} elimina el antiguo certificado asignado a ese usuario.
    		    
    		    \item \textbf{\textit{El servidor autentificador}} asigna el nuevo certificado al usuario.
    		    
    		    \item \textbf{\textit{El servidor autentificador}} envía un mensaje a la extensión de que se revocó correctamente el certificado.
    		\end{enumerate}
    		\paragraph{... Fin Trayectoria principal ...}
    		
    		\paragraph{... Trayectoria Alternativa 1 ...}
    		\begin{enumerate}
    		    \item \textbf{\textit{La extensión}} solicita revocar certificado al usuario actual.
    		    
    		    \item \textbf{\textit{El servidor autentificador}} no puede encontrar el usuario que solicitó revocar el certificado.
    		    
    		    \item \textbf{\textit{El servidor autentificador}} envía un código de error a la extensión.
    		    
    		    \item \textbf{\textit{La extensión}} despliega un mensaje de error.
    		\end{enumerate}
    		\paragraph{... Fin Trayectoria Alternativa 1 ...}
    		
    		\paragraph{... Trayectoria Alternativa 2 ...}
    		\begin{enumerate}
    		    \item \textbf{\textit{La extensión}} solicita revocar el certificado al usuario actual.
                
                \item \textbf{\textit{El servidor autentificador}} valida si se encuentra registrado el usuario.
                
    		    \item \textbf{\textit{El servidor autentificador}} no puede generar un nuevo certificado.
    		     
    		    \item \textbf{\textit{El servidor autentificador}} envía un código de error a la extensión.
    		    
    		    \item \textbf{\textit{La extensión}} despliega un mensaje de error.
    		\end{enumerate}
    		\paragraph{... Fin Trayectoria Alternativa 2 ...}
    		
    		\paragraph{... Trayectoria Alternativa 3 ...}
    		\begin{enumerate}
    		    \item \textbf{\textit{La extensión}} solicita revocar el certificado al usuario actual.
    		    
    		    \item \textbf{\textit{El servidor autentificador}} valida si se encuentra registrado el usuario.
    		    
    		    \item \textbf{\textit{El servidor autentificador}} genera un nuevo certificado.
    		    
    		    \item \textbf{\textit{El servidor autentificador}} elimina el antiguo certificado asignado a ese usuario.
    		    
    		    \item \textbf{\textit{El servidor autentificador}} asigna el certificado al usuario.
    		    
    		    \item \textbf{\textit{El servidor autentificador}} no puede enviarle el certificado al usuario.
    		    
    		    \item \textbf{\textit{El servidor autentificador}} envía un código de error a la extensión.
    		    
    		    \item \textbf{\textit{La extensión}} despliega un mensaje de error.
    		\end{enumerate}
    		\paragraph{... Fin Trayectorai Alternativa 3 ...}
    		\newpage
		
		
		
		
    		\begin{table}[H]
    			\begin{tabular}{ |p{3.5cm}||p{9.5cm}|}
    				\hline
    				\rowcolor{guindapoli}
    				\multicolumn{2}{|c|}{\textbf{\textcolor{white}{Caso de uso: CII\_CU3. Obtener certificado.}}}\\
    				\hline
    				\rowcolor{azulfuerte}Concepto & Descripción\\
    				\hline
    				\cellcolor{azulclaro}Actor & 
    				Componente I. Extensión.\\ 
    				\hline
    				\cellcolor{azulclaro}Propósito &
    				Retornar el certificado del usuario a la extensión desde donde éste inició sesión.\\
    				\hline
    				\cellcolor{azulclaro}Entradas &
    				Usuario y contraseña.\\
    				\hline
    				\cellcolor{azulclaro}Salidas &
    				Certificado asignado a dicho usuario.\\
    				\hline
    				\cellcolor{azulclaro}Pre-condiciones&
    				\textbf{CII\_CU3}.\\
    				\hline
    				\cellcolor{azulclaro}Post-condiciones&
    				-\\
    				\hline
    				\cellcolor{azulclaro}Reglas del negocio&
    				\hyperref[CII_RN1]{\textbf{CII\_RN1}} \newline
    				\hyperref[CII_RN2]{\textbf{CII\_RN2}} \newline
    				\hyperref[CII_RN3]{\textbf{CII\_RN3}} \\
    				\hline
    				\cellcolor{azulclaro}Errores &
    				El servidor autentificador no puede retornar el certificado a la extensión.\\		
    				\hline
    		    \end{tabular}
    		    \caption[DCU: CII\_CU3]{Descripción CU: CII\_CU3}
    		\end{table}
    		
    		\paragraph{... Trayectoria principal ...}
    		    \begin{enumerate}
    		        \item \textbf{\textit{La extensión}} solicita el certificado asignado al usuario con sesión iniciada actualmente.
    		        
    		        \item \textbf{\textit{El servidor autentificador}} valida al usuario registrado en su base de datos.
    		        
    		        \item \textbf{\textit{El servidor autentificador}} envía a la extensión el certificado asociado a este usuario.
    		    \end{enumerate}
    		\paragraph{... Fin Trayectoria principal ...}
    		
    		\paragraph{... Trayectoria Alternativa 1 ...}
    		    \begin{enumerate}
    		        \item \textbf{\textit{La extensión}} solicita el certificado asignado al usuario con sesión iniciada actualmente.
    		        
    		        \item \textbf{\textit{El servidor autentificador}} no encuentra el usuario registrado en su base de datos.
    		        
    		        \item \textbf{\textit{El servidor autentificador}} envía un código de error a la extensión.
    		        
    		        \item \textbf{\textit{La extensión}} despliega un mensaje de error.
    		    \end{enumerate}
    		\paragraph{... Fin Trayectoria Alternativa 1 ...}
    		
    		\paragraph{... Trayectoria Alternativa 2 ...}
    		    \begin{enumerate}
    		        \item \textbf{\textit{La extensión}} solicita el certificado asignado al usuario con sesión iniciada actualmente.
    		        
    		        \item \textbf{\textit{El servidor autentificador}} valida el usuario registrado en su base de datos.
    		        
    		        \item \textbf{\textit{El servidor autentificador}} no puede enviar el certificado a la extensión.
    		        
    		        \item \textbf{\textit{El servidor autentificador}} envía un código de error a la extensión.
    		        
    		        \item \textbf{\textit{La extensión}} despliega un mensaje de error.
    		    \end{enumerate}
    		\paragraph{... Fin Trayectoria Alternativa 1 ...}
		    \newpage
		
		
		
    		\begin{table}[H]
    			\begin{tabular}{ |p{3.5cm}||p{9.5cm}|}
    				\hline
    				\rowcolor{guindapoli}
    				\multicolumn{2}{|c|}{\textbf{\textcolor{white}{Caso de uso: CII\_CU4. Obtener código del certificado.}}}\\
    				\hline
    				\rowcolor{azulfuerte}Concepto & Descripción\\
    				\hline
    				\cellcolor{azulclaro}Actor & 
    			    Componente III. API.\\ 
    				\hline
    				\cellcolor{azulclaro}Propósito &
    				Retornar un SHA256 del certificado de un usuario especificado.\\
    				\hline
    				\cellcolor{azulclaro}Entradas &
    				Código SHA256 del nombre del usuario.\\
    				\hline
    				\cellcolor{azulclaro}Salidas &
    				Código SHA256 del certificado del usuario.\\
    				\hline
    				\cellcolor{azulclaro}Pre-condiciones&
    				Usuario registrado en el servidor autentificador.\\
    				\hline
    				\cellcolor{azulclaro}Post-condiciones&
    				- \\
    				\hline
    				\cellcolor{azulclaro}Reglas del negocio&
    				\hyperref[CII_RN1]{\textbf{CII\_RN1}} \newline
    				\hyperref[CII_RN2]{\textbf{CII\_RN2}} \newline
    				\hyperref[CII_RN6]{\textbf{CII\_RN6}} \\
    				\hline
    				\cellcolor{azulclaro}Errores &
    				El servidor autentificador no puede responder. \\
    				\hline
    		    \end{tabular}
    		    \caption[DCU: CII\_CU4]{Descripción CU: CII\_CU4}
    		\end{table}
    		
    		\paragraph{... Trayectoria principal ...}
    		    \begin{enumerate}
    		        \item \textbf{\textit{El servidor autentificador}} recibe un SHA256 del nombre de usuario enviado por \textbf{\textit{Componente III. API}}.
    		        
    		        \item \textbf{\textit{El servidor autentificador}} busca el certificado de dicho usuario.
    		        
    		        \item \textbf{\textit{El servidor autentificador}} retorna un código SHA256 del certificado del usuario.
    		    \end{enumerate}
    		\paragraph{... Fin Trayectoria principal ...}
    		
    		\paragraph{... Trayectoria alternativa 1 ...}
    		    \begin{enumerate}
    		        \item \textbf{\textit{El servidor autentificador}} recibe un SHA256 del nombre de usuario enviado por \textbf{\textit{Componente III. API}}.
    		        
    		        \item \textbf{\textit{El servidor autentificador}} busca el certificado de dicho usuario.
    		        
    		        \item \textbf{\textit{El servidor autentificador}} no encuentra el certificado debido a que el usuario no existe.
    		        
    		        \item \textbf{\textit{El servidor autentificador}} retorna un código de error.
    		    \end{enumerate}
    		\paragraph{... Fin Trayectoria alternativa 1 ...}
    	    \newpage
	    
		    \subsection{Diagrama de flujo.}
		
    		\begin{figure}[H]
            	\begin{center}	\includegraphics[width=11cm]{./imagenes/Disenio/Componente_2/CII_FD.png}
            	\caption{Diagrama de flujo de Componente II.}
            	\end{center}
    		\end{figure}
    		
    		\subsubsection{Descripción diagrama de flujo.}
    		El flujo de este componente empieza al recibir una petición, posteriormente analiza que es lo que se le solicitó, ya que este componente tiene múltiples propósitos, principalmente registrar usuarios, verificar el certificado que recibe y revocar certificados ya existentes. El componente recibirá una petición y si esta es para el inicio de sesión, entonces el servidor autentificador validará al usuario, revisando si se encuentra registrado o no, para el primer caso validará que la fecha de expiración de dicho certificado siga vigente, si es así entonces devolverá el certificado de este usuario, si no se encuentra vigente entonces se le mandará un mensaje al usuario para solicitar la renovación del mismo.    		Por el otro lado si no se encuentra registrado, se debe de crear un nuevo registro logrando de esta manera la generación de un certificado nuevo el cual se almacenará con los datos ingresados y finalmente se devolverá para que \textit{la extensión} pueda hacer uso del mismo. \\
    		
    	    Existe también el caso en el que se solicite al servidor autentificador revocar un certificado, para ello se va a buscar dicho certificado y si se encuentra quiere decir que el usuario está registrado y se puede empezar el proceso de revocarlo, primero generando un certificado nuevo para este usuario, después se elimina el certificado asociado a este usuario, y se le asigna el certificado previamente creado a este nuevo usuario, por último se le manda un mensaje al usuario para que inicie sesión y se pueda devolver el nuevo certificado. En el caso en el que se pida revocar un certificado y no se encuentre ninguna cuenta registrada a este usuario entonces se mandará un mensaje de error.
    		
    		\subsection{Diagrama de flujo de datos.}
    		\begin{figure}[H]
            	\begin{center}	\includegraphics[width=13cm]{./imagenes/Disenio/Componente_2/CII_DFD.png}
            	\caption{Diagrama de flujo de datos del Componente II.}
            	\end{center}
    		\end{figure}
    		
    		\subsubsection{Descripción diagrama de flujo de datos.}
    		En este caso contamos con dos entidades externas, la extensión que por una parte puede solicitar un certificado mediante los datos solicitados por la autoridad, solicitar la generación de un nuevo usuario o bien solicitar la revocación de un certificado ligado a un usuario. La extensión le proporcionará los datos necesarios a la autoridad certificadora para que esta pueda crear y guardar al usuario con un certificado único para el mismo. Al solicitar un certificado la autoridad le devolverá a la extensión el certificado perteneciente al usuario enviado. Si se solicita la creación de usuario, la autoridad solo devolverá como respuesta creación exitosa. Si la extensión solicita a la autoridad la revocación de un certificado, la autoridad necesita los datos del usuario a quien se le eliminará su certificado, para esto los datos del usuario deben ser correctos, si lo son, se procede a eliminar el certificado y a crear un nuevo, este certificado se guarda como certificado del usuario y se procede a devolver como respuesta dicho certificado creado previamente.
            Por otra parte, la API le solicitará a la autoridad la verificación de un certificado, esto con el fin de definir si el certificado fue generado por dicha autoridad y si ese certificado es valido.
		
		    \subsection{Diagrama de clases.}
		
    		\begin{figure}[H]
            	\begin{center}	\includegraphics[width=14cm]{./imagenes/Disenio/Componente_2/CII_CD.png}
            	\caption{Diagrama de clases de Componente II.}
            	\end{center}
    		\end{figure}
			    
			\subsubsection{Descripción de diagrama de clases}
			   
			\textbf{\textcolor{guindapoli}{Clase: \textit{CAController}}}\\
                    
                    \textbf{Atributos}
                    \begin{enumerate}
    		            \item \textbf{userRepo} : variable con la cual se pueden hacer operaciones con la base de datos.
        		        \begin{itemize}
        		            \item Tipo de dato: \textbf{IUserRepository}.
        		        \end{itemize}
        		        \item \textbf{certUtil} : variable con la que se pueden hacer operaciones con y sobre los certificados, como crearlos o eliminarlos.
        		        \begin{itemize}
        		            \item Tipo de dato: \textbf{CertificateUtil}.
        		        \end{itemize}
        		        \item \textbf{sha} : instancia para cifrar información con SHA256.
        		        \begin{itemize}
        		            \item Tipo de dato: \textbf{SHA256Util}.
        		        \end{itemize}
                    \end{enumerate}
                    
		            \textbf{Métodos}
    		        \begin{enumerate}
    		            \item \textbf{createNewUser(String pass, String email)}: Este método permite crear un nuevo usuario y guardar a éste en la base de datos .
    		            \begin{itemize}
    		                \item Tipo de dato de retorno: \textbf{User}
    		            \end{itemize}
    		            \item \textbf{revokeCert(String email, String pass)}: Este método permite revocar el certificado actual del usuario con las credenciales recibidas.
    		            \begin{itemize}
    		                \item Tipo de dato de retorno: \textbf{int}
    		            \end{itemize}
    		            \item \textbf{login(String pass, String email)}: Este método permite retornar el certificado del usuario de las credenciales recibidas.
    		            \begin{itemize}
    		                \item Tipo de dato de retorno: \textbf{String}
    		            \end{itemize}
    		            \item \textbf{checkCert(String SHAuser)}: Este método permite retornar el sha256 del certificado del usuario recibido.
    		            \begin{itemize}
    		                \item Tipo de dato de retorno: \textbf{String}
    		            \end{itemize}
    		        \end{enumerate}
    		        
    		    \textbf{\textcolor{guindapoli}{Clase: \textit{CertificateUtil}}}\\
                   
		            \textbf{Métodos}
    		        \begin{enumerate}
    		            \item \textbf{createCertForUser(User user)}: Este método permite crear un nuevo certificado con los datos del usuario.
    		            \begin{itemize}
    		                \item Tipo de dato de retorno: \textbf{String}
    		            \end{itemize}
    		            \item \textbf{revokeCert(String userName, String pass)}: Este método permite revocar el certificado actual del usuario.
    		            \begin{itemize}
    		                \item Tipo de dato de retorno: \textbf{int}
    		            \end{itemize}
    		        \end{enumerate}
    		        
    		    
    		    \textbf{\textcolor{guindapoli}{Clase: \textit{SHA256Util}}}\\
                   
		            \textbf{Métodos}
    		        \begin{enumerate}
    		            \item \textbf{doSHA(String message)}: Este método permite cifrar el mensaje recibido con SHA256.
    		            \begin{itemize}
    		                \item Tipo de dato de retorno: \textbf{String}
    		            \end{itemize}
    		        \end{enumerate}
    		        
    		        
    		    \textbf{\textcolor{guindapoli}{Clase: \textit{UserRepository}}}\\
                   
		            \textbf{Métodos}
    		        \begin{enumerate}
    		            \item \textbf{saveUser(User user)}: Este método permite guardar en la base de datos el usuario recibido.
    		            \begin{itemize}
    		                \item Tipo de dato de retorno: \textbf{User}
    		            \end{itemize}
    		        \end{enumerate}
    		        
    		    
    		    \textbf{\textcolor{guindapoli}{Clase: \textit{User}}}\\
                   
		            \textbf{Atributos}
    		        \begin{enumerate}
    		             \item \textbf{email}: variable que almacena el email del usuario.
    		            \begin{itemize}
    		                \item Tipo de dato: \textbf{String}
    		            \end{itemize}
    		             \item \textbf{password}: variable que almacena la contraseña del usuario.
    		            \begin{itemize}
    		                \item Tipo de dato: \textbf{String}
    		            \end{itemize}
    		             \item \textbf{cert}: variable que almacena el certificado del usuario.
    		            \begin{itemize}
    		                \item Tipo de dato: \textbf{X509Certificate}
    		            \end{itemize}
    		        \end{enumerate}
    		    
		    \subsection{Diagramas de secuencias.}
		
		        \subsubsection{Diagrama de secuencia 1}
            	\begin{figure}[H]
    			    \begin{center} \includegraphics[width=12cm]{./imagenes/Disenio/Componente_2/CII_SD_UC1.png}
    			    \caption[Diagrama de secuencia 1 del Componente II]{Diagrama de secuencia del CII\_CU1. Crear Nuevo usuario.}
    		        \end{center}
    		    \end{figure}
    		    
    		    \paragraph{Descripción:} En este diagrama se explica el primer caso de uso, que es el de \textit{crear un nuevo usuario}, donde el usuario solicita mediante la extensión el crear un nuevo usuario en el servidor autentificador, para que pueda empezar a utilizar nuestro servidor y asociar certificados a este usuario. Cabe resaltar que consideraremos cada correo que se reciba como un usuario.
    		    
    		    \subsubsection{Diagrama de secuencia 2}
            	\begin{figure}[H]
    			    \begin{center} \includegraphics[width=12cm]{./imagenes/Disenio/Componente_2/CII_SD_UC2.png}
    			    \caption[Diagrama de secuencia 2 del Componente II]{Diagrama de secuencia del CII\_CU2. Revocar certificado}
    		        \end{center}
    		    \end{figure}
    		    
    		    \paragraph{Descripción:} En este diagrama, el usuario mediante la extensión solicita que se revoque su certificado, el servidor recibe esta solicitud y valida los datos, si son correctos genera un nuevo certificado y se lo asigna a este usuario.
    		    
    		    \subsubsection{Diagrama de secuencia 3}
            	\begin{figure}[H]
    			    \begin{center} \includegraphics[width=12cm]{./imagenes/Disenio/Componente_2/CII_SD_UC3.png}
    			    \caption[Diagrama de secuencia 3 del Componente II]{Diagrama de secuencia del CII\_CU3. Obtener certificado.}
    		        \end{center}
    		    \end{figure}
    		    
    		    \paragraph{Descripción:} Para este diagrama, se obtienen un certificado desde el servidor a la extensión, esta después valida los datos y si son correctos, guarda en el storage de Chrome el certificado de ese usuario.
    		    
    		    \subsubsection{Diagrama de secuencia 4}
            	\begin{figure}[H]
    			    \begin{center} \includegraphics[width=12cm]{./imagenes/Disenio/Componente_2/CII_SD_UC4.png}
    			    \caption[Diagrama de secuencia 4 del Componente II]{Diagrama de secuencia del CII\_CU4. Verificar certificado.}
    		        \end{center}
    		    \end{figure}
    		    
    		    \paragraph{Descripción:} Este diagrama servira para verificar los certificados que obtiene el servidor desde la extensión, donde el servidor valida si el certificado que recibe existe en su base de datos, y le envía una respuesta a la extensión, en este caso se regresa como respuesta un hash 256 del certificado del usuario..
    		    
	        \subsection{Diagrama de actividades.}
	    
        	    \begin{figure}[H]
                	\begin{center}	\includegraphics[width=13cm]{./imagenes/Disenio/Componente_2/CII_DA.png}
                	\caption{Diagrama de actividades de Componente II.}
                	\end{center}
        		\end{figure}
		
        		\subsubsection{Descripción del diagrama de actividades.}
        		El componente cuenta con los siguientes pasos.
        		Por una parte la interacción entre la extensión y la autoridad certificadora, la cual la extensión podrá solicitar la creación de un nuevo usuario, iniciar sesión para obtener el certificado y la revocación de un certificado. Para el inicio de sesión la extensión debe de enviarle los datos de usuario para asi la autoridad poder verificar dichos datos, si los datos son correctos y existen el la base de datos se procede a generar un certificado para el usuario y se registra en la base de datos. Si se solicita el inicio de sesión, es decir obtener un certificado, la autoridad al recibir dicha petición se procede a buscar el usuario solicitado, si existe la autoridad regresa como respuesta el certificado del usuario. Si la extensión solicita la revocación de un certificado, la autoridad debe buscar al usuario en la base de datos, si existe un usuario con esos datos, se procede a eliminar su certificado y crear uno nuevo, devolviendo como respuesta el nuevo certificado. Ahora bien, si la API solicita la verificación de un certificado, la autoridad deberá verificar dicho certificado, si el certificado fue creado por la autoridad, este regresa respuesta satisfactoria.
	
	    \section{Componente III: API.}
	
	        \subsection{Diagrama de casos de uso.}
	    
        	    \begin{figure}[H]
        	        \centering       	        \includegraphics[height=11cm]{imagenes/Disenio/Componente_3/CIII_UCD.png}
        	        \caption{Diagrama de caso de uso del componente III}
        	        \label{fig:CIII_UCD}
        	    \end{figure}
	    
	        \subsubsection{Descripción de diagrama de casos de uso.}
	        
	        \end{comment}
	        \begin{comment}
    	    \begin{table}[H]
    			\begin{tabular}{ |p{3.5cm}||p{9.5cm}|}
    				\hline
    				\rowcolor{guindapoli}
    				\multicolumn{2}{|c|}{\textbf{\textcolor{white}{Caso de uso: CIII\_CU1. Iniciar sesión.}}}\\
    				\hline
    				\rowcolor{azulfuerte}Concepto & Descripción\\
    				\hline
    				\cellcolor{azulclaro}Actor & Usuario\\ 
    				\hline
    				\cellcolor{azulclaro}Propósito &
    				Acceder al servicio web. Para la primera ocasión en que se usa este método de autentificación (con un nuevo certificado) se requerirá iniciar sesión con el usuario y la contraseña, el resto de las ocasiones el usuario tendrá acceso automático al servicio.\\
    				\hline
    				\cellcolor{azulclaro}Entradas &
    				Petición del Componente I con chaffing and winnowing.\\
    				\hline
    				\cellcolor{azulclaro}Salidas &
    				Acceso o no al servicio web.\\
    				\hline
    				\cellcolor{azulclaro}Pre-condiciones&
    				El Componente I debió haber mandado una petición con el certificado inyectado (chaffing).\\
    				\hline
    				\cellcolor{azulclaro}Post-condiciones&
    				-\\
    				\hline
    				\cellcolor{azulclaro}Reglas del negocio&
    				\hyperref[CIII_RN1]{\textbf{CIII\_RN1}}\newline
    				\hyperref[CIII_RN2]{\textbf{CIII\_RN2}}\newline
    				\hyperref[CIII_RN3]{\textbf{CIII\_RN3}}\newline
    				\hyperref[CIII_RN4]{\textbf{CIII\_RN4}}
    				\\
    				\hline
    				\cellcolor{azulclaro}Errores &
    				No se pudo obtener el certificado.\newline
    				Las credenciales son incorrectas.\\
    				\hline
    		    \end{tabular}
    		    \caption[DCU: CIII\_CU1]{Descripción CU: CIII\_CU1}
    		\end{table}
    		
    		\paragraph{... Trayectoria Principal ...}
    			\begin{enumerate}
    				\item \textbf{\textit{El servicio web}} recibe una nueva petición HTTP con el certificado inyectado.
    				
    				\item \textbf{\textit{El servicio web}} manda el chaffing y el pattern al \textbf{\textit{API}}.
    				
    				\item \textbf{\textit{La API}} valida el certificado.
    				
    				\item \textbf{\textit{La API}} retorna que el certificado es válido.
    				
    				\item \textbf{\textit{El servicio web}} reconoce el certificado.
    				
    				\item \textbf{\textit{El servicio web}} le da acceso al sistema al usuario.
    			\end{enumerate}
    		\paragraph{... Fin de la Trayectoria Principal ...}
    		
    		\paragraph{... Trayectoria Alternativa 1 ...}
    			\begin{enumerate}
    				\item \textbf{\textit{El servicio web}} recibe una nueva petición HTTP con el certificado inyectado.
    				
    				\item \textbf{\textit{El servicio web}} manda el chaffing y el pattern al \textbf{\textit{API}}.
    				
    				\item \textbf{\textit{La API}} valida el certificado.
    				
    				\item \textbf{\textit{La API}} retorna que el certificado es inválido.
    				
    				\item \textbf{\textit{El servicio web}} le niega el acceso al sistema al usuario.
    			\end{enumerate}
    		\paragraph{... Fin de la Trayectoria Alternativa 1 ...}
    			
    		\paragraph{... Trayectoria Alternativa 2 ...}
    			\begin{enumerate}
    				\item \textbf{\textit{El servicio web}} recibe una nueva petición HTTP con el certificado inyectado.
    				
    				\item \textbf{\textit{El servicio web}} manda el chaffing y el pattern al \textbf{\textit{API}}.
    				
    				\item \textbf{\textit{La API}} valida el certificado.
    				
    				\item \textbf{\textit{La API}} retorna que el certificado es válido.
    				
    				\item \textbf{\textit{El servicio web}} no reconoce el certificado.
    				
    				\item \textbf{\textit{El servicio web}} pide al \textbf{\textit{Usuario}} que inicie sesión con las credenciales para vincular el certificado.
    				
    				\item \textbf{\textit{El servicio web}} valida las credenciales.
    				
    				\item \textbf{\textit{El servicio web}} vincula el certificado.
    				
    				\item \textbf{\textit{El servicio web}} le da acceso al sistema al usuario.
    			\end{enumerate}
    		\paragraph{... Fin de la Trayectoria Alternativa 2 ...}
    		
    		\paragraph{... Trayectoria Alternativa 3 ...}
    			\begin{enumerate}
    				\item \textbf{\textit{El servicio web}} recibe una nueva petición HTTP con el certificado inyectado.
    				
    				\item \textbf{\textit{El servicio web}} manda el chaffing y el pattern al \textbf{\textit{API}}.
    				
    				\item \textbf{\textit{La API}} valida el certificado.
    				
    				\item \textbf{\textit{La API}} retorna que el certificado es válido.
    				
    				\item \textbf{\textit{El servicio web}} no reconoce el certificado.
    				
    				\item \textbf{\textit{El servicio web}} pide al \textbf{\textit{Usuario}} que inicie sesión con las credenciales para vincular el certificado.
    				
    				\item \textbf{\textit{El servicio web}} detecta que las credenciales no son válidas.
    				
    				\item \textbf{\textit{El servicio web}} le niega el acceso al sistema al usuario.
    			\end{enumerate}
    		\paragraph{... Fin de la Trayectoria Alternativa 3 ...}
    		
    		\paragraph{... Trayectoria Alternativa 4 ...}
    			\begin{enumerate}
    				\item \textbf{\textit{El servicio web}} recibe una nueva petición HTTP con el certificado inyectado.
    				
    				\item \textbf{\textit{El servicio web}} manda el chaffing y el pattern al \textbf{\textit{API}}.
    				
    				\item \textbf{\textit{La API}} no puede obtener el certificado.
    				
    				\item \textbf{\textit{La API}} retorna que el certificado es inválido.
    				
    				\item \textbf{\textit{El servicio web}} niega el acceso al sistema al usuario.
    			\end{enumerate}
    		\paragraph{... Fin de la Trayectoria Alternativa 4 ...}
    	    \newpage
    		\end{comment}
    		\begin{comment}
    		
    		
    		\begin{table}[H]
    			\begin{tabular}{ |p{3.5cm}||p{9.5cm}|}
    				\hline
    				\rowcolor{guindapoli}
    				\multicolumn{2}{|c|}{\textbf{\textcolor{white}{Caso de uso: CIII\_CU1. Comprobar certificado.}}}\\
    				\hline
    				\rowcolor{azulfuerte}Concepto & Descripción\\
    				\hline
    				\cellcolor{azulclaro}Actor & Componente II. Servidor autentificador\\ 
    				\hline
    				\cellcolor{azulclaro}Propósito &
    				Saber si el certificado ha sido o no revocado por el usuario.\\
    				\hline
    				\cellcolor{azulclaro}Entradas &
    				SHA256 del usuario del certificado.\\
    				\hline
    				\cellcolor{azulclaro}Salidas &
    				SHA256 del certificado en el servidor autentificador del usuario.\\
    				\hline
    				\cellcolor{azulclaro}Pre-condiciones&
    				-\\
    				\hline
    				\cellcolor{azulclaro}Post-condiciones&
    				-\\
    				\hline
    				\cellcolor{azulclaro}Reglas del negocio&
    				\hyperref[CIII_RN3]{\textbf{CIII\_RN3}}\\
    				\hline
    				\cellcolor{azulclaro}Errores &
    			    No se encuentra al usuario.\\
    				\hline
    		    \end{tabular}
    		    \caption[DCU: CIII\_CU1]{Descripción CU: CIII\_CU1}
    		\end{table}
    		
    		\paragraph{... Trayectoria Principal ...}
    			\begin{enumerate}
    				\item \textbf{\textit{La API}} envía un SHA256 del usuario del certificado al \textbf{\textit{Servidor autentificador}}.
    				
    				\item \textbf{\textit{El servidor autentificador}} retorna un SHA256 del certificado actual en el servidor de ese usuario.
    				
    				\item \textbf{\textit{La API}} comprueba que el SHA256 del certificado obtenido de la petición es igual al SHA256 del certificado obtenido del \textbf{\textit{Servidor autentificador}}
    				
    				\item \textbf{\textit{La API}} retorna al \textbf{\textit{Servicio web}} que el certificado es válido
    			\end{enumerate}
    		\paragraph{... Fin de la Trayectoria Principal ...}
    			
    		\paragraph{... Trayectoria Alternativa 1 ...}
    			\begin{enumerate}
    				\item \textbf{\textit{La API}} envía un SHA256 del usuario del certificado al \textbf{\textit{Servidor autentificador}}.
    				
    				\item \textbf{\textit{El servidor autentificador}} retorna un SHA256 del certificado actual en el servidor de ese usuario.
    				
    				\item \textbf{\textit{La API}} comprueba que el SHA256 del certificado obtenido de la petición no es igual al SHA256 del certificado obtenido del \textbf{\textit{Servidor autentificador}}
    				
    				\item \textbf{\textit{La API}} retorna al \textbf{\textit{Servicio web}} que el certificado es inválido
    			\end{enumerate}
    		\paragraph{... Fin de la Trayectoria Alternativa 1 ...}
    		
    		\paragraph{... Trayectoria Alternativa 2 ...}
    			\begin{enumerate}
    				\item \textbf{\textit{La API}} envía un SHA256 del usuario del certificado al \textbf{\textit{Servidor autentificador}}.
    				
    				\item \textbf{\textit{El servidor autentificador}} no encuantra al usuario y retorna un código de error.
    				
    				\item \textbf{\textit{La API}} retorna al \textbf{\textit{Servicio web}} que el certificado es inválido
    			\end{enumerate}
    		\paragraph{... Fin de la Trayectoria Alternativa 1 ...}
    		\newpage
    			
    			
    		
    		\begin{table}[H]
    			\begin{tabular}{ |p{3.5cm}||p{9.5cm}|}
    				\hline
    				\rowcolor{guindapoli}
    				\multicolumn{2}{|c|}{\textbf{\textcolor{white}{Caso de uso: CIII\_CU2. Enviar petición.}}}\\
    				\hline
    				\rowcolor{azulfuerte}Concepto & Descripción\\
    				\hline
    				\cellcolor{azulclaro}Actor & Componente I. Extensión \\ 
    				\hline
    				\cellcolor{azulclaro}Propósito &
    				Recibir la petición HTTP con el certificado inyectado.\\
    				\hline
    				\cellcolor{azulclaro}Entradas &
    				Petición HTTP con certificado inyectado para poder iniciar sesión.\\
    				\hline
    				\cellcolor{azulclaro}Salidas &
    			    Acceso o no para el usuario al servicio web.\\
    				\hline
    				\cellcolor{azulclaro}Pre-condiciones&
    				-.\\
    				\hline
    				\cellcolor{azulclaro}Post-condiciones&
    				\textbf{CIII\_CU4} \\
    				\hline
    				\cellcolor{azulclaro}Reglas del negocio&
    				\hyperref[CIII_RN4]{\textbf{CIII\_RN4}}\\
    				\hline
    				\cellcolor{azulclaro}Errores &
    				La petición no tiene el certificado inyectado.\\
    				\hline
    		    \end{tabular}
    		    \caption[DCU: CIII\_CU2]{Descripción CU: CIII\_CU2}
    		\end{table}
    		
    		\paragraph{... Trayectoria Principal ...}
    			\begin{enumerate}
    				\item \textbf{\textit{La extensión}} envía el certificado inyectado en la petición HTTP
    				
    				\item \textbf{\textit{El servicio web}} recibe la petición.
    			\end{enumerate}
    		\paragraph{... Fin de la Trayectoria Principal ...}
    		
    		\paragraph{... Trayectoria Alternativa 1 ...}
    			\begin{enumerate}
    				\item \textbf{\textit{La extensión}} no envía el certificado inyectado en la petición HTTP
    				
    				\item \textbf{\textit{El servicio web}} recibe la petición.
    			\end{enumerate}
    		\paragraph{... Fin de la Trayectoria Alternativa 1 ...}
            \newpage
    		
    		
    		\begin{table}[H]
    			\begin{tabular}{ |p{3.5cm}||p{9.5cm}|}
    				\hline
    				\rowcolor{guindapoli}
    				\multicolumn{2}{|c|}{\textbf{\textcolor{white}{Caso de uso: CIII\_CU3. Redirigir página.}}}\\
    				\hline
    				\rowcolor{azulfuerte}Concepto & Descripción\\
    				\hline
    				\cellcolor{azulclaro}Actor & Navegador \\ 
    				\hline
    				\cellcolor{azulclaro}Propósito &
    			    Redirigir la respuesta del servicio web para darle información al usuario acerca de su inicio de sesión.\\
    				\hline
    				\cellcolor{azulclaro}Entradas &
    				Respuesta del servicio web.\\
    				\hline
    				\cellcolor{azulclaro}Salidas &
                    Despliegue de la respuesta en el navegador.\\
    				\hline
    				\cellcolor{azulclaro}Pre-condiciones&
    				\textbf{CIII\_CU3}.\\
    				\hline
    				\cellcolor{azulclaro}Post-condiciones&
    				-\\
    				\hline
    				\cellcolor{azulclaro}Reglas del negocio&
    				-\\
    				\hline
    				\cellcolor{azulclaro}Errores &
    				No se puede desplegar la respuesta del servicio web.\\
    				\hline
    		    \end{tabular}
    		    \caption[DCU: CIII\_CU3]{Descripción CU: CIII\_CU3}
    		\end{table}
    		
    		\paragraph{... Trayectoria Principal ...}
    			\begin{enumerate}
    				\item \textbf{\textit{El servicio web}} envía la respuesta.
    				
    				\item \textbf{\textit{El navegador}} muestra la respuesta.
    			\end{enumerate}
    		\paragraph{... Fin de la Trayectoria Principal ...}
    		
    		\paragraph{... Trayectoria Alternativa 1 ...}
    			\begin{enumerate}
    				\item \textbf{\textit{El servicio web}} envía la respuesta.
    				
    				\item \textbf{\textit{El navegador}} no muestra la respuesta.
    			\end{enumerate}
    		\paragraph{... Fin de la Trayectoria Alternativa 1 ...}
			\newpage
	     
	     
            \subsection{Diagrama de flujo.}
            
                \begin{figure}[H]
                	\begin{center}	\includegraphics[width=11cm]{./imagenes/Disenio/Componente_3/CIII_DF.png}
                	\caption{Diagrama de flujo de Componente III.}
                	\end{center}
        		\end{figure}
        
                \subsubsection{Descripción diagrama de flujo.}
                Para el caso de este diagrama se inicia con una petición, la cual es recibida por la API y analizada para verificar si el encabezado tiene el certificado. Si la petición no tiene el certificado, este componente ignora dicha petición. Si la petición contiene un certificado, se procede a iniciar sesión con certificado,  Si se realiza satisfactoriamente el winnowing, se inicia sesión donde puede o no estar registrado el usuario, de cualquier manera el usuario al registrarse o iniciar sesión se usara el certificado.
        
            \subsection{Diagrama de flujo de datos.}
        
                \begin{figure}[H]
                	\begin{center}	\includegraphics[width=12cm]{./imagenes/Disenio/Componente_3/CIII_DFD.png}
                	\caption{Diagrama de flujo de datos de Componente III.}
                	\end{center}
        		\end{figure}
        
                \subsubsection{Descripción diagrama de flujo de datos.}
            
                A continuación tenemos el diagrama de flujo de datos, donde podemos ver cláramente como viaja la información principal a traves de este componente y con las entidades externas, primero la extensión envía una petición al servicio web, este lo recibe, la API lo intercepta y verifica si es una petición con Chaffing que debe de ser analizada, si es así realiza la etapa de winnowing descifrando el patrón con su llave privada y por último obtiene el certificado dentro de esta petición y la compara con las que cuenta con el servicio web, para saber si debe de dar una respuesta de usuario o solicitar que inicie sesión en este mismo.
        
            \subsection{Diagrama de clases.}
            
                \begin{figure}[H]
                	\begin{center}	\includegraphics[width=14cm]{./imagenes/Disenio/Componente_3/CIII_CD.png}
                	\caption{Diagrama de Clases de componente 3.}
                	\end{center}
        		\end{figure}      
            
                \subsubsection{Descripción diagrama de clases.}
               
                \textbf{\textcolor{guindapoli}{Clase: \textit{APIController}}}\\
                    
                    \textbf{Atributos}
                    \begin{enumerate}
    		            \item \textbf{winnowingService} : variable donde se tendrá acceso a los métodos necesarios para realizar la etapa de winnowing.
        		        \begin{itemize}
        		            \item Tipo de dato: \textbf{IWinnowing}.
        		        \end{itemize}
                    \end{enumerate}
                    
		            \textbf{Métodos}
    		        \begin{enumerate}
    		            \item \textbf{makeWinnowing(WinnowingModel wm)}: Este método realiza la etapa de winnowing, retorna el certificado.
    		            \begin{itemize}
    		                \item Tipo de dato de retorno: \textbf{String}
    		            \end{itemize}
    		        \end{enumerate}
			    
			    
                
                \textbf{\textcolor{guindapoli}{Clase: \textit{WinnowingModel}}}\\
                    
                    \textbf{Atributos}
                    \begin{enumerate}
    		            \item \textbf{chaffing} : variable donde se tiene guardado el chaffing de la petición.
        		        \begin{itemize}
        		            \item Tipo de dato: \textbf{String}.
        		        \end{itemize}
        		        \item \textbf{pattern} : variable donde se tiene guardado el patrón de la petición.
        		        \begin{itemize}
        		            \item Tipo de dato: \textbf{String}.
        		        \end{itemize}
                    \end{enumerate}


                   
                \textbf{\textcolor{guindapoli}{Clase: \textit{CryptoService}}}\\
                    
                    \textbf{Atributos}
                    \begin{enumerate}
    		            \item \textbf{rsa} : instancia para descifrar información con RSA.
        		        \begin{itemize}
        		            \item Tipo de dato: \textbf{CipherUtilityRSA}.
        		        \end{itemize}
        		        \item \textbf{sha} : instancia para cifrar información con SHA256.
        		        \begin{itemize}
        		            \item Tipo de dato: \textbf{CipherUtilitySHA256}.
        		        \end{itemize}
        		        \item \textbf{sig} : instancia para validar información de un certificado.
        		        \begin{itemize}
        		            \item Tipo de dato: \textbf{SignatureUtility}.
        		        \end{itemize}
                    \end{enumerate}
                    
		            \textbf{Métodos}
    		        \begin{enumerate}
    		            \item \textbf{doSHA(String message)}: Este método calcula el sha256 de la cadena de texto message.
    		            \begin{itemize}
    		                \item Tipo de dato de retorno: \textbf{String}
    		            \end{itemize}
    		            \item \textbf{decryptPattern(String encryptedPattern)}: Este método descifra el patrón de chaffing.
    		            \begin{itemize}
    		                \item Tipo de dato de retorno: \textbf{String}
    		            \end{itemize}
    		            \item \textbf{verifyCertificate(String certificate)}: Este método verifica la autenticidad de un certificado.
    		            \begin{itemize}
    		                \item Tipo de dato de retorno: \textbf{int}
    		            \end{itemize}
    		        \end{enumerate} 
    		        
    		        
    		    
    		    \textbf{\textcolor{guindapoli}{Clase: \textit{WinnowingService}}}\\
                    
                    \textbf{Atributos}
                    \begin{enumerate}
    		            \item \textbf{cryptoService} : instancia para acceder a todas las utilidades de cifrado.
        		        \begin{itemize}
        		            \item Tipo de dato: \textbf{CryptoService}.
        		        \end{itemize}
        		        \item \textbf{base64} : instancia para decodificar información en formato BASE64.
        		        \begin{itemize}
        		            \item Tipo de dato: \textbf{Base64}.
        		        \end{itemize}
        		        \item \textbf{chaffing} : variable para guardar el chaffing actual.
        		        \begin{itemize}
        		            \item Tipo de dato: \textbf{String}.
        		        \end{itemize}
        		        \item \textbf{pattern} : variable para guardar el patron de chaffing actual.
        		        \begin{itemize}
        		            \item Tipo de dato: \textbf{String}.
        		        \end{itemize}
                    \end{enumerate}
                    
		            \textbf{Métodos}
    		        \begin{enumerate}
    		            \item \textbf{makeWinnowing()}: Este método realiza la etapa de winnowing.
    		            \begin{itemize}
    		                \item Tipo de dato de retorno: \textbf{String}
    		            \end{itemize}
    		        \end{enumerate} 
    		        
    		        
    		    
    		    \textbf{\textcolor{guindapoli}{Clase: \textit{CipherUtilityRSA}}}\\
                    
		            \textbf{Métodos}
    		        \begin{enumerate}
    		            \item \textbf{encrypt(String message, PublicKey publicKey)}: Este método cifra un mensaje con la llave pública especificada.
    		            \begin{itemize}
    		                \item Tipo de dato de retorno: \textbf{String}
    		            \end{itemize}
    		            \item \textbf{decrypt(String message, PrivateKey privateKey)}: Este método descifra un mensaje con la llave privada especificada.
    		            \begin{itemize}
    		                \item Tipo de dato de retorno: \textbf{String}
    		            \end{itemize}
    		        \end{enumerate} 
    		        
    		  
    		    
    		    \textbf{\textcolor{guindapoli}{Clase: \textit{CipherUtilitySHA256}}}\\
                    
		            \textbf{Métodos}
    		        \begin{enumerate}
    		            \item \textbf{doSHA(String message)}: Este método cifra el mensaje que se le manda.
    		            \begin{itemize}
    		                \item Tipo de dato de retorno: \textbf{String}
    		            \end{itemize}
    		        \end{enumerate}
    		        
    		        
    		        
    		    \textbf{\textcolor{guindapoli}{Clase: \textit{SignatureUtility}}}\\
                    
		            \textbf{Métodos}
    		        \begin{enumerate}
    		            \item \textbf{verifyCertificado(String certificate)}: Este método verifica la validez de un certificado.
    		            \begin{itemize}
    		                \item Tipo de dato de retorno: \textbf{int}
    		            \end{itemize}
    		        \end{enumerate}
    		        
    		        
    		        
    		    \textbf{\textcolor{guindapoli}{Clase: \textit{Base64}}}\\
                    
		            \textbf{Métodos}
    		        \begin{enumerate}
    		            \item \textbf{encode(String message)}: Este método codifica el mensaje recibido a base64.
    		            \begin{itemize}
    		                \item Tipo de dato de retorno: \textbf{String}
    		            \end{itemize}
    		            \item \textbf{decode(String message)}: Este método decodifica el mensaje recibido de base64 a UTF-8.
    		            \begin{itemize}
    		                \item Tipo de dato de retorno: \textbf{String}
    		            \end{itemize}
    		        \end{enumerate}
                    
            \subsection{Diagramas de secuencias.}
        
                \begin{comment}
                \textbf{Diagrama de Secuencia 1.Iniciar sesión}    
                \begin{figure}[H]
                	\begin{center}	\includegraphics[width=12cm]{./imagenes/Disenio/Componente_3/CIII_SD_UC1.png}
                	\caption{Diagrama de Secuencia de Caso de Uso 1. Iniciar sesión}
                	\end{center}
        		\end{figure}    
            
                \subsubsection{Descripción de diagrama de Secuencia CIII\_SD\_UC1.}
                Para este diagrama, el servicio web recibirá una petición, y posteriormente la API se encargará de interceptarla, realizará la etapa de Winnowing y obtendrá el certificado de la petición con Chaffing, se la enviará al Servidor Autentificador para que este busque si ya tiene registrado a este certificado, enviando una respuesta al servicio web del tipo acceso que dará, ya que si no lo tiene registrado tendrá que iniciar sesión de forma tradicional por primera vez en el servicio, de forma contraria si se encuentra registrado dará acceso de forma inmediata a la sesión del usuario correspondiente.\\
                \end{comment}
                \begin{comment}
                \textbf{Diagrama de Secuencia 1. Comprobar Certificado.} 
                \begin{figure}[H]
                	\begin{center}	\includegraphics[width=12cm]{./imagenes/Disenio/Componente_3/CIII_SD_UC2.png}
                	\caption{Diagrama de Secuencia de Caso de Uso 1. Comprobar certificado.}
                	\end{center}
        		\end{figure}
            
                \subsubsection{Descripción de diagrama de Secuencia CIII\_SD\_UC1.}
                Después de que el servicio web reciba una petición, la API la interceptará para realizar la etapa de Winnowing y posteriormente envíe el certificado al servidor autentificador, el Servidor Autentificador validará los datos del usuario y dependiendo si existe o no ese usuario registrado, con un certificado válido(actualizado) o uno inválido(revocado) le regresará un status a la API y está sabrá el como darle respuesta al Servicio Web basado en el estatus que recibió del Servidor Autentificador.\\
        
                \textbf{Diagrama de Secuencia 2. Enviar Petición.}    
                \begin{figure}[H]
                	\begin{center}	\includegraphics[width=12cm]{./imagenes/Disenio/Componente_3/CIII_SD_UC3.png}
                	\caption{Diagrama de Secuencia de Caso de Uso 2. Enviar petición.}
                	\end{center}
        		\end{figure}
    		
                \subsubsection{Descripción de diagrama de Secuencia CIII\_SD\_UC2.}
                La extensión envía una petición con Chaffing al Servicio Web, donde la API intercepta dicha petición para analizar si es una petición que contenga un Chaffing el cuál podamos analizar, posteriormente realizará la etapa de Winnowing si es una petición de nuestro interés y enviará el certificado al servicio web.\\
        
                \textbf{Diagrama de Secuencia 3. Redirigir Página.} 
                \begin{figure}[H]
                	\begin{center}	\includegraphics[width=12cm]{./imagenes/Disenio/Componente_3/CIII_SD_UC4.png}
                	\caption{Diagrama de Secuencia de Caso de Uso 3. Redirigir página.}
                	\end{center}
        		\end{figure}
    		
                \subsubsection{Descripción de diagrama de Secuencia CIII\_SD\_UC3.}
                El navegador teniendo lista la petición con Chaffing la envía al Servicio Web, donde éste mediante la API, valida dicha petición y elige el inicio de sesión para el usuario, posteriormente le envía una respuesta al navegador y por último este mismo redirige la página dependiendo es esta misma respuesta.
            
            \subsection{Diagrama de actividades}
            
                \begin{figure}[H]
                	\begin{center}	\includegraphics[width=13cm]{./imagenes/Disenio/Componente_3/CIII_DA.png}
                	\caption{Diagrama de Actividades de componente 3.}
                	\end{center}
        		\end{figure}   
        
                \subsubsection{Descripción diagrama de actividades.}
                Cuando el servicio web recibe una petición con chaffing, realiza la etapa de Winnowing para poder obtener el certificado, y posteriormente se comunica con la autoridad certificadora, este comparará el certificado, ya que existen 3 casos posibles: el primero es que el certificado sea válido, y simplemente le dará acceso al servicio web con los datos del certificado correspondiente, el segundo caso es que el usuario sea válido, pero el certificado haya sido revocado antes y necesite actualizar su certificado en ese ordenador y el tercero es que el certificado sea inválido, por lo cual le dará un mensaje al usuario de que el certificado no es el correcto.
            
            \subsection{Interfaz de usuario.}
                Una vez que el usuario tiene una cuenta y ha obtenido su certificado de la autoridad certificadora. El usuario puede proceder a navegar en la red para hacer uso de la extensión. En el componente 3, las interfaces que se muestran al usuario son las vistas del servicio web que se ha implementado para llevar a cabo el proceso de \textit{Winnowing}. \\
                El servicio web de prueba que se utilizará para este trabajo es una pagina web para usuarios y doctores de una \textit{veterinaria}, la cual se utilizará para la modificación correspondiente donde el servicio lleve a cabo la autenticación por \textit{Chaffing and Winnowing}.\\
        		El servicio web de la veterinaria se muestra en la figura \ref{fig:UI_homeServicioWeb}.
    		
        		\begin{figure}[H]
    				\begin{center}	\includegraphics[width=12cm]{./imagenes/Disenio/Componente_1/UI_homeServicioWeb.PNG}
    					\caption[Mensaje de éxito]{Interfaz principal del servicio web de veterinaria.}
    				\label{fig:UI_homeServicioWeb}
    				\end{center}
    			\end{figure}
			
    			En esta interfaz dada por el servicio web, le permite al usuario iniciar sesión. Como la extensión esta activada, ésta realizará la tarea de bloquear la petición, para hacer el proceso de \textit{Chaffing} y posteriormente inyectar el resultado en el encabezado de la petición (ver Componente 1).
        	        
    	        Al recibir la petición en el servicio web, éste enviará la petición a la API que se implementa en el servicio web. La API es la encargada de realizar el proceso de \textit{Winnowing} y verificará la autenticidad del certificado.\\
    	        
    	        Una vez verificada la autenticidad del certificado, se muestra la interfaz de la figura \ref{fig:UI_iniciarSesionChaffing} en el cual se le da a indicar al usuario que se ha detectado un certificado y esta listo para vincularlo a una cuenta existente en el servicio web.
	        
    	        \begin{figure}[H]
    				\begin{center}	\includegraphics[width=9cm]{./imagenes/Disenio/Componente_1/UI_iniciarSesionChaffing.PNG}
    					\caption[Interfaz]{Interfaz para iniciar sesión en el servicio web con certificado en el encabezado de la petición.}
    				\label{fig:UI_iniciarSesionChaffing}
    				\end{center}
    			\end{figure}
	        
    	        Una vez que el usuario ingrese sus credenciales correctas del servicio web, éste vinculará el certificado a este usuario y a su vez le dará acceso al servicio con su cuenta.
    	        La figura \ref{fig:UI_homeUsuario} muestra la página de inicio del servicio web con la cuenta del usuario la cual ya tiene un certificado vinculado.
	        
    	        \begin{figure}[H]
    				\begin{center}	\includegraphics[width=10cm]{./imagenes/Disenio/Componente_1/UI_homeUsuario.PNG}
    					\caption[Interfaz]{Interfaz del servicio web con acceso de una cuenta.}
    				\label{fig:UI_homeUsuario}
    				\end{center}
    			\end{figure}
    	        
        	   El servicio web le permite al usuario crear una nueva cuenta en el servicio, ya que es necesario para poder vincular el certificado. La figura \ref{fig:UI_nuevoUsuarioServicioWeb} muestra la interfaz que se le presenta al usuario para la creación de una nueva cuenta en el servicio web.
        	   
        	   \begin{figure}[H]
    				\begin{center}	\includegraphics[width=12cm]{./imagenes/Disenio/Componente_1/UI_nuevoUsuarioServicioWeb.PNG}
    					\caption[Interfaz]{Interfaz del servicio web para crear nuevo usuario.}
    				\label{fig:UI_nuevoUsuarioServicioWeb}
    				\end{center}
    			\end{figure}
    			
    	        Otro caso a considerar es cuando el usuario no activa la extensión y quiere ingresar al servicio web, por lo que si el usuario realiza una petición al servicio con la extensión deshabilitada, el servicio web no encontrará ningún certificado con \textit{Chaffing} inyectado en el encabezado, por lo que el servicio web pedirá realizar un inicio de sesión ordinario, el cual le pedirá usuario y contraseña para ingresar al servicio con su cuenta. 
	        
    	        La figura \ref{fig:UI_inicioSesionComun} muestra la interfaz que se le muestra al usuario cuando se requiere un inicio de sesión ordinario.
    	        
    	        \begin{figure}[H]
    				\begin{center}	\includegraphics[width=10cm]{./imagenes/Disenio/Componente_1/UI_inicioSesionComun.PNG}
    					\caption[Interfaz]{Interfaz de inicio de sesión ordinario del servicio web.}
    				\label{fig:UI_inicioSesionComun}
    				\end{center}
    			\end{figure}
			
    			Como sabemos, el usuario puede revocar un certificado lo cual quiere decir que puede generar uno nuevo. Esto le permite al usuario poder generar un nuevo certificado si el usuario dejo su sesión iniciada en otra maquina publica. El tener la capacidad de generar otro certificado, le permite al servicio web no dar acceso si el certificado recibido se encuentra revocado, pidiendo asi iniciar sesión de nuevo con credenciales (usuario y contraseña) para vincular el nuevo certificado. 
    			La figura \ref{fig:UI_certificadoRevocado} es la interfaz que se le muestra al usuario si el certificado que el servicio web obtiene del encabezado se encuentra revocado.
    			
    			\begin{figure}[H]
    				\begin{center}	\includegraphics[width=9cm]{./imagenes/Disenio/Componente_1/UI_certificadoRevocado.jpeg}
    					\caption[Mensaje de error]{Mensaje de inicio de sesión incorrecto por revocación de certificado ó credenciales inválidas.}
    				\label{fig:UI_certificadoRevocado}
    				\end{center}
    			\end{figure}
            
            \subsection{Algoritmos.}
		        Los algoritmos necesarios para hacer el \textit{winnowing} así como la verificación de los certificados son expuestos a continuación:\\
                \begin{algorithm}[H]
                    \SetAlgoLined
                    \KwData{chaffing, patternCipher, aesCipher, privateKey}
                    \KwResult{cert,header}
                    
                    $chaffingDecode[] \longleftarrow base64.decode(chaffing)$\;
                    $aesKey \longleftarrow rsa.decipher(aesCipher, privateKey)$\;
                    $pattern[] \longleftarrow aes.decipher(patternCipher, aesKey)$\;
                    $cert[]$\;
                    $header[]$\;
                    $i \longleftarrow 0$\;
                    
                    \While{$ i < pattern.length$}{
                      \eIf{$pattern[i] == 1$}{
                            $header.add(chaffingDecode[i])$\;
                       }{
                            $cert.add(chaffingDecode[i])$\;
                       }
                       
                       $i \longleftarrow i + 1$\;
                     }
                    %%falta la chingadera de verificaaar y ese pedo pero yo creo que ese se va en otro algoritmo eeeh
                   
                    
                    \SetAlgorithmName{Algoritmo}{algoritmo}{Algoritmos}
                    \caption{Obtenci\'on de la cabecera y certificado mediante el proceso de winnowing.}
                \end{algorithm}

                 En este algoritmo se toman en consideración los siguientes datos de la petición HTTP recibida:
                    \begin{itemize}
                        \item \textbf{chaffing}: es la cadena que se encuentra en base 64 que contiene el certificado y la basura mezclada.
                        \item \textbf{aesCipher}: Se encuentra en el mismo campo del patrón, contiene la llave AES cifrada mediante el cifrado asimétrico RSA.
                        \item \textbf{patternCipher}: es el otro apartado que contiene el patrón necesario para realizar el proceso de winnowing, el cual está cifrado con el algoritmo de cifrado simétrico AES.
                        
                    \end{itemize}
                Así como la llave privada necesaria para realizar el decifrado RSA \textbf{privateKey} la cual si bien no se envía en la petición, se encuentra almacenada localmente como dato estático. El proceso de winnowing se realiza analizando el patrón para dividir cada uno de los bits del chaffing en certificado y cabecera.\\
                En cuanto a la salida de este algoritmo se hace mención de dos datos los cuales se despliegan a continuación:
                \begin{itemize}
                    \item \textbf{cert}: Contiene el certificado del usuario completamente íntegro asumiendo que el proceso fue correcto.
                    \item \textbf{header}: Contiene la información del campo \textit{Accept} de la cabecera HTTP.
                    
                \end{itemize}

               
                Por otro lado se cuenta con otro algoritmo para realizar la verificación del certificado buscando que no se incluya algún certificado no válido o que sea expedido por alguna AC externa a la desarrollada por nosotros:
                
                \begin{algorithm}[H]
                    \SetAlgoLined
                    \KwData{cert,publicKey}
                    \KwResult{certR,flag}
                    
                    \eIf{$cert!=null$ $and$ $cert.verify(publicKey) == 1$}{
                        $dataCert\longleftarrow getDataCert(cert)$\;
                        $emailUser \longleftarrow dataCert.getEmail()$\;
                        $response \longleftarrow CA.getValidation(sha256.doSha(emailUser))$\;
                        $shaCert \longleftarrow sha256.doSha(cert)$\;
                        \eIf{$response == 0 $}{
                            $certR = 0$\;
                            $flag = 0$\;
                        }{
                        \eIf{$response == shaCert$}{
                            $certR = cert$\;
                            $flag = 1$\;
                        }{
                            $certR = 0$\;
                            $flag = 2$\;
                        }
                        }
                        
                    }{
                        $certR = 0$\;
                        $flag = 0$\;
                    }
                    
                    %%falta la chingadera de verificaaar y ese pedo pero yo creo que ese se va en otro algoritmo eeeh
                   
                    
                    \SetAlgorithmName{Algoritmo}{algoritmo}{Algoritmos}
                    \caption{Algoritmo para la verificaci\'on del certificado.}
                \end{algorithm}
                
                 \subsubsection{Complejidad computacional.}
                    Haciendo un analisis del algoritmo de winnowing anteriormente mostrado, deducimos que la complejidad del algoritmo de \textit{winnowing} es:
                    $O(n)$ donde $n$ es el tamaño de carácteres del certificado.
                
                Para el caso de la verificaci\'on se cuenta con los siguientes datos de entrada:
                \begin{itemize}
                    \item \textbf{cert}: Contiene el certificado recibido en la petici\'on HTTP proveniente del anterior algoritmo.
                    \item \textbf{publicKey}: Contiene la public Key de la AC necesaria para la comprobaci\'on del certificado.
                \end{itemize}
                Tambi\'en es importante dejar en claro algunas funciones con las que se cuenta en el algoritmo como es el caso de \textbf{getDataCert(key)} que es el algoritmo que se encarga de comprobar tanto la fecha del certificado como validar que el mismo haya sido expedido por la AC correspondiente, por otro lado la funci\'on \textbf{CA.getValidation(sha256)} es una función que se encarga de comunicarse con la AC enviándole un sha del usuario, que en este caso es el email, para que busque en sus repositorios si el certificado aún se encuentra activo para el nombre de usuario dado, a lo que nos responderá \textit{0} si el usuario no existe o \textit{ShaCertificadoUsuario} si el usuario cuenta con un certificado válido para realizar una comparación con el recibido de la petición. 
                La salida de este algoritmo se compone de lo siguiente:
                \begin{itemize}
                    \item \textbf{certR}: Contiene el certificado después de validarse para que el servicio web pueda almacenarlo en su base de datos.
                    \item \textbf{flag}: Contiene una bandera de control para saber el resultado de la validación cuyos valores posibles son los siguientes:
                    \begin{itemize}
                        \item[--] \textit{0}: Se refiere a que el certificado recibido no es válido.
                        \item[--] \textit{1}: Se refiere a que el certificado recibido es válido.
                        \item[--] \textit{2}: Se refiere a que el certificado recibido es válido pero ya no se encuentra asignado al usuario en cuesti\'on, es decir, fue revocado.
                    \end{itemize}
                \end{itemize}
                Por lo que de esta manera el servicio web puede decidir qué hacer en cada uno de los casos, tomando en cuenta la recomendación de no mostrar demasiada información al usuario cuidando siempre la seguridad de los sistemas.
   %   %   %   %   %   %   %   %   %    %
   %		        Capítulo 5					%
   %   			DESARROLLO 				   %
   %                               				  	  %
   %   %   %   %   %   %   %   %   %	%             
    \chapter{\textcolor{azulescom}{Desarrollo}.}
        
        \section{Componente I. Extensión.}

            
            Como podemos ver hay dos archivos que acompañan al \textbf{manifest.json} llamados \textbf{popup.html} y \textbf{background.js} de los cuales hablaremos a continuación.\\
            \subsection{Archivo popup.html}
            Es un HTML que contiene la página que se mostrará al hacer click en el botón de la extensión, la idea de esta \textit{mini página} es que nos permita gestionar las funciones de la extensión, por tal motivo contará con 4 botones necesarios para su funcionamiento:
            \begin{itemize}
                \item \textit{btnActivar :} Botón necesario para activar o desactivar el funcionamiento de la extensión.
                \item \textit{btnIniciarSesion :} Botón en el que el usuario podrá iniciar sesión o registrarte para empezar a utilizar la extensión.
                \item \textit{btnCerrarSesion :} Botón que nos permite cerrar la sesión del usuario actual.
                \item \textit{btnAviso :} Botón que nos redirige a una página en donde podremos aceptar el certificado, así como muestra algunos datos de cómo funciona el proceso de autentificación.
            \end{itemize}
            \subsection{Archivo background.js}
            Es el archivo que contiene la mayor parte de la lógica del comportamiento de este componente, vamos a explicar alguna de las funciones más relevantes de su desarrollo\\
            Lo primero que debe de realizar la extensión es interceptar la petición antes de que esta salga a red para poder realizar todo el proceso de Chaffing.
            
            \subsubsection{Interceptar petición.}
            Primero que nada, se necesita comprobar que la extensión se encuentre habilitada, checando la bandera de \textit{btnActivar}, si 
            
            \subsubsection{Creación del patrón de Chaffing}
            Esta función nos servirá para poder generar el patrón de 
            
            \subsubsection{Generación del chaffing.}
	         Esta función nos va a generar el Chaffing resultante 
	         

        	
	\newpage
	
	    \section{Componente II. Servidor Autentificador.}
    	    En esta sección se mostrará el proceso que se llevo a cabo para la creación del componente II. La implementación de este componente se desarrollo en \textbf{NodeJS} el cual es un entorno en tiempo de ejecución multiplataforma para la capa del servidor basado en lenguaje de programación \textit{ECMAScript}. 
    	    
    	    \subsection{Manejador de paquetes de Node (npm).}
    	    npm \textit{(Node Package Manager)} es el sistema de gestión de paquetes por defecto para Node.js, un entorno de ejecución para 
            
            npm nos permite poder instalar una infinidad de librerías, para este componente haremos uso de algunas librerías como las que se describen a continuación:
        
            \begin{itemize}
                \item mongoose (versión 5.7.3): Nos permite de una manera mas rápida tener una conexión con mongodb. Sólo se necesita especificar la base de datos a la que se desea conectar y el puerto. Mongoose se encarga de establecer la conexión para así poder manipular la base de datos.
                \item express (versión 4.17.1): Es una infraestructura de aplicaciones web que proporciona un conjunto de características como son: 
                    \begin{itemize}
                        \item Escritura de manejadores de peticiones.
                        \item Integración con motores de renderización de "vistas".
                        \item Añade procesamiento de peticiones "middleware".
                    \end{itemize}
                \item morgan (versión 1.9.1): Nos permite poder ver el estatus de las peticiones http que se hacen al servidor. 
                \item nodemon (versión 1.19.3): Nos permite poder reiniciar el servidor automáticamente. Esta librería y la anterior no son necesarias en producción, mas sin embargo en desarrollo se recomienda.
                \item ejs (versión 2.7.1): Este paquete nos permite crear vistas en lenguaje \textit{ejs}. 
                \item crypto-js (versión 3.1.9): Este paquete tiene gran variedad de funciones criptográficas, desde hash hasta RSA, entre muchas otras funciones.
                \item node-openssl-cert (versión 0.0.98): Este paquete tiene gran variedad de funciones para la creación de certificados openSSL. 
                \item Por si solo NodeJS tiene librerías tales como fs, path, https, entre otras, las cuales también haremos uso.
            \end{itemize}
	        
        \subsection{Componentes.}
	        
	        En la figura \ref{fig:paqueteAutoridadCert2} se muestra los componentes que tendrá la autoridad certificadora.
	        
	        \begin{figure}[H]
        		\begin{center}
        		\caption{Estructura del desarrollo de la Autoridad Certificadora.}
	            \label{fig:paqueteAutoridadCert2}
	            \end{center}
	        \end{figure} 
            
             Los componentes son los siguientes:
             \begin{itemize}
                 \item node\_modules: Son los módulos de NodeJs donde se encuentran todas las librerías que se estarán ocupando en el proyecto.
                 \item src: En esta carpeta se almacena el código de las diferentes rutas, así como las vistas y los modelos para la base de datos.
                 \item database.js: Es complemento de keys.js donde se establece la conexión a la base de datos con los datos obtenidos en keys.js
                 \item index.js: Aquí se declaran las rutas y los \textit{middlewares} del servidor, así como el puerto y la configuración https.
                 \item keys.js: Aquí se especifica el puerto y la base de datos a la que queremos conectarnos.
                 \item Usuarios\_CRT: Esta carpeta contendrá todos los archivos .csr y .crt de los usuarios registrados
                 \item package.json: Este es el paquete del proyecto, donde se especifica la versión, componentes, etc. (ver figura \ref{cod:package})
             \end{itemize}
             
            
            Antes que nada se debe de inicializar el servidor en un 
             
            Ahora necesitaremos declarar todas las rutas a las cuales la extensión (Componente I) podrá hacer peticiones. Las 
            Una vez teniendo nuestro servidor corriendo con una conexión segura y con las rutas, debemos configurar el servidor para tener acceso a la base de datos de MongoDB, ya que necesitaremos guardar ciertos datos del usuario.
            Necesitamos especificar el púerto y nombre de la base de datos a la que queremos conectarnos. La figura \ref{cod:keys} muestra los parámetros que se necesitan para conectarse a la base de datos, mientras que la figura \ref{cod:database} muestra 
            Debemos crear nuestro modelo de la base de datos, ya que será de tipo JSON el modelo que debemos almacenar en la base de datos, por lo que se necesita especificar los atributos del modelo \textbf{usuario}. Gracias a la clase \textbf{Schema} que mongoose nos proporciona nos facilita el modelado del usuario para la base de datos. Los parámetros que tendrá nuestro esquema y nuestra b
            
            \subsubsection{Crear nuevo usuario.}
            Esta función en la autoridad certificadora permitirá al usuario poder crear una cuenta desde la extensión, para así poder obtener un certificado y hacer uso de la misma en peticiones futuras. Usando la librería \textit{node-openssl-cert} nos permite crear certificados formato x509 de OpenSSL. El código 
       
            \subsubsection{Obtener certificado.}
            Esta función le permite al usuario obtener su certificado. La autoridad certificadora se encargará de buscar que exista el usuario tanto en la base de datos como su certificado. Si existe el usuario y un certificado vinculado al mismo, su certificado se regresa como respuesta al usuario. Si no se encuentra el usuario o su certificado, se regresa como respuesta un estatus de error. El có
	         
	        \subsubsection{Revocar certificado.}
            Esta función será la encargada de revocar el certificado, esto con el fin de crear un nuevo certificado y reasignarlo al usuario quien realiza dicha revocación. Una vez terminado el proceso, el usuario necesitará hacer la petición correspondiente para obtener su certificado de nuevo, ya que el actual no tendrá v
       
            \subsubsection{Verificar certificado.}
            La verificación del certificado se hará por medio de peticiones que haga la API, la cual solicitará el hash 256 del usuario quien esta iniciando sesión en el servicio. Esta con el fin de verificar validez del certificado para saber si es valido o se ha revocado el certificado que esta llegando a través del encabezado de la petición. El código \ref{cod:verificarCertificadoCert} muestra la implementación de esta función.
            
       
       
        \section{Componente III. API.}
	        En esta sección se mostrarán los pasos que se realizaron para implementar el Componenete III. Dichos pasos constan de versiones de los softwares utilizados y configuraciones necesarias para el funcionamiento.
	        
	        \subsection{API}
	            Para la realización de la API, utilizamos \textit{Java JDK 8.0} junto con el framework \textit{Spring Framework 5.0} con el objetivo de ahorrar tiempo de desarrollo gracias a las facilidades que nos brinda \textit{Spring}.
	            
	            \subsubsection{Creación.}
	                Utilizando una herramienta web llamada \textit{\href{https://start.spring.io/}{Spring initializr}} creamos una aplicación con los siguientes parámetros de configuración.
	                \begin{itemize}
	                    \item Proyect: \textbf{Maven Proyect}
	                    \item Language: \textbf{Java}
	                    \item Spring Boot: \textbf{2.2.0}
	                    \item Proyect Metadata:
	                    \begin{itemize}
	                        \item Group: \textbf{TT2018B003.comp3}
	                        \item Artifact: \textbf{API}
	                        \item Options:pretende
	                        \begin{itemize}
	                            \item Name: \textbf{API}
	                            \item Description: \textbf{API for winnowing}
	                            \item Packaging: \textbf{JAR}
	                            \item Java: \textbf{8} 
	                        \end{itemize}
	                    \end{itemize}
	                    \item Dependencies:
	                    \begin{itemize}
	                        \item \textbf{Spring Boot Actuator}
	                        \item \textbf{Spring Web}
	                        \item \textbf{Spring Boot DevTools}
	                        \item \textbf{Lombok}
	                    \end{itemize}
	                \end{itemize}
	        
	            
	                Una vez descargado el proyecto desde la página web, utilizamos \textit{Spring Tools Suite}, el cual es un IDE desarrollado por Spring basado en Eclipse. Este IDE, nos brinda la posibilidad de poder cargar el proyecto como \textit{Proyecto Maven Existente} para comenzar a implementar la solución.
	                
	           \subsubsection{Implementación de la solución.}
	                Basándonos en los diagramas de la sección de Diseño para el Componente III, creamos los paquetes y clases necesarios para la solución. En la Figura \ref{fig:estructuraAPI} se muestra la organización del proyecto: 
	                
	                \begin{figure}[H]
                		\begin{center}
                		\caption{Estructura del desarrollo de la API.}
        	            \label{fig:estructuraAPI}
        	            \end{center}
        	        \end{figure} 
        	        
        	        \paragraph{ApiApplication.java} es una clase creada automáticamente por Spring initializr. Dicha clase se encarga de correr todo lo necesario para la ejecución de la aplicación, es decir, creación de beans de Spring y su configuración.
        	        
        	        \paragraph{APIController.java} es una clase creada para este trabajo la cual contiene un método llamado \textit{makeWinnowing}. Este método realiza lo necesario para devolver el certificado al Servicio Web. 
        	        
        	        Gracias a Spring, la invocación de este método se hace mediante un @RequestMapping y se leen los datos con @RequestBody, por lo que desde el servicio web sólo es necesario mandar un modelo de datos llamado WinnowingModel a la dirección especificada con el modelo de datos como el cuerpo de la petición HTTP. 
        	        
        	        El código \ref{cod:APIController} muestra el mapeo y el método implementado.
        	        
        	        
        	        
        	        \paragraph{El paquete TT2018B003.comp3.API.Utils} contiene 4 clases. En las clases \textit{CipherUtilityAES.java}, \textit{CipherUtilityRSA.java} y \textit{CipherUtilitySHA256} se utilizó \textit{Java Security}. Java Security proporciona las clases e interfaces para el framework de seguridad. Esto incluye clases que implementan una arquitectura de seguridad de control de acceso de fácilmente configurable. Este paquete también admite la generación y el almacenamiento de pares de claves públicas criptográficas, así como una serie de operaciones criptográficas exportables. Finalmente, este paquete proporciona clases que admiten objetos firmados o protegidos y generación segura de números aleatorios \cite{refJavaSecurity}.
        	        
        	        La clase \textit{Base64u.java} es una implementación de la decodificación de Base64 realizada por nosotros para este Proyecto.
        	        
        	        Finalmente, la clase \textit{SignatureUtility} es una clase creada para poder recibir un certificado formato X509 y poder determinar si la firma es válida o no, utilizando \textit{X509Certificate.java}.
        	        
        	        \paragraph{El paquete TT2018B003.comp3.API.Service} contiene 2 servicios y una interfaz. La clase \textit{WinnowingService.java} implementa la interfaz \textit{IWinnowing}. Dicha clase hace uso a su vez de la clase \textit{CryptoService.java} encargada de realizar el decifrado AES y RSA y el cifrado SHA256, apoyándose de las clases del paquete Utils.
        	        
        	        El código \ref{cod:MakeWinnowing} muestra el contenido de esta clase para realizar el winnowing.
        	        
        	        
	        \subsection{Servicio Web.}
	            Para este trabajo terminal, utilizamos un sitio web de prueba desarrollado con \textit{Spring Framework 5}, \textit{Java 8} y montado sobre un servidor \textit{Pivotal Server Developer Edition v4.0}.
	            
	            Para este proyecto partimos bajo la premisa de que el servicio web está desarrollado, por lo que sólo se mostrarán los cambios necesarios para la integración de la API.
	            
	            \subsubsection{Modificaciones al inicio de sesión.}
	                Para la modificación del servicio web, se ubicó el método encargado de desplegar la vista para el inicio de sesión, es decir, el formulario. Dicho método se halló en la clase LoginController.java. El código \ref{cod:LoginOriginal}, muestra el contenido original en el cual se aprecia que en cuando solicita el recurso \textit{'/login'} retorna una vista llamada \textit{login}, la cual es un formulario.
	                
	                
	                En el código \ref{cod:LoginModificado} se muestran todos lo cambios realizados en el método \textit{login()} para implementar el uso de este método de autentificación propuesto. Entre los elementos más destacables se encuentra la implementación de la anotación \textit{@RequestHeader} perteneciente a \textit{Spring Framework}. Esta anotación es capaz de obtener el header con el nombre especificado de la petición HTTP que acaba de recibir el servidor, es por ello que implementamos su uso para obtener el encabezado \textit{Chaffing} y \textit{Pattern} para poder realizar la etapa de \textit{Winnowing}. Además, se hizo uso de \textit{RestTemplate} para poder realizar la conexión con la API y poder obtener el certificado y la lógica de negocio correspondiente.
	                
	                
	                Por otro lado, también fue necesario modificar aquel método encargado de procesar el inicio de sesión por formulario. Este método se encuentra en la misma clase que el método login mostrado anteriormente, dicha clase es \textit{LoginController.java}. 
	                
	                El código \ref{cod:loginByFormOriginal} muestra el método \textit{loginByForm} original del servicio web, donde se aprecia que mapea la dirección \textit{'/loginByForm'} y recibe los parámetros del formulario.
	                
	                
	                En el código \ref{cod:loginByFormModificado} se muestran las modificaciones que se realizaron al método para poder implementar este método de autentificación propuesto. Ente los cambios más significativos se encuentra la operación \textit{saveCertificate} la cual se encarga de guardar el certificado para el usuario correspondiente en la base de datos.
	                
	
	                Después de implementar estas modificaciones al código fuente del servicio web, éste puede hacer uso del método que proponemos para iniciar sesión automáticamente de manera segura y rápida.
	%   %   %   %   %   %   %   %   %   %
	%		        Capítulo 6					%
	%   			PRUEBAS 					%
	%                               				  %
	%   %   %   %   %   %   %   %   %	%
	\chapter{\textcolor{azulescom}{Pruebas.}}
	
	    Las pruebas son un factor importante para comprobar la eficiencia y correcto funcionamiento de nuestro sistema. Las pruebas que aquí se presentan fueron realizadas con los equipos de la sección de Hardware en Herramientas a Usar, sobre una red local creada con un celular Huawei P20, cuya velocidad de conexión es de 76Mb/s, con una intensidad de señal catalogada como 'Excelente', es decir, los equipos estaban dentro de un radio de 3m del punto de conexión (Huawei P20).
        
	    \section{Pruebas de integración.}
    	    Estas pruebas nos sirven para comprobar que los componentes de nuestro sistema estén interactuando de forma correcta entre ellos. Como sabemos, nuestro sistema en general se compone principalmente de 3 componentes, para este primer prototipo existe una comunicación entre el primer componente (Extensión) y el segundo componente (Servidor Autentificador), además de una comunicación entre el primer componente y el tercero (API).\\
    	    
    	    \subsection{Extensión y Servidor autentificador.}
    	    Vamos a empezar con la interacción entre el cliente y el servidor autentificador, uno de los casos en el que estos dos componentes interactuan es cuando el usuario inicia sesión en la extensión, aquí el Servidor Autentificador verifica si el usuario está registrado y si este es el caso, entonces le debe de devolver el certificado junto con un código diciendo que el usuario está registrado dentro del servidor.
    	    
    	    \begin{figure}[H]
    	        \centering
    	        \caption{Sesión iniciada}
    	        \label{fig:sesionIniciada}
    	    \end{figure}
	    
    	    Si aparecen estos mensajes, quiere decir que la extensión se pudo comunicar correctamente con el servidor autentificador, y le envió correctamente el certificado correspondiente a este usuario, que se crea cuando se registra en el servidor. 
    	    
    	    \subsection{Extensión y Servicio Web.}
    	    Ahora vamos a realizar las pruebas entre el componente de la extensión con el componente del Servicio Web, una vez que el usuario inicia sesión y tenga su certificado listo, vamos a ingresar al servicio web de prueba, con ello vamos a comprobar que la comunicación entre estos componentes esté funcionando. Para realizar estas pruebas funcionales vamos a utilizar un sniffer e interceptar la información cuando estas dos se comunican: 
    	    
    	    \begin{figure}[H]
    	        \centering
    	        \caption{Resultados de la API al recibir petición de la extensión.}
    	        \label{fig:outAPI}
    	    \end{figure}
    	    Aquí se muestra una salida de los datos con los que interactua la API del Servicio Web con la extensión, la respuesta en la que la API si encuentra al usuario registrado, podemos ver que entre la información se encuentra que el certificado recibido es un certificado válido. 
    	    
    	    \subsection{Servidor Autentificador y API.}
    	    Cuando el usuario intenta iniciar una sesión en el servicio web, entonces este se debe de comunicar con el Servidor Autentificador para corroborar que el certificado que le ha llegado, vamos a utilizar un sniffer para comprobar que la información que le esté llegando a uno de los dos servidores sea el correcto: 
    	    
    	    \begin{figure}[H]
    	        \centering
    	        \caption{Resultados de la comunicación entre la API y el Servidor Autentificador.}
    	        \label{fig:com2comp3}
    	    \end{figure}
	    
        \section{Tiempos de ejecución.}
	     
	        En esta sección mostraremos el tiempo de ejecución de los algoritmos desarrollados para el Prototipo 1. Todas las pruebas fueron realizadas en igualdad de condiciones, es decir, la computadora recién prendida (sin ningún programa abierto) y sólo ejecutando el algoritmo.
	        
	        \subsection{Algoritmo de Chaffing.}
	        
	            Se realizó una prueba al algoritmo de \textit{Chaffing}, el cual incluye: creación de patrón, proceso de chaffing y cifrado de patrón. Para esta prueba se utilizó una función de JavaScript llamada \textit{performance.now()}, la cual mide el tiempo con una precisión de milisegundos.\\
	            
	            \paragraph{Tiempo de ejecución: }  434.2925ms.\\
	            
	        \subsection{Algoritmo de Winnowing.}
	        
	            Se realizó una pruebaa al algoritmo de \textit{Winnowing}, el cual incluye: descifrado del patrón y proceso de winnowing. para esta prueba se utilizó una función de Java llamada \textit{System.currentTimeMillis()}, la cual mide el tiempo con una precisión de milisegundos.\\
	            
	            \paragraph{Tiempo de ejecución: } 53.99 ms.\\
	            
	        \subsection{Inicio de sesión.}
	    
	            Se realizaron un total de 100 pruebas del inicio de sesión por medio de este Trabajo Terminal. Este proceso incluye el tiempo que transcurre desde que el usuario da click en el botón de iniciar sesión en el Servicio Web, es decir, la interceptación de la petición por parte del Componente I) hasta la impresión de la respuesta del Servicio Web, es decir, el inicio de sesión completado exitosamente mostrando la pantalla de inicio del Servicio Web.
	            Para medir el tiempo, se utilizó la función de JavaScript llamada \textit{performance.now()}, al igual que en la medición del algoritmo de Chaffing, puesto que esta medición se realiza desde el Componente I. Extensión.\\
	            El tiempo que se muestra a continuación es el promedio de 100 inicios de sesión.
	            
	            \paragraph{Tiempo promedio de ejecución: } 1101.68 ms.\\
	            \end{comment}
% ESTO SI VAAAAAAAAAAAAAAAA EN TT1

\begin{comment}
%   %   %   %   %   %   %   %   %	%	%
%		       											%
%   				Iteración 3 					  %
%                               				  		%
%   %   %   %   %   %   %   %   %	%   %     
\chapter{\textcolor{azulescom}{Iteración 3.}}
\section{Herramientas} % Herramientas para generar el modelo
\section{Modelo} % Explicar las partes del modelo con sus diagramas
\section{Implementación} % Explicar como se implementó con código o pasos para obtener el modelo en AWS
\section{Resultados} % Explicar los Resultados
%   %   %   %   %   %   %   %   %	%	%
%		       											%
%   				Iteración 4 					  %
%                               				  		%
%   %   %   %   %   %   %   %   %	%   %     
\chapter{\textcolor{azulescom}{Iteración 4.}}
\section{Integración} % Explicar cómo se configuró el modelo en la nube 
\section{Herramientas} % Explicar las herramientas más a fondo
\section{Implementación} % Explicar como se implementó con código o pasos para obtener el modelo en AWS
\section{Resultados} % Explicar los Resultados

	    \begin{comment}
contenidos...
%   %   %   %   %   %   %   %   %   %
%		         								  %
%   	TRABAJO TERMINAL 2 		 %
%                               				  %
%   %   %   %   %   %   %   %   %	%
\textcolor{guindapoli}{\part{Trabajo Terminal II}}

%   %   %   %   %   %   %   %   %   %
%		        								  %
%   			SPRINT 2 					%
%                               				  %
%   %   %   %   %   %   %   %   %	%            

\part{Sprint 2.}
%   %   %   %   %   %   %   %   %   %
%		        Capítulo 7					%
%   			ANÁLISIS 					%
%                               				  %
%   %   %   %   %   %   %   %   %	%
\chapter{\textcolor{azulescom}{Análisis.}}

\section{Arquitectura del sistema.}
\begin{figure}[H]
\begin{center}
\caption{Arquitectura General del Sistema}
\end{center}
\end{figure}    
\subsection{Descripci\'on de la arquitectura del sistema.}
El sistema seguirá manteniendo la misma arquitectura del prototipo 1, debido a que los cambios para el prototipo 2 son cambios internos que no afectan el flujo de la información.

Los cambios por componentes son los siguientes:
\begin{enumerate}
\item \textbf{Navegador Chrome con la Extensi\'on instalada}: En este bloque la modificación que se lleva a cabo es en el proceso de \textit{Chaffing} el cual ahora se inyectara \textit{Chaffing} ó bien basura en el certificado. Recordemos que en el prototipo 1 se inyectaba el certificado en el apartado de Accept el cual ocupaba como \textit{Chaffing} el valor del header Accept.

%Explicar aqui el cambio que se hizo

\item \textbf{Servidor autentificador:} Este componente tendrá pocas modificaciones en el aspecto de seguridad. Ya que en la base de datos del componente se guarda la ruta donde esta el certificado de cada usuario, por lo que, se necesitará tener un control de accesos para evitar cualquier robo de certificados. Éste control de accesos por FTP, sólo le dará acceso a la autoridad certificadora, para que la misma pueda obtener los certificados de los usuarios y mandárselos cuando lo necesiten.

\item \textbf{Servidor web con API instalada:} Las modificaciones para este componente son en el desarrollo del \textit{Winnowing}, debido a que el componente 1 tiene cambios en el \textit{Chaffing} por lo tanto, el proceso de Winnowing debe ser modificado también.
\end{enumerate}

\section{Componente I. Extensión.}
\subsection{Descripción.}

Como sabemos, el componente 1 será el encargado de interceptar las peticiones de los usuarios para inyectar el \textit{Chaffing} y enviar la petición modificada.\\
La modificación que tendrá el componente 1 es en el aspecto de seguridad, el cual se modificará en el patrón del \textit{Chaffing} para tener un patrón repetido, y asi tener menos caracteres en el mismo y por lo tanto, tener menos caracteres en el patrón. \\
En el Prototipo 1 por tener una gran cantidad de caracteres en el \textit{Chaffing} se necesitaba cifrar con AES y después la llave cifrarla con RSA por lo que representaba un gasto computacional extra que podría ser eliminado. Para este prototipo se pretende reducir el patrón en tamaño de tal manera que se pueda repetir siguiendo el comportamiento de una \textit{lista circular enlazada}.\\

\subsection{Cambios planteados.}
\begin{itemize}
\item \textbf{Eliminación del atributo usuario: } En este prototipo ya no se contará con un nombre de usuario ya que con el email y la contraseña del mismo eran suficientes para que el componente funcionara correctamente.
\item \textbf{Chaffing sin caracteres inválidos: }Debido a la nueva implementación del Chaffing se consideró desde un inicio generar solamente caracteres imprimibles por lo que la aplicación del \textit{Base 64} ya no será necesaria. 
\item \textbf{Reducción del tamaño del patrón: }  Uno de los problemas que pudimos detectar en el primer componente fue que el patrón tenía la misma longitud del chaffing lo que representaba un envío de información un tanto grande a lo que planteamos modificar el algoritmo de generación de patrón así como el de generación del chaffing permitiendo reducir la cantidad de datos enviados manteniendo la confidencialidad que en un principio se planteó.
\item \textbf{Eliminación del cifrado AES: } Como consecuencia del punto anterior se considerará dejar de un tamaño definido la longitud del patrón, logrando de esta manera que solo sea necesario cifrarlo mediante RSA sin la necesidad de AES con lo que se busca reducir el costo computacional tanto en este componente como en los que dependan del mismo. 
\end{itemize}

\subsection{Cambios al estudio de requerimientos y reglas de negocio.}

Para este componente los cambios son un tanto mínimos los cuales se enlistan a continuación:
\paragraph{Requerimientos funcionales} En cuanto a los requerimientos funcionales que se modificaron nos encontramos con el requerimiento funcional \textbf{CI RF13. Registro en servidor autentificador.} de la sección \ref{CI_RF13} en el cual se hace mención de un nombre de usuario que para este prototipo no es necesario por lo que los datos a ingresar serán: email y contraseña.

\paragraph{Requerimientos no funcionales} Debido a que en este prototipo ya se consideran enviar caracteres válidos en las peticiones HTTP, el requerimiento no funcional  \textbf{CI RNF4. Codificación en base64 del Chaffing. } de la sección \ref{CI_RNF4} ya no será necesario.

\paragraph{Reglas del negocio} Como consecuencia de la eliminación del atributo \textit{nombre de usuario} igualmente ya no son necesarias las siguientes reglas del negocio: \textbf{CI RN7. Longitud del campo de usuario.} y \textbf{CI RN9. Caracteres permitidos en campo usuario.} de la sección \ref{CI_RN7}. \\

El resto de los requerimientos y reglas del negocio quedan de la misma manera.


\section{Componente II: Servidor autentificador.}
\subsection{Descripción.}

Este componente es el encargado de guardar los certificados del usuario, para mandarlos a sus respectivos usuarios (siempre que los usuarios lo necesiten). Por lo que es importante que los certificados por ningún motivo estén vulnerables. Dando así lugar a la creación de un control de accesos por FTP para que la autoridad certificadora sólo tenga acceso a estos archivos (Certificados) de los usuarios. Este control de accesos evitará que atacantes quienes quieran obtener el certificado de los usuarios, estos no podrán debido al control de accesos. Dando mayor seguridad a los usuarios que hagan uso de la extensión (Componente I).

\subsection{Cambios Planteados.}
\begin{itemize}
\item \textbf{Control de accesos por FTP: } En este prototipo se creara un servidor FTP la cual solamente la autoridad certificadora tendrá acceso, debido a que en este servidor la autoridad almacenará los certificados de los usuarios. Donde, en la base de datos la autoridad certificadora solamente tendrá almacenados los datos del usuario como lo son: 
\begin{itemize}
\item Correo electrónico (email)
\item Contraseña
\end{itemize}
\end{itemize}


\subsection{Cambios al estudio de requerimientos y reglas de negocio.}

Los requerimientos funcionales de este componente para este prototipo deberán cambiar en el aspecto de la creación de certificados. En el prototipo 1 los requerimientos funcionales hacían alusión en crear los certificados por cada usuario dentro del servidor (Autoridad certificadora). \\
Puesto que esto deja una vulnerabilidad en la autoridad certificadora, todos los requerimientos funcionales deberán: 
\begin{itemize}
\item Generar
\item Guardar
\item Revocar
\item Actualizar
\item Comprobar
\end{itemize}
Todo lo anterior, desde los certificados almacenados en el servidor FTP y ya no desde los certificados almacenados en la autoridad certificadora.
Como requerimiento no funcional se agrega:\\

\setlength{\parindent}{0cm}

\textbf{CII\_RNF5. Servidor de acceso.} la autoridad certificadora deberá tener acceso y privilegios en el servidor FTP donde se almacenarán los certificados.


\section{Componente III: API.}
\subsection{Descripción.}

Para este prototipo, hemos planteado la modificación del proceso de \textit{Winnowing} que realiza la API, debido a que el patrón y por tanto el proceso de \textit{Chaffing} en el Componente I han cambiado. Además, como una mejora en la seguridad, nos hemos planteado que el Servicio Web no tenga almacenado en su base de datos el certificado del usuario. De esta manera la única entidad que almacenará los certificados de los usuarios será el Componente II.

El resto de la funcionalidad se mantendrá tal y como estaba en el Prototipo 1. 

\subsection{Cambios planteados.}

\begin{itemize}
\item \textbf{Nuevo proceso de Winnowing: } debido a los cambios realizados para el Prototipo 2 al Componente I, es necesario crear otro algoritmo capaz de obtener el certificado del header de la petición HTTP que el usuario mando. Este algoritmo se adaptará al nuevo patrón teniendo en cuenta que su tamaño ahora es menor y buscando el óptimo desempeño. En la sección de diseño se puede apreciar las cambios al algoritmo.
\item \textbf{Eliminación del descifrado AES: } el patrón ya no se descifrará con AES puesto que en el Componente I se elimino ese cifrado para mejorar la velocidad del algoritmo, quedando sólo el cifrado RSA.
\item \textbf{Guardado del certificado: } buscando evitar que el servicio web guarde el certificado en su base de datos, hemos planteado que la API, una vez realizado el proceso de winnowing y la validación del certificado regrese un código SHA256 del mismo. De esta manera, el servicio web sólo guardará en su base de datos un código el cual, gracias a la función SHA, no puede ser revertido en caso de que un atacante quisiera obtener el certificado. Una vez realizado los cambios, el certificado sólo se almacenará en el Componente II.
\end{itemize}

\subsection{Cambios al estudio de requerimientos y reglas de negocio.}
El requerimiento funcional \textbf{CIII\_RF10. Retorno del certificado} de la sección \ref{CIII_RF10}, explica que la API deberá retornar el certificado y su estatus al servicio web. Este requerimiento corregido queda de la siguiente manera.\\

{
\setlength{\parindent}{0cm}

\textbf{CIII\_RF10. Retorno de certificado.} La API deberá retornar el código SHA256 del certificado y su estatus al servicio para que este pueda realizar la lógica del negocio necesaria para el inicio de sesión.\\
\label{P2_CIII_RF10}
}

Finalmente, el requerimiento funcional \textbf{CIII\_RF11. Descifrado AES al patrón} de la sección \ref{CIII_RF11} se eliminará como consecuencia del cambios planteado en la sección anterior.\\ 

El resto de los requerimientos y reglas del negocio quedan de la misma manera.
\end{comment}
%   %   %   %   %   %   %   %   %   %
%		        Capítulo 8					%
%   			DISEÑO 					    %
%                               				  %
%   %   %   %   %   %   %   %   %	%       
\begin{comment}         
\chapter{\textcolor{azulescom}{Diseño.}}

\section{Componente I: Extensión.}

\subsection{Cambios en el algoritmo.}
Los cambios planteados son muy parecidos entre si ya que uno es consecuencia del otro por lo que la parte que se modificó recae completamente en el algoritmo de este componente, resultando como se muestra a continuación:\\
Algoritmo utilizado para la generación del patrón de chaffing:\\
\begin{algorithm}[H] % Algoritmo para obtener el patrón de Chaffing_v2
\SetAlgoLined
\KwData{low, high}
\KwResult{pattern,ones}
$lenPattern \longleftarrow 150*8 $\;
$ones \longleftarrow rand(low,high) $\;
$pattern[lenPattern] \longleftarrow 0$\;
$i \longleftarrow 0$\;
\While{$ i < ones$}{
$x \longleftarrow securerand(size)$\;
\If{$pattern[x] != 1$}{
$pattern[x] \longleftarrow 1$\;
$i \longleftarrow i + 1$\;
}


}
\SetAlgorithmName{Algoritmo}{algoritmo}{Algoritmos}
\caption{getPattern: Generación de patrón de chaffing del Prototipo 2}
\end{algorithm}
Consideramos este algoritmo muy simple ya que es bastante intuitivo analizar su comportamiendo, el algoritmo inicia definiendo la longitud del patrón para luego calcular de manera aleatoria la cantidad de unos que contendrá, esto en un rango definido por nosotros que más adelante se explicará, se inicializa el patrón con ceros para mas tarde iniciar con un ciclo que con base en posiciones aleatorias obtendrá una posición el la cual se colocará un 1 terminando en el momento que todos los unos calculados se encuentren en el patrón.\\

Algoritmo utilizado para la generación del Chaffing:\\
\begin{algorithm}[H]
\SetAlgoLined
\KwData{Cert}
\KwResult{Chaffing,pattern}
$lenCert \longleftarrow lenght(Cert)$\;
$pattern,ones \longleftarrow getPattern(550, 650)$\;
$Chaffing \longleftarrow " "$\;
%Contadores para agarrar el siguiente byte del cert o Accept.
$rep \longleftarrow roof(lenCert/ones)$\;
%Para cada elemento de como resultó el patrón.
$i \longleftarrow 0$\;
$k \longleftarrow 0$\;
\While{$ i < rep$}{
\ForEach{$j$ in $pattern$}{
\eIf{$j == '1'$}{
$Chaffing \longleftarrow Chaffing + Cert[k]$\;
$k \longleftarrow k + 1$\;
$ones \longleftarrow ones - 1$\;
}{%Else, es un 0, entonces se coloca el siguiente byte del Accept.
$Chaffing \longleftarrow Chaffing + fake()$\;
}
\If{$ones == 0$}{
$break$\;
}
}
$i \longleftarrow i + 1$\;
}
\SetAlgorithmName{Algoritmo}{algoritmo}{Algoritmos}
\caption{getChaff: Generación de chaffing del prototipo 2}
\end{algorithm}
En este caso el algoritmo que genera gran parte de los datos que se incluirán en la cabecera \textit{HTTP} inicia calculando la longitud del certificado original para luego mandar a llamar a la función \textit{getPattern(int, int)} que calcula el patrón como se explicó previamente; esta función recibe dos enteros como entrada los cuales estáticamente definimos como 150 bytes necesarios para determinar el números de unos con los que contará el patrón, esto debido a que la longitud máxima de RSA es de 254 bytes, sin embargo requiere de ciertos bytes extras que utiliza para poder realizar el cifrado y descifrado de manera correcta, por lo que a nuestro sistema se ajusto a esta cantidad de 150 bytes.

\subsubsection{Complejidad computacional.}
Haciendo un analisis de los algoritmos anteriormente mostrados, deducimos que la complejidad del algoritmo de \textit{chaffing} del prototipo 2 es:
$O(n)$ donde $n$ es el


\section{Componente II: Servidor Autentificador.}
\subsection{Cambios realizados.}
Debido a las reglas del negocio que se tuvieron que ajustar, también se realizaron unos cambios en alguno de los diagramas, a continuación se mostraran los cambios realizados. 

\subsection{Diagrama de Actividades.}
Para el Diagrama de Actividades, no se realizaron cambios drásticos en la lógica de su diseño, sin embargo se agregó el servidor FTP y con esto tener un control de acceso para que solo puedan acceder usuarios permitidos, y con ello este diagrama se ajustó a como sigue:
\begin{figure}[H]
\centering                    
\caption{Ajustes al Diagrama de Actividades del Componente II.}
\label{fig:CII_DA_PII}
\end{figure}

\subsection{Diagrama de Secuencia.}
\subsubsection{Diagrama de secuencia: Crear nuevo usuario.}
Se ajusto para que los certificados se almacenaran en el servidor de acceso FTP.
\begin{figure}[H]
\centering              
\caption{Ajustes al Diagrama de Secuencia crear nuevo usuario del componente II.}
\label{fig:CII_SD_UC1_PII}
\end{figure}

\subsubsection{Diagrama de secuencia: Revocar certificado.}
Se realizaron cambios para que cuando se revoque el certificado, todas las validaciones ahora se contemplen para acceder en el servidor de acceso FTP.
\begin{figure}[H]
\centering                        
\caption{Ajustes al Diagrama de Secuencia revocar certificado del componente II.}
\label{fig:CII_SD_UC2_PII}
\end{figure}

\subsubsection{Diagrama de secuencia: Obtener certificado.}
Al momento de que la extensión requiera del certificado, este componente tendrá que acceder al servidor de acceso FTP para obtener y devolverle el certificado.
\begin{figure}[H]
\centering                        
\caption{Ajustes al Diagrama de Secuencia obtener certificado del componente II.}
\label{fig:CII_SD_UC3_PII}
\end{figure}

\subsubsection{Diagrama de secuencia: Verificar certificado.}
Nuevamente se realizaron cambios para ajustar la procedencia del certificado para que se obtenga del servidor de acceso FTP al momento en que la API requiera buscar a un usuario en el Servidor Autentificador. Autentificador.
\begin{figure}[H]
\centering                        
\caption{Ajustes al Diagrama de Secuencia verificar certificado del componente II.}
\label{fig:CII_SD_UC4_PII}
\end{figure}


\section{Componente III: API.}
\subsection{Cambios realizados.}
Para estos cambios se eliminó la lógica en la que el mismo Servicio Web guarda los certificados con los que le da acceso a los usuarios en una base de datos, en cambio se sustituyó con una comunicación con el componente dos, Servidor Autentificador, para que le pregunte únicamente si el usuario se encuentra registrado, y le devuelva un hash del certificado para compararlo con el que recibió por parte de la Extensión, esto con el fin de saber si existe el usuario y su certificado está actualizado o si requiere de actualizarlo por alguna revocación.

\subsection{Cambio en el algoritmo.}
Debido a que se realizaron cambios en la etapa de \textit{Chaffing} del componente 1, es necesario ajustar igualmente la forma en la que se realiza el Winnowing para este componente, ahora el algoritmo que realiza la obtención del patrón y la etapa de \textit{Winnowing} se explica a continuación.

\begin{algorithm}[H]
\SetAlgoLined
\KwData{chaffing, patternCipher, privateKey, sizeCert}
\KwResult{cert,header}

$pattern[] \longleftarrow rsa.decipher(patternCipher, privateKey)$\;
$cert \longleftarrow " "$\;
$rep \longleftarrow sizeCert / numOnes(pattern)$\;
% contadores
$contRep \longleftarrow 0$\; 
\While{$i < rep$}{
\While{$ j < pattern.length$}{
%Obteniendo el certificado
\If{$cert.length == sizeCert$}{
$break$\;
}
\If{$patt[j] == 1$}{
$cert +=     chaffing[i+(pattern.length * contRep)]$\;
}
$j \longleftarrow j + 1$\;
}
$i \longleftarrow i + 1$\;
}
\SetAlgorithmName{Algoritmo}{algoritmo}{Algoritmos}
\caption{Cambios en el algoritmo de Winnowing para obtener el certificado.}
\end{algorithm}

Para este prototipo, ahora vamos a tener un el patrón de chaffing cifrado únicamente mediante RSA, y este patrón será de un tamaño aproximado de 150 bytes, para realizar este algoritmo se requiere del chaffing con el que se va a trabajar, el patrón cifrado con RSA, la llave privada para descifrar el patrón y el tamaño en bytes del certificado.\\

Primero se realiza el descifrado de este patrón mediante RSA, una vez que tenemos descifrado el patrón, vamos a proceder a realizar la etapa de Winnowing. Necesitamos saber cuántas veces necesitamos recorrer este patrón, por lo cual se obtiene en una variable dividiendo el tamaño del certificado, mencionado anteriormente, y el número de unos que contiene el patrón, ahora recorreremos el patrón en busca de un uno, cuando lo encontramos quiere decir que este es un caracter válido para tomar como caracter del certificado, por lo que lo agregamos a nuestra variable del certificado, y repetiremos esto tantas veces como el resultado de la división anteriormente mencionada.\\

Al finalizar esta etapa de \textit{Winnowing}, nos quedamos con el certificado que fue enviado en la petición, así podremos proseguir con la lógica de nuestro sistema, que es comunicarnos con el Servidor Autentificador, para verificar con este mismo si el certificado es uno válido para nuestro inicio de sesión.

\subsubsection{Complejidad computacional.}
Haciendo un analisis del algoritmo anteriormente mostrado, deducimos que la complejidad del algoritmo de \textit{winnowing} del prototipo 2 es:
$O(n)$

\subsection{Diagrama de Flujo.}
\begin{figure}[H]
\centering               
\caption{Ajustes al Diagrama de Flujo Componente III.}
\label{fig:CII_FD_PII}
\end{figure}

Se realizaron unos cambios para este diagrama, se agregó un flujo en el cuál se muestra la forma en el que la API se comunica con el Servidor Autentificador para saber la validez del certificado, primero le envía el Hash del usuario que se puede obtener con el certificado, el Servidor Autentificador lo busca y le envía el hash de este certificado, después la API transforma el primer certificado que recibió mediante la petición y lo compara con el Hash que recibe del Servidor Autentificador, si son iguales entonces podemos dar acceso al usuario que envío esta petición, sin embargo si la comparación resulta diferente, entonces quiere decir que el usuario no tiene un certificado válido o actualizado, por lo que debe de volver a iniciar sesión en el Servicio Web para que pueda reasignarle correctamente su certificado válido a esta cuenta.
\end{comment}
%   %   %   %   %   %   %   %   %   %
%		        Capítulo 9					%
%   			DESARROLLO 			   %
%                               				  %
%   %   %   %   %   %   %   %   %	%
\begin{comment}	    
\chapter{\textcolor{azulescom}{Desarrollo.}}

\section{Componente I: Extensión.}
Los cambios planteados y diseñados en las secciones Análisis y Diseño del Prototipo 2 tuvieron consecuencias en el desarrollo del Componente I. En esta sección se mostrarán los cambios hechos para la implementación del Prototipo 2.

\subsection{Cambios en el manifest.}
Para este segundo prototipo hemos hecho cambios en el archivo manifest de la extensión. Como se puede notar en el código \ref{cod:manifest_P2} la versión de la extensión cambios de 1.0 a 2.0, con el objetivo de tener un control de versiones y se removió el uso de la librería \textit{CryptoJS} que usabamos para cifrar con AES, ya que en este prototipo removimos ese método de cifrado.


\subsection{Implementación del Chaffing.}
Los cambios planteados en este prototipo, repercuten por sobre todas las cosas en la implementación del algoritmo de \textit{Chaffing}. El código \ref{cod:chaffingProcess_P2} muestra el proceso de chaffing, invocando a las funciones \textit{getPattern} del código \ref{cod:getPattern_P2} para obtener el patrón del chaffing y \textit{makeChaffing} el código \ref{cod:makeChaffing_P2} para realizar el chaffing. Además en la línea 11 del código \ref{cod:chaffingProcess_P2} se muestra que ahora sólo se cifra con RSA el patrón de chaffing.


Con el código mostrado anteriormente, ahora el Componente I es capaz de cumplir con los cambios propuestos y las ventajas que ellos lograron.

\section{Componente II: Servidor Autentificador.}

Los cambios planteados en Análisis y Diseño del prototipo 2, tuvieron como consecuencia modificaciones en las diferentes funciones del componente 2. En esta sección se mostrarán los cambios realizados para tener la funcionalidad requerida del prototipo 2.

\subsection{Servidor de accesos FTP.}
Debido a que en el prototipo 1, la autoridad certificadora (Componente 2) queda vulnerable a posibles ataques para la obtención de certificados guardados de los usuarios. Este prototipo proporciona a la autoridad un servidor de accesos por FTP, en el que éste le dará únicamente acceso a la autoridad certificadora, para que así ningún otro usuario pueda obtener los certificados guardados. La autoridad certificadora, sólo necesitará tener almacenado el correo email y contraseña del usuario en su base de datos, por lo que la misma no necesita estar protegida ante posibles ataques.
El paquete \textbf{ftp} para NodeJs nos ayudará a llevar a cabo esta conexión.

\begin{itemize}
\item ftp (versión 0.3.10): Este paquete nos ayuda a establecer conexión a un servidor ftp con funciones preestablecidas de la librería, por lo que en NodeJs solo bastará con llamar a las funciones put o get para obtener el archivo o guardar un archivo según sea el caso. 
\end{itemize}

El código \ref{cod:conexionFTPget} muestra la implementación necesaria para poder conectar al servidor FTP y obtener certificados del mismo. En este caso el usuario es \textbf{diegoarturomg} y contraseña \textbf{211096} para acceder al servidor FTP.\\

Esta implementación se llevará a cabo tanto en obtener certificado como en verificar certificado. Debido a que son funciones get de FTP.




El código \ref{cod:conexionFTPput} muestra la implementación necesaria para poder conectar al servidor FTP y guardar certificados en el mismo. El usuario y contraseña son los mismos.\\

Esta implementación se llevará a cabo tanto en guardar usuario como en revocar certificado. Debido a que son funciones get de FTP.





\section{Componente III: API.}

\subsection{Nuevo Proceso de Winnowing y eliminación del descifrado AES.}

Debido  a  los  cambios  realizados  con  el  patrón  en  el  Componente  I  es necesario realizar la adaptación pertinente a este proceso considerando que la parte encargada de este proceso involucra completamente a la API, por lo tanto el código del proceso de winnowing quedó como se muestra en el siguiente fragmento de código: 


En donde nos podemos percatar de la eliminación por un lado del decode en base 64 y del descifrado del patrón mediante AES y por otro de la nueva implementación del patrón y el ajuste para obtener el mismo resultado eliminando algunos pasos. 

\subsection{Guardado del certificado}

Algo muy importante a considerar en este prototipo es que de ahora en adelante el único que almacena el certificado del usuario es el mismo usuario en su extensión y el servidor autentificador, por lo que el servicio web solo pasa a recibir un código SHA256 del mismo certificado, cuidando siempre la confidencialidad de este dato. Esta implementación de manera similar a la anterior se realiza en la API en estas simples lineas de código:

Es importante aclarar que esta salida solo se obtiene mediante el mismo proceso de validación que se mencionó en el Prototipo 1 de este Componente, de la misma manera se sigue regresando el status como se había analizado desde un inicio.
\end{comment}	           
%   %   %   %   %   %   %   %   %   %
%		        Capítulo 10					%
%   			PRUEBAS 			      %
%                               				  %
%   %   %   %   %   %   %   %   %	%	
\begin{comment}
\chapter{\textcolor{azulescom}{Pruebas.}}
Las pruebas que aquí se presentan fueron realizadas con los equipos de la sección de Hardware en Herramientas a Usar, sobre una red local creada con un celular Huawei P20, cuya velocidad de conexión es de 76Mb/s, con una intensidad de señal catalogada como 'Excelente', es decir, los equipos estaban dentro de un radio de 3m del punto de conexión (Huawei P20). 

\section{Pruebas de integración.}

En esta sección se mostrará las pruebas de comunicación entre los distintos componentes, comprobando así que los datos enviados entre uno y otro sean los indicados para llevar a cabo un proceso de \textit{Chaffing} y \textit{Winnowing} correctamente, y por ende un inicio de sesión correcto.

\subsection{Extensión y Servidor autentificador}
La comunicación entre el Componente I y el Componente II se establece por medio de una conexión SSL, por que los tanto los datos de inicio de sesión como el certificado viajaran de forma segura. La figura \ref{fig:WiresharkC1_C2} muestra los datos recibidos por el componente 1, el cual es el certificado. Cabe destacar que los datos que se muestran están planos ya que los datos se descifran y se muestran en el apartado \textit{Network} de Google Chrome.

\begin{figure}[H]
\centering

\caption{Respuesta de la Autoridad Certificadora hacia la Extensión}
\label{fig:WiresharkC1_C2}
\end{figure}

Por otra parte al ver el mensaje de la figura \ref{fig:sesionIniciadaExito} se puede decir que la conexión se estableció satisfactoriamente y se obtuvo el certificado.

\begin{figure}[H]
\centering 

\caption{Sesión iniciada con éxito (Obtención de certificado de la Autoridad certificadora.)}    	 \label{fig:sesionIniciadaExito}
\end{figure}

\subsection{Extensión y Servicio Web}

La comunicación entre el Componente I y Componente III es donde viajará el \textit{Chaffing}, esta parte es importante debido a que el Chaffing viaja en el encabezado de la petición, y para saber que esta viajando en red hicimos uso del analizador de protocolos \textbf{Wireshark}. La figura \ref{fig:Wireshark_C1_C3} mustra la petición en wireshark donde se puede ver el apartado de \textbf{Chaffing} y el apartado de \textbf{Pattern}.

\begin{figure}[H]
\centering 

\caption{Analizador de protocolos Wireshark de la petición dada por el usuario hacia servicio web.)}    	 \label{fig:Wireshark_C1_C3}
\end{figure}

\subsection{Servidor Autentificador y API}

La comunicación entre el Componente II y Componente III sirve para verificar el certificado, debido a que la API (después de obtener el Winnowing) hace una petición a la autoridad certificadora para verificar si el certificado que obtuvo aún no ha sido revocado. En la figura \ref{fig:Wireshark_C2_C3} muestra desde el analizador de protocolos \textit{Wireshark} la respuesta que le manda la autoridad certificadora a la API.

\begin{figure}[H]
\centering 

\caption{Analizador de protocolos Wireshark de la petición dada por la API hacia la autoridad certificadora.)}    	 
\label{fig:Wireshark_C2_C3}
\end{figure}


\section{Pruebas funcionales.}
Realizamos pruebas del funcionamiento de nuestro prototipo 2, para asegurarnos que sean correctas las entradas y salidas de nuestro sistema para este prototipo. Empezaremos con la parte del usuario, donde comprobaremos que la extensión está recibiendo el certificado del usuario de manera correcta al iniciar sesión. 

\begin{figure}[H]
\centering

\caption{Inicio de sesión en la extensión}
\label{fig:pruebaInicioSesionUI}
\end{figure}

\begin{figure}[H]
\centering

\caption{Certificado del cliente devuelto por el Servidor Autentificador}
\label{fig:certClienteRed}
\end{figure}

Si el usuario logra obtener su certificado, entonces puede proceder a utilizar la extensión, vamos ahora a ingresar a nuestro Servicio Web de prueba, la página de la veterinaria. Si es la primera vez que accedemos, entonces debemos de registrarnos para crear nuestra cuenta de usuario en este servicio, y después iniciar sesión de manera rutinaria con nuestras credenciales, esto para que la extensión pueda asociar el certificado de la extensión de este usuario a esta cuenta, y así cuando vuelva a acceder a este mismo Servicio Web, nuestro sistema le dará acceso. 

\begin{figure}[H]
\centering

\caption{Página de inicio del Servicio Web}
\label{fig:pruebaHomeUser}
\end{figure}

\begin{figure}[H]
\centering

\caption{Inicio de sesión con la extensión habilitada.}
\label{fig:pruebasLoginChaffing}
\end{figure}

Como podemos observar, nuestro inicio de sesión en la extensión nos pide que iniciemos sesión para vincular las credenciales, si este mensaje aparece quiere decir que todo va bien, porque la extensión detectó que es la primera vez que se inicia sesión en el Servicio Web, por lo que requiere de tu inicio de sesión para poder asignarle este usuario al certificado que va a ir inyectado con \textit{Chaffing} en la petición al servidor.\\

Cuando ingresemos nuestras credenciales, vamos a probar cerrando sesión en el Servicio Web y volviendo a entrar a la misma: 
\begin{figure}[H]
\centering

\caption{Inicio de sesión exitoso a la cuenta con el certificado asignado.}
\label{fig:pruebasLoginHomeUser}
\end{figure}
Como podemos darnos cuenta, al querer volver a iniciar sesión nos da acceso de inmediato a nuestra cuenta para este Servicio Web, debido a que anteriormente se había asignado el mismo certificado a la cuenta que se le dio acceso.


\section{Tiempos de ejecución.}
En esta sección mostraremos el tiempo de ejecución de los algoritmos desarrollados para el Prototipo 2, así como la comparación con el Prototipo 1 para demostrar la mejoría que hubo. Todas las pruebas fueron realizadas en igualdad de condiciones, es decir, la computadora recién prendida (sin ningún programa abierto) y sólo ejecutando el algoritmo.\\

\subsection{Algoritmo de Chaffing.}

Se realizó una prueba al algoritmo de \textit{Chaffing}, el cual incluye: creación de patrón, proceso de chaffing y cifrado de patrón, todo del prototipo 2. Para la prueba se utilizó una función de JavaScript llamada \textit{performance.now()}, la cual mide el tiempo con una precisión de milisegundos.\\

\paragraph{Tiempo de ejecución: } 17.43 ms.\\

Si comparamos el tiempo de ejecución del algoritmo de \textit{Chaffing} del Prototipo 1, el cual era de 434.2925 ms, podemos ver que la mejoría fue de 416.8625ms gracias, principalmente, a la disminución del tamaño del patrón (que afecta directamente a la manera en que se crea y en que se recorre para hacer el Chaffing) y a la eliminación del cifrado por bloques AES.

\subsection{Algoritmo de Winnowing.}

Se realizó una prueba al algoritmo de \textit{Winnowing}, el cual incluye: descifrado del patrón y proceso de winnowing. Para la prueba se utilizó una función de Java llamada \textit{System.currentTimeMillis()}, la cual mide el tiempo con una precisión de milisegundos.\\

\paragraph{Tiempo de ejecución: } 12.46 ms.\\

Si comparamos el tiempo de ejecución del algoritmo de \textit{Winnowing} del Prototipo 1, el cual era de 53.99ms, podemos ver que la mejoría fue de 41.53ms gracias a la disminución del tamaño del patrón y a la eliminación del descifrado por bloques AES.

\subsection{Inicio de sesión.}

Se realizaron un total de 100 pruebas del inicio de sesión por medio de este Trabajo Terminal. Este proceso incluye el tiempo que transcurre desde que el usuario da click en el botón de iniciar sesión en el Servicio Web, es decir, la interceptación de la petición por parte del Componente I) hasta la impresión de la respuesta del Servicio Web, es decir, el inicio de sesión completado exitosamente mostrando la pantalla de inicio del Servicio Web.
Para medir el tiempo, se utilizó la función de JavaScript llamada \textit{performance.now()}, al igual que en la medición del algoritmo de Chaffing, puesto que esta medición se realiza desde el Componente I. Extensión.\\
El tiempo que se muestra a continuación es el promedio de 100 inicios de sesión.

\paragraph{Tiempo promedio de ejecución: } 335.28 ms.\\

Si comparamos el tiempo de inicio de sesión del Prototipo 1, el cual era de 1101.68ms, podemos ver que la mejoría fue de 766.4ms con lo cual se comprueba que el Prototipo 2 es una mejora sustancial al sistema desarrollado en este trabajo sin la pérdida de seguridad.

%   %   %   %   %   %   %   %   %   %
%		        Capítulo 11					%
%   			CONCLUSIONES 		%
%                               				  %
%   %   %   %   %   %   %   %   %	%
\chapter{\textcolor{azulescom}{Conclusiones}}


%   %   %   %   %   %   %   %   %   %
%		        Capítulo 12					%
%   	TRABAJO A FUTURO		  %
%                               				  %
%   %   %   %   %   %   %   %   %	%
\chapter{\textcolor{azulescom}{Trabajo a futuro.}}

\newpage


\end{comment}

%\end{document}


